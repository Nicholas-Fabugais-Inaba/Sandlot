\documentclass{article}

\usepackage{float}
\restylefloat{table}

\usepackage{booktabs}

\title{Team Contributions: POC\\\progname}

\author{\authname}

\date{}

\input{../Comments}
\input{../Common}

\begin{document}

\maketitle

This document summarizes the contributions of each team member up to the POC
Demo.  The time period of interest is the time between the beginning of the term
and the POC demo.

\section{Demo Plans}

To be a successful project, the project should be able to intake a set of 
availability data and generate an optimized schedule based on that data. 
Additionally, it should be able to display this schedule through a basic
user interface. The main risks for the success of this project are whether 
we are able to optimize schedule generation to a point which satisfies our 
stakeholders, and our team's lack of domain knowledge on scheduling problems 
and web development. \\

In order to demonstrate that these obstacles can be 
overcome, the goal of this proof of concept demonstration is to develop
an algorithm to generate highly optimized schedules based on team availability 
data. Furthermore these optimized schedules should be displayed through a 
basic UI hosted on a webvserver.

Demonstration Plan:
\begin{itemize}
  \item Quick overview of the POC demonstration
    \subitem Scheduling algorithm and UI visualizer
  \item Explanation of scheduling algorithm
    \subitem Problem formulation
    \subitem Algorithm to solve
    \subitem Algorithm details and other notes
  \item Demonstration of POC
    \subitem Show UI
    \subitem Generate schedule
    \subitem Show that each constraint is satisfied
\end{itemize}

\section{Team Meeting Attendance}

\begin{table}[H]
\centering
\begin{tabular}{ll}
\toprule
\textbf{Student} & \textbf{Meetings}\\
\midrule
Total & 5\\
Nicholas Fabugais-Inaba & 4 \\
Alex Verity & 4 \\
Jung Woo Lee & 5 \\
Casra Ghazanfari & 5 \\
\bottomrule
\end{tabular}
\end{table}

The total for the number of meetings attended could be seen as
being understated for all members. This is because the team has 
had many minor impromptu meetings both online in person which 
did not have specific agendas or were simply to complete more work
without discussion. These minor meetings were not deemed significant
enough for team meeting issues to be created for them.

\section{Supervisor/Stakeholder Meeting Attendance}

\begin{table}[H]
\centering
\begin{tabular}{ll}
\toprule
\textbf{Student} & \textbf{Meetings}\\
\midrule
Total & 4\\
Nicholas Fabugais-Inaba & 4 \\
Alex Verity & 4 \\
Jung Woo Lee & 4 \\
Casra Ghazanfari & 4 \\
\bottomrule
\end{tabular}
\end{table}

\section{Lecture Attendance}

\begin{table}[H]
\centering
\begin{tabular}{ll}
\toprule
\textbf{Student} & \textbf{Lectures}\\
\midrule
Total & 9\\
Nicholas Fabugais-Inaba & 9 \\
Alex Verity & 7 \\
Jung Woo Lee & 8 \\
Casra Ghazanfari & 8 \\
\bottomrule
\end{tabular}
\end{table}

\section{TA Document Discussion Attendance}

\begin{table}[H]
\centering
\begin{tabular}{ll}
\toprule
\textbf{Student} & \textbf{Meetings}\\
\midrule
Total & 3\\
Nicholas Fabugais-Inaba & 3 \\
Alex Verity & 3 \\
Jung Woo Lee & 3 \\
Casra Ghazanfari & 3 \\
\bottomrule
\end{tabular}
\end{table}

\section{Commits}

\begin{table}[H]
\centering
\begin{tabular}{lll}
\toprule
\textbf{Student} & \textbf{Commits} & \textbf{Percent}\\
\midrule
Total & 236 & 100\% \\
Nicholas Fabugais-Inaba & 48 & 20\% \\
Alex Verity & 85 & 36\% \\
Jung Woo Lee & 59 & 25\% \\
Casra Ghazanfari & 44 & 19\% \\
\bottomrule
\end{tabular}
\end{table}

Alex's commits are higher as we have worked off his computer together as a team.

\section{Issue Tracker}

\begin{table}[H]
\centering
\begin{tabular}{lll}
\toprule
\textbf{Student} & \textbf{Authored (O+C)} & \textbf{Assigned (C only)}\\
\midrule
Nicholas Fabugais-Inaba & 39 & 14 \\
Alex Verity & 0 & 15 \\
Jung Woo Lee & 10 & 13 \\
Casra Ghazanfari & 22 & 14 \\
\bottomrule
\end{tabular}
\end{table}

Nicholas has created all the issues for the deliverable documents and Casra has created
most TA Feedback issues.

\section{CICD}

\begin{itemize}

\item Continuous Integration
\subitem Continuous integration will be implemented via GitHub Actions. When a pull request is made, automated
tests will run against the code being tested.
\item Unit testing frameworks
\subitem The pytest framework will be used to create unit tests for the middleware
code. We chose pytest over other python testing frameworks due to its simplicity, 
small amount of boilerplate code, and plugins which can add useful functionalities
like coverage reporting. We plan to incorporate these pytest unit tests as a part
of our CI plans for the project via Github actions.
\subitem Testing the database will likely be done using a dummy / development 
PostgreSQL database prior to making any changes to the production database to 
ensure that minimal migrations are required during development.
\item Code coverage measuring tools
\subitem The Coverage.py Python library will be used to measure the code coverage
of our middleware program. For the webserver's React code, Jest is included by 
default when using the "create-react-app" command and will be used to measure 
the code coverage of the webserver.

\end{itemize}

\end{document}