\documentclass{article}

\usepackage{float}
\restylefloat{table}

\usepackage{booktabs}

\title{Team Contributions: POC\\\progname}

\author{\authname}

\date{}

\input{../Comments}
%% Common Parts

\newcommand{\progname}{Sandlot} % PUT YOUR PROGRAM NAME HERE
\newcommand{\authname}{Team 29
\\ Nicholas Fabugais-Inaba
\\ Casra Ghazanfari
\\ Alex Verity
\\ Jung Woo Lee} % AUTHOR NAMES                  

\usepackage{hyperref}
    \hypersetup{colorlinks=true, linkcolor=blue, citecolor=blue, filecolor=blue,
                urlcolor=blue, unicode=false}
    \urlstyle{same}
                                


\begin{document}

\maketitle

This document summarizes the contributions of each team member up to the POC
Demo.  The time period of interest is the time between the beginning of the term
and the POC demo.

\section{Demo Plans}

\wss{What will you be demonstrating}

To be a successful project, the project should be able to intake a set of availabil-
ity data and generate an optimized schedule based on that data. Additionally,
it should be able to display this schedule through a basic user interface. The
main risks for the success of this project are whether we are able to optimize
schedule generation to a point which satisfies our stakeholders, and our team’s
lack of domain knowledge on scheduling systems and web development. In or-
der to demonstrate that these obstacles can be overcome, the goal of this proof
of concept demonstration is to develop an algorithm to generate highly opti-
mized schedules based on team availability data. Furthermore these optimized
schedules should be displayed through a basic UI hosted on a webvserver.

\section{Team Meeting Attendance}

\wss{For each team member how many team meetings have they attended over the
time period of interest.  This number should be determined from the meeting
issues in the team's repo.  The first entry in the table should be the total
number of team meetings held by the team.}

\begin{table}[H]
\centering
\begin{tabular}{ll}
\toprule
\textbf{Student} & \textbf{Meetings}\\
\midrule
Total & Num\\
Name 1 & Num\\
Name 2 & Num\\
Name 3 & Num\\
Name 4 & Num\\
Name 5 & Num\\
\bottomrule
\end{tabular}
\end{table}

\wss{If needed, an explanation for the counts can be provided here.}

\section{Supervisor/Stakeholder Meeting Attendance}

\wss{For each team member how many supervisor/stakeholder team meetings have
they attended over the time period of interest.  This number should be determined
from the supervisor meeting issues in the team's repo.  The first entry in the
table should be the total number of supervisor and team meetings held by the
team.  If there is no supervisor, there will usually be meetings with
stakeholders (potential users) that can serve a similar purpose.}

\begin{table}[H]
\centering
\begin{tabular}{ll}
\toprule
\textbf{Student} & \textbf{Meetings}\\
\midrule
Total & 4\\
Nicholas Fabugais-Inaba & 4 \\
Alex Verity & 4 \\
Jung Woo Lee & 4 \\
Casra Ghazanfari & 4 \\
\bottomrule
\end{tabular}
\end{table}

\wss{If needed, an explanation for the counts can be provided here.}

\section{Lecture Attendance}

\wss{For each team member how many lectures have they attended over the time
period of interest.  This number should be determined from the lecture issues in
the team's repo.  The first entry in the table should be the total number of
lectures since the beginning of the term.}

\begin{table}[H]
\centering
\begin{tabular}{ll}
\toprule
\textbf{Student} & \textbf{Lectures}\\
\midrule
Total & 9\\
Nicholas Fabugais-Inaba & 9 \\
Alex Verity & 7 \\
Jung Woo Lee & 8 \\
Casra Ghazanfari & 8 \\
\bottomrule
\end{tabular}
\end{table}

\wss{If needed, an explanation for the lecture attendance can be provided here.}

\section{TA Document Discussion Attendance}

\wss{For each team member how many of the informal document discussion meetings
with the TA were attended over the time period of interest.}

\begin{table}[H]
\centering
\begin{tabular}{ll}
\toprule
\textbf{Student} & \textbf{Meetings}\\
\midrule
Total & 3\\
Nicholas Fabugais-Inaba & 3 \\
Alex Verity & 3 \\
Jung Woo Lee & 3 \\
Casra Ghazanfari & 3 \\
\bottomrule
\end{tabular}
\end{table}

\wss{If needed, an explanation for the attendance can be provided here.}

\section{Commits}

\wss{For each team member how many commits to the main branch have been made
over the time period of interest.  The total is the total number of commits for
the entire team since the beginning of the term.  The percentage is the
percentage of the total commits made by each team member.}

\begin{table}[H]
\centering
\begin{tabular}{lll}
\toprule
\textbf{Student} & \textbf{Commits} & \textbf{Percent}\\
\midrule
Total & 236 & 100\% \\
Nicholas Fabugais-Inaba & 48 & 20\% \\
Alex Verity & 85 & 36\% \\
Jung Woo Lee & 59 & 25\% \\
Casra Ghazanfari & 44 & 19\% \\
\bottomrule
\end{tabular}
\end{table}

\wss{If needed, an explanation for the counts can be provided here.  For
instance, if a team member has more commits to unmerged branches, these numbers
can be provided here.  If multiple people contribute to a commit, git allows for
multi-author commits.}

Alex's commits are higher as we have worked off his computer together as a team.

\section{Issue Tracker}

\wss{For each team member how many issues have they authored (including open and
closed issues (O+C)) and how many have they been assigned (only counting closed
issues (C only)) over the time period of interest.}

\begin{table}[H]
\centering
\begin{tabular}{lll}
\toprule
\textbf{Student} & \textbf{Authored (O+C)} & \textbf{Assigned (C only)}\\
\midrule
Nicholas Fabugais-Inaba & 39 & 14 \\
Alex Verity & 0 & 15 \\
Jung Woo Lee & 10 & 13 \\
Casra Ghazanfari & 22 & 14 \\
\bottomrule
\end{tabular}
\end{table}

\wss{If needed, an explanation for the counts can be provided here.}

Nicholas has created all the issues for the deliverable documents and Casra has created
most TA Feedback issues.

\section{CICD}

\wss{Say how CICD will be used in your project}

\wss{If your team has additional metrics of productivity, please feel free to
add them to this report.}
\begin{itemize}

\item CI
\subitem Continuous integration will be implemented via GitHub Actions. When a pull request is made, automated
tests will run against the code being tested.
\item Unit testing frameworks
\subitem The pytest framework will be used to create unit tests for the middleware
code. We chose pytest over other python testing frameworks due to its simplicity, 
small amount of boilerplate code, and plugins which can add useful functionalities
like coverage reporting. We plan to incorporate these pytest unit tests as a part
of our CI plans for the project via Github actions.
\subitem Testing the database will likely be done using a dummy / development 
PostgreSQL database prior to making any changes to the production database to 
ensure that minimal migrations are required during development.
\item Code coverage measuring tools
\subitem The Coverage.py Python library will be used to measure the code coverage
of our middleware program. For the webserver's React code, Jest is included by 
default when using the "create-react-app" command and will be used to measure 
the code coverage of the webserver.

\end{itemize}

\end{document}