\documentclass{article}

\usepackage{tabularx}
\usepackage{booktabs}

\title{Problem Statement and Goals\\\progname}

\author{\authname}

\date{}

\input{../Comments}
%% Common Parts

\newcommand{\progname}{Sandlot} % PUT YOUR PROGRAM NAME HERE
\newcommand{\authname}{Team 29
\\ Nicholas Fabugais-Inaba
\\ Casra Ghazanfari
\\ Alex Verity
\\ Jung Woo Lee} % AUTHOR NAMES                  

\usepackage{hyperref}
    \hypersetup{colorlinks=true, linkcolor=blue, citecolor=blue, filecolor=blue,
                urlcolor=blue, unicode=false}
    \urlstyle{same}
                                


\begin{document}

\maketitle

\begin{table}[hp]
\caption{Revision History} \label{TblRevisionHistory}
\begin{tabularx}{\textwidth}{llX}
\toprule
\textbf{Date} & \textbf{Developer(s)} & \textbf{Change}\\
\midrule
September 23, 2024 & NFI, JL, CG, AV & Initial Draft\\
Date2 & Name(s) & Description of changes\\
... & ... & ...\\
\bottomrule
\end{tabularx}
\end{table}

\section{Problem Statement}

\wss{You should check your problem statement with the
\href{https://github.com/smiths/capTemplate/blob/main/docs/Checklists/ProbState-Checklist.pdf}
{problem statement checklist}.} 

\wss{You can change the section headings, as long as you include the required
information.}

\subsection{Problem}

\subsection{Inputs and Outputs}

\subsubsection{Inputs}

\begin{itemize}
    \item Student course feedback
    \item item 2
    \item item 3
\end{itemize}

\subsubsection{Outputs}

\begin{itemize}
    \item item 1
    \item item 2
    \item item 3
\end{itemize}

\wss{Characterize the problem in terms of ``high level'' inputs and outputs.  
Use abstraction so that you can avoid details.}

\subsection{Stakeholders}

\begin{itemize}
    \item Course instructors
    \item Data scientists
    \item Educational research experts
    \item Administrators
\end{itemize}

\subsection{Environment}

\begin{description}
    \item [Software] Course instructors
    \item [Hardware] Data scientists
\end{description}

\wss{Hardware and software environment}

\section{Goals}

\begin{description}
    \item [Accomplish everything the existing league website does]
    The current website allows captains to log in and record their matches
    and scores. It allows scheduling and rescheduling, and provides a place
    to see the league rules, parking information and other information. The
    current website often breaks, requiring the current website admin to fix
    issues as they arise. First and foremost we need to recreate the original
    league website functionality.
    \item [User interface should be intuitive to all users.] The current
    interface is unintuiative and awkward to use. Users should understand
    how to log in and how to view their schedule just by looking at their
    homepage. No external information should be required.
    \item [Allow players to make accounts] Currently, only captains have
    accounts in the system. Player accounts should only be able to view the
    contact information of their team captain, captains should only be able
    to view the contact of their players and other captains, and commisioners
    should be able to see everything.
    \item [Matches should be able to be scheduled and rescheduled.] 
    Team captains should be able to give their team's availablity and the
    software will algorithmically schedule the season's matches. If a team
    isn't available for a match after it has been scheduled, captains can
    send a reschedule request with a selection of possible alternative times
    that the opposition team's captain can agree to.
    \item [Commisioners should be able to notify captains with information]
    Commisioner level accounts should be able to easily send out a
    notification to specific users or entire groups of users, such as all
    captains or all players. The information in the notification should be
    customizable by the commisioner.
\end{description}

\section{Stretch Goals}

\begin{description}
    \item [Commisioners should be able to "rain out" matches] After a match
    has been scheduled, commisioner level accounts should be able to force
    a reschedule if the weather makes the game unreasonable to play. This
    will send a notification to the two team captains so they can
    choose a date that works.
    \item [League template saving] A season's teams and players should be
    able to be saved as a template that can be loaded the next season. This
    is useful as many teams remain the same or similar between seasons, and
    it would be convienient for all returning teams to avoid reinviting all
    returning players.
    \item [A mobile application companion] Users would be able to perform
    some actions they can on the website, like viewing schedules and
    standings.
\end{description}

\section{Challenge Level and Extras}

\wss{State your expected challenge level (advanced, general or basic).  The
challenge can come through the required domain knowledge, the implementation or
something else.  Usually the greater the novelty of a project the greater its
challenge level.  You should include your rationale for the selected level.
Approval of the level will be part of the discussion with the instructor for
approving the project.  The challenge level, with the approval (or request) of
the instructor, can be modified over the course of the term.}

\wss{Teams may wish to include extras as either potential bonus grades, or to
make up for a less advanced challenge level.  Potential extras include usability
testing, code walkthroughs, user documentation, formal proof, GenderMag
personas, Design Thinking, etc.  Normally the maximum number of extras will be
two.  Approval of the extras will be part of the discussion with the instructor
for approving the project.  The extras, with the approval (or request) of the
instructor, can be modified over the course of the term.}

\newpage{}

\section*{Appendix --- Reflection}

\wss{Not required for CAS 741}

\input{../Reflection.tex}

\begin{enumerate}
    \item What went well while writing this deliverable? 
    \item What pain points did you experience during this deliverable, and how
    did you resolve them?
    \item How did you and your team adjust the scope of your goals to ensure
    they are suitable for a Capstone project (not overly ambitious but also of
    appropriate complexity for a senior design project)?
\end{enumerate}  

\end{document}