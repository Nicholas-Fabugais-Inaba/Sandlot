\documentclass{article}

\usepackage{tabularx}
\usepackage{booktabs}

\title{Problem Statement and Goals\\\progname}

\author{\authname}

\date{}

\input{../Comments}
%% Common Parts

\newcommand{\progname}{Sandlot} % PUT YOUR PROGRAM NAME HERE
\newcommand{\authname}{Team 29
\\ Nicholas Fabugais-Inaba
\\ Casra Ghazanfari
\\ Alex Verity
\\ Jung Woo Lee} % AUTHOR NAMES                  

\usepackage{hyperref}
    \hypersetup{colorlinks=true, linkcolor=blue, citecolor=blue, filecolor=blue,
                urlcolor=blue, unicode=false}
    \urlstyle{same}
                                


\begin{document}

\maketitle

\begin{table}[hp]
\caption{Revision History} \label{TblRevisionHistory}
\begin{tabularx}{\textwidth}{llX}
\toprule
\textbf{Date} & \textbf{Developer(s)} & \textbf{Change}\\
\midrule
September 23, 2024 & NFI, JL, CG, AV & Initial Draft\\
Date2 & Name(s) & Description of changes\\
... & ... & ...\\
\bottomrule
\end{tabularx}
\end{table}

\section{Problem Statement}

\subsection{Background}

\wss{You should check your problem statement with the
\href{https://github.com/smiths/capTemplate/blob/main/docs/Checklists/ProbState-Checklist.pdf}
{problem statement checklist}.} 

\wss{You can change the section headings, as long as you include the required
information.}

The McMaster GSA softball league is used every summer by 30-40 teams and
as many as 1,000 unique participants. The league is currently organized through
an old software platform accessible via a web browser, but it is outdated and
does not include features for administrators to maintain the site without
extensive knowledge of computer programming.

\subsection{Problem}

\indent \indent The GSA league is aware that paid-for and ad-supported services are available,
and features present in those applications should be explored and added if
possible. The GSA league is a minimal-cost non-profit and would like a
personalized platform by which to operate without committing to paid-for
services. The platform will be responsible for including all functionalities
such as scheduling, division management, communication between captains, waiver
management, rescheduling, score and league standings management, and other
tasks identified by the stakeholders. The platform would be a web-based service
with the same functionalities, but in an updated form that also enables league
and schedule management from a convenient user interface
(specific access privileges for team representatives and league administrators).

\subsection{Inputs and Outputs}

\subsubsection{Inputs}

\begin{itemize}
    \item Player/Captain/Commissioner login information
    \item Player/Captain/Commissioner contact information
    \item Team name
    \item Game score
\end{itemize}

\subsubsection{Outputs}

\begin{itemize}
    \item League standings
    \item League scheduling
\end{itemize}

\wss{Characterize the problem in terms of ``high level'' inputs and outputs.  
Use abstraction so that you can avoid details.}

\subsection{Stakeholders}

\begin{itemize}
    \item The supervisor of the project, Dr. Jake Nease
    \item Captains and players of the softball league
\end{itemize}

\subsection{Environment}

\begin{description}
    \item [Software] Windows, Linux or Mac OS
    \item [Hardware] Computers with access to the internet
\end{description}

\wss{Hardware and software environment}

\section{Goals}

\begin{description}
    \item [Goal1] desc
    \item [Goal2] desc
    \item [Goal3] desc
    \item [Goal4] desc
    \item [Goal5] desc
\end{description}

\section{Stretch Goals}

\begin{description}
    \item [Goal1] desc
    \item [Goal2] desc
    \item [Goal3] desc
\end{description}

\section{Challenge Level and Extras}

\wss{State your expected challenge level (advanced, general or basic).  The
challenge can come through the required domain knowledge, the implementation or
something else.  Usually the greater the novelty of a project the greater its
challenge level.  You should include your rationale for the selected level.
Approval of the level will be part of the discussion with the instructor for
approving the project.  The challenge level, with the approval (or request) of
the instructor, can be modified over the course of the term.}

\wss{Teams may wish to include extras as either potential bonus grades, or to
make up for a less advanced challenge level.  Potential extras include usability
testing, code walkthroughs, user documentation, formal proof, GenderMag
personas, Design Thinking, etc.  Normally the maximum number of extras will be
two.  Approval of the extras will be part of the discussion with the instructor
for approving the project.  The extras, with the approval (or request) of the
instructor, can be modified over the course of the term.}

\newpage{}

\section*{Appendix --- Reflection}

\wss{Not required for CAS 741}

\input{../Reflection.tex}

\begin{enumerate}
    \item What went well while writing this deliverable? 
    \item What pain points did you experience during this deliverable, and how
    did you resolve them?
    \item How did you and your team adjust the scope of your goals to ensure
    they are suitable for a Capstone project (not overly ambitious but also of
    appropriate complexity for a senior design project)?
\end{enumerate}  

\end{document}