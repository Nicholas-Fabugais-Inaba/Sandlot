% THIS DOCUMENT IS FOLLOWS THE VOLERE TEMPLATE BY Suzanne Robertson and James Robertson
% ONLY THE SECTION HEADINGS ARE PROVIDED
%
% Initial draft from https://github.com/Dieblich/volere
%
% Risks are removed because they are covered by the Hazard Analysis
\documentclass[12pt]{article}

\usepackage{booktabs}
\usepackage{tabularx}
\usepackage{hyperref}
\usepackage{multicol}
\usepackage[dvipsnames]{xcolor}
\usepackage{enumitem}
\usepackage{tcolorbox}
\usepackage{array}

\hypersetup{
    bookmarks=true,         % show bookmarks bar?
      colorlinks=true,      % false: boxed links; true: colored links
    linkcolor=red,          % color of internal links (change box color with linkbordercolor)
    citecolor=green,        % color of links to bibliography
    filecolor=magenta,      % color of file links
    urlcolor=cyan           % color of external links
}

\newenvironment{myreq}[1]{%
\setlist[description]{font=\normalfont\color{darkgray}}%
\begin{tcolorbox}[colframe=black,colback=white, sharp corners, boxrule=1pt]%
\bfseries\color{blue}%
\begin{description}#1}%
{\end{description}\end{tcolorbox}}

\newcommand{\threeinline}[3]{\begin{multicols}{3}#1 #2 #3\end{multicols}}
\newcommand{\twoinline}[2]{\begin{multicols}{2}#1 #2\end{multicols}}

\newcommand{\reqno}{\item[Requirement \#:]}
\newcommand{\reqtype}{\item[Requirement Type:]}
\newcommand{\reqevent}{\item[Event/BUC/PUC \#:]}
\newcommand{\reqdesc}{\item[Description:]}
\newcommand{\reqrat}{\item[Rationale:]}
\newcommand{\reqorig}{\item[Originator:]}
\newcommand{\reqfit}{\item[Fit Criterion:]}
\newcommand{\reqsatis}{\item[Customer Satisfaction:]}
\newcommand{\reqdissat}{\item[Customer Dissatisfaction:]}
\newcommand{\reqdep}{\item[Dependencies:]}
\newcommand{\reqconf}{\item[Conflicts:]}
\newcommand{\reqmater}{\item[Materials:]}
\newcommand{\reqhist}{\item[History:]}

\newcommand{\lips}{\textit{Insert your content here.}}

\input{../Comments}
%% Common Parts

\newcommand{\progname}{Sandlot} % PUT YOUR PROGRAM NAME HERE
\newcommand{\authname}{Team 29
\\ Nicholas Fabugais-Inaba
\\ Casra Ghazanfari
\\ Alex Verity
\\ Jung Woo Lee} % AUTHOR NAMES                  

\usepackage{hyperref}
    \hypersetup{colorlinks=true, linkcolor=blue, citecolor=blue, filecolor=blue,
                urlcolor=blue, unicode=false}
    \urlstyle{same}
                                


\begin{document}

\title{Software Requirements Specification for \progname: Softball League Scheduling and Management Web Application} 
\author{\authname}
\date{\today}
	
\maketitle

~\newpage

\pagenumbering{roman}

\tableofcontents

~\newpage

\section*{Revision History}

\begin{tabularx}{\textwidth}{p{3cm}p{2cm}X}
\toprule {\textbf{Date}} & {\textbf{Version}} & {\textbf{Notes}}\\
\midrule
Oct. 7, 2024 & 1.0 & TA Feedback\\
Oct. 11, 2024 & 1.1 & Rev0\\
\bottomrule
\end{tabularx}

~\\

~\newpage
\section{Purpose of the Project}

\subsection{User Business}

The McMaster GSA softball league's current scheduling and management platform
is used from the 1st week of May until the last week of August. The website
creates a season schedule based on the 30-40 teams that are entered into the
league by their respective captains. If scheduling conflicts or weather concerns
occur, games are able to be rescheduled by the team captains based on a team's
availability. For the many users interacting with the platform, individuals need
an intuitive interface that is robust and will allow administrators to easily
maintain the system, especially when the website experiences problems. The current
platform lacks the capabilities to provide these functionalities to the players,
captains, and commissioners. With this project, our team is provided an
opportunity to apply our software engineering background to fulfill a
desired need for an upgrade to an outdated website.

\subsection{Goals of the Project}
Our goals with the project are to recreate everything the current website
solution does, with a better user interface and a more stable foundation, so
that future site admins and league commissioners don't have to deal with the
solution breaking or captains/players not understanding how to use the tool.
We also plan to add features such as player accounts to help players view
their schedules, and a standings viewer to see league scores.

\section{Stakeholders}
\subsection{Client}

The client of the project, Dr. Jake Nease, is an active participant in the
McMaster GSA softball league and understands the difficulties associated with
the current scheduling and management platform. The stability and maintainability
concerns with the website are driving factors that contribute to the need for
an improved interface.

\subsection{Customer}

The customers for this project include the players, captains, commissioners,
umpires, and other individuals that may interact with the website. These
individuals require an easy-to-use platform that allows them to seamlessly
enter the website and view the season schedule, whether or not they have an
account created. 

\subsection{Other Stakeholders}

Future website administrators and maintainers have an interest in the
maintainability, learnability of administrative functions, and the robustness
of the website.

\subsection{Hands-On Users of the Project}

\begin{itemize}
  \item [Players]
  \begin{itemize}
    \item User creates an account, logins to their account,
    and requests to join a team.
    \item User needs to know their login details, the team they
    will join, game details from looking at the schedule.
  \end{itemize}
  \item [Captains]
  \begin{itemize}
    \item User creates an account, logins to their account,
    creates a team, and reschedules a game.
    \item User needs to know their login details, game details from
    looking at the schedule, days the team is free to reschedule a game.
  \end{itemize}
  \item [Commissioners]
  \begin{itemize}
    \item User creates an account, logins to their account,
    can send leage alerts, and assign teams to divisions.
    \item User needs to know their login details, and what alerts
    they may need to send to the league.
  \end{itemize}
  \item [Umpires]
  \begin{itemize}
    \item User needs to view the schedule for the game they will
    be the umpire for.
    \item User needs to know the game details from looking at the
    schedule.
  \end{itemize}
  \item [Project Supervisor]
  \begin{itemize}
    \item User will help gather stakeholders to test the product,
    meet with the team to discuss the project, provide insight
    to requirements desired in the product.
    \item User needs to know how to contact the team, which stakeholders
    can be contacted to test the product, what requirements are needed
    to satisfy the needs of the project.
  \end{itemize}
\end{itemize}

\subsection{Personas}

\begin{itemize}
  \item [1.]
  
  Josh Brown is a 26 year old player that has recently joined the McMaster GSA
  softball league and he is unfamiliar with how the website functions. As someone
  who understands how technology works though, he is able to navigate the interface
  quite well. However, some links he interacts with give him a 404 page not found
  error. This aggravates Josh as he just wants to understand certain information
  about the softball league, but the website isn't able to provide it to him
  because the links on the website are faulty or other issues occur. Josh, along
  with many others who are new to the league, may be technically literate, but due
  to the structural integrity of the system, there are many times where users may
  not be able to access certain information because there is either an error or a
  link that leads to nothing.
  \item [2.]
  
  Ken Phillips is a 58 year old captain for his McMaster GSA softball team and he has
  been using the current website for as long as he can remember. Although he is not
  too familiar with how technology works, he is still able to navigate and utilize the
  website's functionalities as he has used them for quite some time now. Unfortunately,
  with the creation of the new website, even though the website has the same existing
  functions as the old system, he is not as familiar with how to navigate the interface
  the same way he has before. Ken and other individuals that may be comfortable and
  familiar with the current outdated platform, need the new website to be easy-to-use,
  especially for people that are either older or not as technically literate.
\end{itemize}

\subsection{Priorities Assigned to Users}

\begin{itemize}
  \item [Key Users]
  \begin{itemize}
    \item Players
    \item Captains
    \item Commissioners
  \end{itemize}
  \item [Secondary Users]
  \begin{itemize}
    \item Umpires
  \end{itemize}
  \item [Unimportant Users]
  \begin{itemize}
    \item Spectators
  \end{itemize}
\end{itemize}

\subsection{User Participation}

Mainly user participation will be for testing the product. This can
be done by many users included below:

\begin{itemize}
  \item Players
  \item Captains
  \item Commissioners
  \item Umpires
  \item Spectators
  \item Project Supervisor
\end{itemize}

Additionally, the project supervisor will provide valuable
insight about the existing system and its capabilities. These
will be used to improve the overall design that will be implemented
in the new system.

\subsection{Maintenance Users and Service Technicians}

The team will be maintaining the product over the development phase until
March 2025. Knowledge transfer will then be handed over to the
project supervisor along with documentation and other information to aid
in the maintaining of the new platform after development is completed.

\section{Mandated Constraints}
\subsection{Solution Constraints}

\subsection{Implementation Environment of the Current System}

\begin{myreq}
  \threeinline
    {\reqno 0}
    {\reqtype 0}
    {\reqevent 0}
  \reqdesc The system must be accessible by the internet.
  \reqrat Users must be able to access all functionalities from
  their device.
  \reqorig Nicholas Fabugais-Inaba
  \reqfit Users can interact with the system when connected to the
  internet.
  \twoinline
    {\reqsatis 5}
    {\reqdissat 5}
  \twoinline
  {\reqdep None}
  {\reqconf None}
  \reqmater
  \reqhist
\end{myreq}

\begin{myreq}
  \threeinline
    {\reqno 0}
    {\reqtype 0}
    {\reqevent 0}
  \reqdesc The system must implement a database.
  \reqrat The system must store information including user login
  information, the season schedule, team composition,
  player/captain/commissioner contact information, and game scores.
  \reqorig Nicholas Fabugais-Inaba
  \reqfit The system is able to store and access information pertaining
  to the league.
  \twoinline
    {\reqsatis 5}
    {\reqdissat 5}
  \twoinline
  {\reqdep None}
  {\reqconf None}
  \reqmater
  \reqhist
\end{myreq}

\subsection{Partner or Collaborative Applications}

Not applicable.

\subsection{Off-the-Shelf Software}
Not applicable.
\subsection{Anticipated Workplace Environment}
Not applicable.
\subsection{Schedule Constraints}

\begin{myreq}
  \threeinline
    {\reqno 0}
    {\reqtype 0}
    {\reqevent 0}
  \reqdesc The project shall be completed before the final demo.
  \reqrat Project deadline is non-negotiable and the product must
  be completed before the final presentation.
  \reqorig Nicholas Fabugais-Inaba
  \reqfit The product is successfully completed before the final demo.
  \twoinline
    {\reqsatis 5}
    {\reqdissat 5}
  \twoinline
  {\reqdep None}
  {\reqconf None}
  \reqmater
  \reqhist
\end{myreq}

\subsection{Budget Constraints}

\begin{myreq}
  \threeinline
    {\reqno 0}
    {\reqtype 0}
    {\reqevent 0}
  \reqdesc The project shall be subject to a \$750 budget.
  \reqrat Resources required for the project must be under a sum total of
  \$750.
  \reqorig Nicholas Fabugais-Inaba
  \reqfit The total amount spent for the project is under \$750.
  \twoinline
    {\reqsatis 5}
    {\reqdissat 5}
  \twoinline
  {\reqdep None}
  {\reqconf None}
  \reqmater
  \reqhist
\end{myreq}

\subsection{Enterprise Constraints}

\begin{myreq}
  \threeinline
    {\reqno 0}
    {\reqtype 0}
    {\reqevent 0}
  \reqdesc The product shall be made available to the project supervisor.
  \reqrat After project completion, the project supervisor will have
  access to the product for future use. 
  \reqorig Nicholas Fabugais-Inaba
  \reqfit The project supervisor must be able to access all functionalities
  of the product.
  \twoinline
    {\reqsatis 5}
    {\reqdissat 5}
  \twoinline
  {\reqdep None}
  {\reqconf None}
  \reqmater
  \reqhist
\end{myreq}

\section{Naming Conventions and Terminology}
\subsection{Glossary of All Terms, Including Acronyms, Used by Stakeholders
involved in the Project}
Sandlot: Management software for a softball baseball league, the software that
is the subject of this document.\\\\
Player: A person who plays on a baseball team in the league. They have an
account on Sandlot and are a member of a team.\\\\
Captain: A person who plays and leads a baseball team in the league. They are
in charge of defining the team's information on Sandlot.\\\\
Team: A name, list of players and record of match scores that represents a
baseball team defined by a captain. Teams are stored on a database on Sandlot.
\\\\
Commissioner: A person who manages the league. They may also play in the
league. Commissioners have top level permissions on Sandlot, they may edit any
team information like player list and past scores.\\\\



\section{Relevant Facts And Assumptions}
\subsection{Relevant Facts}
\begin{itemize}
  \item The current solution is a website with url 
  \url{https://www.gsasoftball.ca/}
  \item There are currently 25-32 teams in the league playing an average of
  100 games a month.
  \item Many users are older and require an intuitive UI to enjoy using the
  site
\end{itemize}

\subsection{Business Rules}
Not applicable

\subsection{Assumptions}
\begin{itemize}
  \item All users will understand how to log in to a website using a username
  and password.
  \item Users will know how a softball league is structured and how it functions.
\end{itemize}

\section{The Scope of the Work}
\subsection{The Current Situation}
It is important to note that we will not be using the existing solution other
than as a feature guide. The current solution is hosted on the web and is
written in PHP. The current login system does not use a username and password.
Only captains can log in, and they are emailed an ASCII code which they use to
access the website to schedule games and submit scores. Commissioners can login
in the same way as captains and can modify schedules, scores and team
compositions as needed. Currently the standings functionality, which would
allow users to view the scores of played games, is not working.
\subsection{The Context of the Work}
\includegraphics[scale=0.6]{6b_context_diagram.png}
\subsection{Work Partitioning}
  \begin{center}
    \begin{tabular}{|m{4cm}|m{4cm}|m{6cm}|}
      \hline
      Event Name & Input and Output & Summary of BUC\\
      \hline
      1. User creates an account & Account Data (in) & A player, captain or
      commissioner enters in a username and password along with account details
      including their name, email, phone number, and gender.\\
      2. User logs in & Login Data (in) & A player, captain or commissioner
      enters their username and password and the system grants them access to
      their account.\\
      3. Captain creates a team & New Team Information (in) & At the
      start of the season, captains can enter team information such as a team
      name. This registers a new team.\\
      4. Player requests to join a team & Team Join Request (in) & At the
      start of the season, players are not assigned to a team and must request
      to join one.\\
      5. Season starts and availability entered & Season Availability Schedule
      (in) & Record the team captain's entered availability schedule. This
      will be used to generate the league schedule.\\
      6. Reschedule request entered & Reschedule Dates (in) & Record
      availability dates the requesting captain entered as alternates for the
      planned date.\\
      7. Reschedule request received & Reschedule Dates (out) & Send the dates
      the captain who sent the request submit to the other team's captain.\\
      8. User navigates to schedule section & Season Schedule (out) & Display
      stored season schedule (if available) to site user.\\
      9. User navigates to contact information & Contact Infromation (out) & Display
      stored contact information (requires specific access) to site user.\\
      10. User submits alert & Alert Content (in) & Commissioners can submit custom
      alerts to send to any chosen users.\\
      11. System sends alert & Alert Content (out) & Send the alert to any user the
      alert must reach.\\
      12. Commissioner inputs admin command & Commissioner Admin Commands (in)
      & Commissioners have the ability to overwrite team composition and
      schedule.\\
      \hline
    \end{tabular}
  \end{center}
\subsection{Specifying a Business Use Case (BUC)}
Not applicable as events are simple and described above
in the work partitioning table.

\subsection{User Type Hierarchy}
\includegraphics[scale=1.1]{business_data_model.png}

\section{Business Data Model and Data Dictionary}
\subsection{Business Data Model}
\lips

\subsection{Data Dictionary}
\lips

\section{The Scope of the Product}
\subsection{Product Boundary}
\lips
\subsection{Product Use Case Table}
\lips
\subsection{Individual Product Use Cases (PUC's)}
\lips

\section{Functional Requirements}
\subsection{Functional Requirements}

\begin{myreq}
  \threeinline
    {\reqno 01}
    {\reqtype 0}
    {\reqevent 7}
  \reqdesc System must display the season schedule and standings to all users
  without login.
  \reqrat Users who don't have a login (e.g. spectators and umpires) will still
  need to access the schedule and standings, so it should be visible to all.
  \reqorig Alex Verity
  \reqfit The schedule and standings shall be viewable without entering a
  username and password.
  \twoinline
    {\reqsatis 3}
    {\reqdissat 3}
  \twoinline
  {\reqdep None}
  {\reqconf None}
  \reqmater
  \reqhist
\end{myreq}

\begin{myreq}
  \threeinline
    {\reqno 02}
    {\reqtype 0}
    {\reqevent 2}
  \reqdesc Captains should be able to create a team which is added to the
  Sandlot database.
  \reqrat Teams are defined by captains, in charge of scheduling and
  recording scores. Captains must be able to define teams at the start of the
  season.
  \reqorig Alex Verity
  \reqfit When captains make a team, it should be added to the Sandlot
  database.
  \twoinline
    {\reqsatis 5}
    {\reqdissat 5}
  \twoinline
  {\reqdep None}
  {\reqconf None}
  \reqmater
  \reqhist
\end{myreq}

\begin{myreq}
  \threeinline
    {\reqno 03}
    {\reqtype 0}
    {\reqevent 2}
  \reqdesc Users should be able to create a new account by providing 
  the necessary information. 
  \reqrat An account strcuture is necessary to be able to change 
  what a user of the system can see/do based on who they are. 
  For example, a player and captain should not be able to see/do 
  the same things or 2 players from different teams should not be 
  able to see/do the same things.
  \reqorig Casra Ghazanfari
  \reqfit When a user provides the necessary information, an account 
  should be created from that information
  \twoinline
    {\reqsatis 5}
    {\reqdissat 5}
  \twoinline
  {\reqdep None}
  {\reqconf None}
  \reqmater
  \reqhist
\end{myreq}

\begin{myreq}
  \threeinline
    {\reqno 04}
    {\reqtype 0}
    {\reqevent 2}
  \reqdesc Users should be able to change their account information 
  by providing the necessary information.
  \reqrat User information does not stay the same forever, 
  therefore the system should have a way for the user to change 
  their information if it ever changes.
  \reqorig Casra Ghazanfari
  \reqfit When a user provides the necessary information, 
  their account information should change.
  \twoinline
    {\reqsatis 5}
    {\reqdissat 5}
  \twoinline
  {\reqdep None}
  {\reqconf None}
  \reqmater
  \reqhist
\end{myreq}

\begin{myreq}
  \threeinline
    {\reqno 05}
    {\reqtype 0}
    {\reqevent 2}
  \reqdesc Users should be able delete their account by providing 
  the necessary information.
  \reqrat If a user wants to quit the league they should be able to
  delete any of their personal information at any time.
  \reqorig Casra Ghazanfari
  \reqfit When a user provides the necessary information, 
  their account should be deleted.
  \twoinline
    {\reqsatis 5}
    {\reqdissat 5}
  \twoinline
  {\reqdep None}
  {\reqconf None}
  \reqmater
  \reqhist
\end{myreq}

\begin{myreq}
  \threeinline
    {\reqno 06}
    {\reqtype 0}
    {\reqevent 8,9}
  \reqdesc Commissioners should be able to send alerts with custom information
  to a specified user or group of user.
  \reqrat Commissioners have the need to notify league members with any new
  information relevant to the league. 
  \reqorig Alex Verity
  \reqfit When a commissioner enters information to alert league members, the
  league members recieve a notification with the relevant information.
  \twoinline
    {\reqsatis 2}
    {\reqdissat 3}
  \twoinline
  {\reqdep None}
  {\reqconf None}
  \reqmater
  \reqhist
\end{myreq}

\begin{myreq}
  \threeinline
    {\reqno 07}
    {\reqtype 0}
    {\reqevent 9}
  \reqdesc Commissioners should be able to update the team information of any
  team, including player list and scores.
  \reqrat Commissioners have the need to easily fix any errors made by users.
  \reqorig Alex Verity
  \reqfit When a commissioner enters team information to be changed, the
  changes are made in the database.
  \twoinline
    {\reqsatis 2}
    {\reqdissat 2}
  \twoinline
  {\reqdep None}
  {\reqconf None}
  \reqmater
  \reqhist
\end{myreq}

\begin{myreq}
  \threeinline
    {\reqno 08}
    {\reqtype 0}
    {\reqevent 3}
  \reqdesc Before the season starts, captains must have the option to give a
  list of days over the season marking their team's availabilty.
  \reqrat Each team will have members who may only be free on certain days of
  the season. This availability will inform the season schedule so that teams
  will have as many people attending each game as possible.
  \reqorig Alex Verity
  \reqfit Before the seaon starts, captains shall be able to view the option to
  enter their availability and once entered, it shall be stored by the system.
  \twoinline
    {\reqsatis 3}
    {\reqdissat 3}
  \twoinline
  {\reqdep None}
  {\reqconf None}
  \reqmater
  \reqhist
\end{myreq}

\begin{myreq}
  \threeinline
    {\reqno 09}
    {\reqtype 0}
    {\reqevent 3}
  \reqdesc Once the season start due date is reached, all captain's
  availability will be used to generate a season schedule.
  \reqrat Once the season starts all users need to know the schedule to know
  when and where to go to games.
  \reqorig Alex Verity
  \reqfit When the season start due date is reached, a season schedule shall
  be displayed on the website.
  \twoinline
    {\reqsatis 4}
    {\reqdissat 5}
  \twoinline
  {\reqdep None}
  {\reqconf None}
  \reqmater
  \reqhist
\end{myreq}

\begin{myreq}
  \threeinline
    {\reqno 10}
    {\reqtype 0}
    {\reqevent 6}
  \reqdesc Captains must be able to request a reschedule for upcoming games
  as long as the game is at least 24 hours in the future. Rescheduling
  involves giving a list of alternate dates which are sent to the opposing
  team's captain to choose a date and accept.
  \reqrat Rescheduling is a core feature of the product, and schedule changes
  less than a day from when they are to occur may be too little warning for
  players to prepare for.
  \reqorig Alex Verity
  \reqfit A captain shall have the option to request a reschedule for games
  only more than 24 hours in the future.
  \twoinline
    {\reqsatis 4}
    {\reqdissat 5}
  \twoinline
  {\reqdep None}
  {\reqconf None}
  \reqmater
  \reqhist
\end{myreq}

\begin{myreq}
  \threeinline
    {\reqno 11}
    {\reqtype 0}
    {\reqevent 3}
  \reqdesc When recieving a reschedule request from another captain, the system should
  prompt the captain to either accept the reshedule request and choose a date or 
  deny the request.
  \reqrat Sometimes a captain may not be able to reschedule a game due to prior commitments
  or some other external factors. Therefore there should be both an option to accept or deny 
  any reshedule request.
  or deny
  \reqorig Casra Ghazanfari
  \reqfit A captain shall have the option to either accepy or deny any request to 
  reschedule a game.
  \twoinline
    {\reqsatis 4}
    {\reqdissat 5}
  \twoinline
  {\reqdep None}
  {\reqconf None}
  \reqmater
  \reqhist
\end{myreq}

\begin{myreq}
  \threeinline
    {\reqno 12}
    {\reqtype 0}
    {\reqevent 3}
  \reqdesc When a captains rechedule request is either accepted or denied the system should
  notify them about about the outcome.
  \reqrat It is important that the sender of the rechedule request is informed about the status of 
  the request quickly
  \reqorig Casra Ghazanfari
  \reqfit A captain who sent a reschedule request shall be notified on the status of the request
  when it is responded to. 
  \twoinline
    {\reqsatis 4}
    {\reqdissat 5}
  \twoinline
  {\reqdep None}
  {\reqconf None}
  \reqmater
  \reqhist
\end{myreq}

\begin{myreq}
  \threeinline
    {\reqno 13}
    {\reqtype 0}
    {\reqevent 9}
  \reqdesc Captains should be able to update the team information of only their
  team, including team name, player list and scores.
  \reqrat Captains are representatives of their teams and should have the ability
  to control information about their team on a high level. However, they should not
  have the same power as the comissioner and therefore should only be able to change 
  information related to their team.
  \reqorig Casra Ghazanfari
  \reqfit A captain shall have the ability to update team information.
  \twoinline
    {\reqsatis 4}
    {\reqdissat 5}
  \twoinline
  {\reqdep None}
  {\reqconf None}
  \reqmater
  \reqhist
\end{myreq}

\begin{myreq}
  \threeinline
    {\reqno 14}
    {\reqtype 0}
    {\reqevent 9}
  \reqdesc Users should not be able to update the team information of any
  team, including team name, player list and scores.
  \reqrat Users are generally low level users and should not have the ability to
  make changes that affect their whole team.
  \reqorig Casra Ghazanfari
  \reqfit A user shall not have the ability to update team information.
  \twoinline
    {\reqsatis 4}
    {\reqdissat 5}
  \twoinline
  {\reqdep None}
  {\reqconf None}
  \reqmater
  \reqhist
\end{myreq}

\begin{myreq}
  \threeinline
    {\reqno 15}
    {\reqtype 0}
    {\reqevent 9}
  \reqdesc The score of a game should not be able to be updated until after
  the it is completed.
  \reqrat The score of a game is unknown until it is completed having the option to 
  update the score of the game before it is completed is useless at best and harmful
  at worst.
  \reqorig Casra Ghazanfari
  \reqfit The ability to update the score of a game will only display after the 
  game has been played.
  \twoinline
    {\reqsatis 4}
    {\reqdissat 5}
  \twoinline
  {\reqdep None}
  {\reqconf None}
  \reqmater
  \reqhist
\end{myreq}

\begin{myreq}
  \threeinline
    {\reqno 16}
    {\reqtype 0}
    {\reqevent 2,3}
  \reqdesc Non-commissioner accounts can only be a member of one team.
  \reqrat To ensure competition is fair and to avoid captains accidentally
  making extranious teams, users are limited to one team each.
  \reqorig Alex Verity
  \reqfit Once a team is created by a captain or a team is joined by a player,
  they shall not be allowed to use the functionality to create a team or join
  a team.
  \twoinline
    {\reqsatis 2}
    {\reqdissat 2}
  \twoinline
  {\reqdep None}
  {\reqconf None}
  \reqmater
  \reqhist
\end{myreq}

\section{Look and Feel Requirements}
\subsection{Appearance Requirements}

\begin{myreq}
  \threeinline
    {\reqno 17}
    {\reqtype 0}
    {\reqevent xx}
  \reqdesc All user input elements should be distinctive such that they can be
  contrasted. 
  \reqrat User input should be clear to all users so users know where to enter
  inputs.
  \reqorig Alex Verity
  \reqfit User input elements shall have a minimum size of 44x44
  pixels, maintain at least a 3:1 contrast ratio with the background.
  \twoinline
    {\reqsatis 2}
    {\reqdissat 2}
  \twoinline
  {\reqdep None}
  {\reqconf None}
  \reqmater
  \reqhist
\end{myreq}

\begin{myreq}
  \threeinline
    {\reqno 17}
    {\reqtype 0}
    {\reqevent xx}
  \reqdesc All user input elements should provide feedback.
  \reqrat Users should be able to know if their inputs are working or not.
  \reqorig Alex Verity
  \reqfit User input elements shall include distinct visual feedback (e.g., color change or shadow) on hover or
  click for clear visibility and interactivity.
  \twoinline
    {\reqsatis 2}
    {\reqdissat 2}
  \twoinline
  {\reqdep None}
  {\reqconf None}
  \reqmater
  \reqhist
\end{myreq}

\begin{myreq}
  \threeinline
    {\reqno 18}
    {\reqtype 0}
    {\reqevent xx}
  \reqdesc All images and visuals made for/by Sandlot should be high quality.
  \reqrat Low quality images and visuals are unprofessional, and the solution
  should appear professional when possible. Only images made for/by Sandlot
  are included in this requirement as older photos or other visuals may need
  to be displayed that don't meet this standard.
  \reqorig Alex Verity
  \reqfit All images made for/by Sandlot shall be free of pixelation or
  blurring at their displayed size.
  \twoinline
    {\reqsatis 2}
    {\reqdissat 2}
  \twoinline
  {\reqdep None}
  {\reqconf None}
  \reqmater
  \reqhist
\end{myreq}

\begin{myreq}
  \threeinline
    {\reqno 19}
    {\reqtype 0}
    {\reqevent xx}
  \reqdesc Navigation should be straightforward, with menus and links easily
  accessible and readable.
  \reqrat Users should know what section of the site they will be accessing
  when they click on a navigation option so they don't get lost or confused.
  \reqorig Alex Verity
  \reqfit Navigation options shall be consistently placed across pages and
  their destination should be visibly written.
  \twoinline
    {\reqsatis 2}
    {\reqdissat 2}
  \twoinline
  {\reqdep None}
  {\reqconf None}
  \reqmater
  \reqhist
\end{myreq}

\subsection{Style Requirements}

\begin{myreq}
  \threeinline
    {\reqno 20}
    {\reqtype 0}
    {\reqevent xx}
  \reqdesc The solution should use a consistent style across the entire user
  interface.
  \reqrat To ensure the professionalism of the solution, the style should
  feel unified and consistent to all users.
  \reqorig Alex Verity
  \reqfit All interface elements shall use consistent fonts, colors, layout
  grids, and spacing, ensuring a cohesive visual style and branding.
  \twoinline
    {\reqsatis 2}
    {\reqdissat 2}
  \twoinline
  {\reqdep None}
  {\reqconf None}
  \reqmater
  \reqhist
\end{myreq}

\section{Usability and Humanity Requirements}
\subsection{Ease of Use Requirements}

\begin{myreq}
  \threeinline
    {\reqno 21}
    {\reqtype 0}
    {\reqevent 7}
  \reqdesc All users must be able to easily find the season schedule.
  \reqrat Many users of Sandlot will not be using it often, and the
  schedule will be one of the most frequented parts of Sandlot. It must be
  easy to find and access. 
  \reqorig Alex Verity
  \reqfit On average, a new user shall not take more than one minute to find
  the schedule, and it should not take more than 2 clicks to access.
  \twoinline
    {\reqsatis 3}
    {\reqdissat 5}
  \twoinline
  {\reqdep None}
  {\reqconf None}
  \reqmater
  \reqhist
\end{myreq}

\subsection{Personalization and Internationalization Requirements}
\lips
\subsection{Learning Requirements}
\lips
\subsection{Understandability and Politeness Requirements}
\lips
\subsection{Accessibility Requirements}

\begin{myreq}
  \threeinline
    {\reqno 22}
    {\reqtype 0}
    {\reqevent xx}
  \reqdesc The fonts used should be readable by all users.
  \reqrat Sandlot will have a wide variety of users, and we must make sure the
  font is an appropriate size for those with reduced vision.
  \reqorig Alex Verity
  \reqfit Body text shall have a minimum font size of 16 pixels, a contrast
  ratio of at least 4.5:1, and a line length between 45 and 75 characters for
  optimal readability across all devices.
  \twoinline
    {\reqsatis 3}
    {\reqdissat 3}
  \twoinline
  {\reqdep None}
  {\reqconf None}
  \reqmater
  \reqhist
\end{myreq}

\section{Performance Requirements}
\subsection{Speed and Latency Requirements}

\begin{myreq}
  \threeinline
    {\reqno 23}
    {\reqtype 0}
    {\reqevent xx}
  \reqdesc Solution must load quickly and provide smooth user interactions,
  ensuring minimal delays when accessing content.
  \reqrat Sandlot will need to be a solution users are not frustrated by to
  encourage use, and long load times are frustrating.
  \reqorig Alex Verity
  \reqfit The website shall load in under 3 seconds on a stable internet
  connection.
  \twoinline
    {\reqsatis 3}
    {\reqdissat 3}
  \twoinline
  {\reqdep None}
  {\reqconf None}
  \reqmater
  \reqhist
\end{myreq}

\subsection{Safety-Critical Requirements}
\lips
\subsection{Precision or Accuracy Requirements}
\lips
\subsection{Robustness or Fault-Tolerance Requirements}
\lips
\subsection{Capacity Requirements}
\lips
\subsection{Scalability or Extensibility Requirements}
\lips
\subsection{Longevity Requirements}
\lips

\section{Operational and Environmental Requirements}
\subsection{Expected Physical Environment}
\lips
\subsection{Wider Environment Requirements}
\lips
\subsection{Requirements for Interfacing with Adjacent Systems}
\lips
\subsection{Productization Requirements}
\lips
\subsection{Release Requirements}
\lips

\section{Maintainability and Support Requirements}
\subsection{Maintenance Requirements}
\lips
\subsection{Supportability Requirements}
\lips
\subsection{Adaptability Requirements}
\lips

\section{Security Requirements}
\subsection{Access Requirements}

\begin{myreq}
  \threeinline
    {\reqno 24}
    {\reqtype 15}
    {\reqevent 4}
  \reqdesc Only players can join a team.
  \reqrat Users not logged in as a player should not be able to join a team.
  \reqorig Nicholas Fabugais-Inaba
  \reqfit Player is able to request to join the team their captain created.
  \twoinline
    {\reqsatis 5}
    {\reqdissat 5}
  \twoinline
  {\reqdep None}
  {\reqconf None}
  \reqmater None
  \reqhist Created October 10, 2024 
\end{myreq}

\begin{myreq}
  \threeinline
    {\reqno 25}
    {\reqtype 15}
    {\reqevent 3}
  \reqdesc Only captains can create their own teams.
  \reqrat Users not logged in as a captain should not be able to create their
  own team.
  \reqorig Nicholas Fabugais-Inaba
  \reqfit Captains are able to create their own teams.
  \twoinline
    {\reqsatis 5}
    {\reqdissat 5}
  \twoinline
  {\reqdep None}
  {\reqconf None}
  \reqmater None
  \reqhist Created October 10, 2024 
\end{myreq}

\begin{myreq}
  \threeinline
    {\reqno 26}
    {\reqtype 15}
    {\reqevent 6}
  \reqdesc Only captains can reschedule games for their team.
  \reqrat Users not logged in as a captain should not be able to reschedule games.
  \reqorig Nicholas Fabugais-Inaba
  \reqfit Captains are able to request a reschedule for their upcoming game.
  \twoinline
    {\reqsatis 5}
    {\reqdissat 5}
  \twoinline
  {\reqdep None}
  {\reqconf None}
  \reqmater None
  \reqhist Created October 10, 2024 
\end{myreq}

\begin{myreq}
  \threeinline
    {\reqno 27}
    {\reqtype 15}
    {\reqevent 10}
  \reqdesc Only commissioners can send a league-wide alert.
  \reqrat Users not logged in as a commissioner should not be able to send a
  league-wide alert.
  \reqorig Nicholas Fabugais-Inaba
  \reqfit Commissioners are able to send an alert to the entire league.
  \twoinline
    {\reqsatis 5}
    {\reqdissat 5}
  \twoinline
  {\reqdep None}
  {\reqconf None}
  \reqmater None
  \reqhist Created October 10, 2024 
\end{myreq}

\begin{myreq}
  \threeinline
    {\reqno 28}
    {\reqtype 15}
    {\reqevent 12}
  \reqdesc Only commissioners can assign teams to a division.
  \reqrat Users not logged in as a commissioner should not be able to assign
  teams to a division.
  \reqorig Nicholas Fabugais-Inaba
  \reqfit Commissioners are able to assign a team to a division.
  \twoinline
    {\reqsatis 5}
    {\reqdissat 5}
  \twoinline
  {\reqdep None}
  {\reqconf None}
  \reqmater None
  \reqhist Created October 10, 2024 
\end{myreq}

\begin{myreq}
  \threeinline
    {\reqno 29}
    {\reqtype 15}
    {\reqevent 12}
  \reqdesc Only commissioners can change any team's composition.
  \reqrat Users not logged in as a commissioner should not be able to change
  a team's composition.
  \reqorig Nicholas Fabugais-Inaba
  \reqfit Commissioners are able to change any team's composition.
  \twoinline
    {\reqsatis 5}
    {\reqdissat 5}
  \twoinline
  {\reqdep None}
  {\reqconf None}
  \reqmater None
  \reqhist Created October 10, 2024 
\end{myreq}

\begin{myreq}
  \threeinline
    {\reqno 30}
    {\reqtype 15}
    {\reqevent 12}
  \reqdesc Only captains can change their own team's composition.
  \reqrat Users not logged in as a captain should not be able to change
  a team's composition.
  \reqorig Nicholas Fabugais-Inaba
  \reqfit Captains are able to change their own team's composition.
  \twoinline
    {\reqsatis 5}
    {\reqdissat 5}
  \twoinline
  {\reqdep None}
  {\reqconf None}
  \reqmater None
  \reqhist Created October 11, 2024 
\end{myreq}

\begin{myreq}
  \threeinline
    {\reqno 31}
    {\reqtype 15}
    {\reqevent 12}
  \reqdesc Only commissioners can change the season schedule.
  \reqrat Users not logged in as a commissioner should not be able to change
  the season's schedule.
  \reqorig Nicholas Fabugais-Inaba
  \reqfit Commissioners are able to change the season's schedule.
  \twoinline
    {\reqsatis 5}
    {\reqdissat 5}
  \twoinline
  {\reqdep None}
  {\reqconf None}
  \reqmater None
  \reqhist Created October 10, 2024 
\end{myreq}

\begin{myreq}
  \threeinline
    {\reqno 32}
    {\reqtype 15}
    {\reqevent 9}
  \reqdesc Only players can see the contact information of their own captain.
  \reqrat A player's email and phone number shall remain anonymous to
  individuals outside the league and everyone else but their captain.
  \reqorig Nicholas Fabugais-Inaba
  \reqfit Contact information for players can only be seen by their own captain.
  \twoinline
    {\reqsatis 5}
    {\reqdissat 5}
  \twoinline
  {\reqdep None}
  {\reqconf None}
  \reqmater None
  \reqhist Created October 10, 2024 
\end{myreq}

\begin{myreq}
  \threeinline
    {\reqno 33}
    {\reqtype 15}
    {\reqevent 9}
  \reqdesc Captains can see the contact information of players on their own team and
  other captains on other teams.
  \reqrat A captain's email and phone number shall remain anonymous to
  individuals outside the league or to players on other teams.
  \reqorig Nicholas Fabugais-Inaba
  \reqfit Contact information for captains can only be seen by
  the players on their own team or other captains from other teams.
  \twoinline
    {\reqsatis 5}
    {\reqdissat 5}
  \twoinline
  {\reqdep None}
  {\reqconf None}
  \reqmater None
  \reqhist Created October 10, 2024 
\end{myreq}

\begin{myreq}
  \threeinline
    {\reqno 34}
    {\reqtype 15}
    {\reqevent 9}
  \reqdesc Only commissioners can see the contact information of everyone in the league.
  \reqrat Player and captains shall remain anonymous to
  individuals outside the league or to players on other teams.
  \reqorig Nicholas Fabugais-Inaba
  \reqfit Commissioners are the only individuals in the league that can see everyone's
  contact information.
  \twoinline
    {\reqsatis 5}
    {\reqdissat 5}
  \twoinline
  {\reqdep None}
  {\reqconf None}
  \reqmater None
  \reqhist Created October 10, 2024 
\end{myreq}

\subsection{Integrity Requirements}

\begin{myreq}
  \threeinline
    {\reqno 35}
    {\reqtype 15}
    {\reqevent 5}
  \reqdesc Website must not create conflicts when scheduling games.
  \reqrat The system shall create a schedule for all the teams in the league,
  without scheduling conflicts. Rescheduled games must also not conflict with the
  season schedule.
  \reqorig Nicholas Fabugais-Inaba
  \reqfit There exists no conflicts in the season schedule when the season begins or
  throughout the season when games are rescheduled.
  \twoinline
    {\reqsatis 5}
    {\reqdissat 5}
  \twoinline
  {\reqdep None}
  {\reqconf None}
  \reqmater None
  \reqhist Created October 7, 2024 
\end{myreq}

\subsection{Privacy Requirements}

\begin{myreq}
  \threeinline
    {\reqno 36}
    {\reqtype 15}
    {\reqevent 1}
  \reqdesc The product shall not reveal contact information to any
  parties without the necessary access.
  \reqrat Contact information must be kept private within the league.
  \reqorig Nicholas Fabugais-Inaba
  \reqfit Contact information cannot be accessed by users that do not have
  an account registered in the league.
  \twoinline
    {\reqsatis 5}
    {\reqdissat 5}
  \twoinline
  {\reqdep None}
  {\reqconf None}
  \reqmater None
  \reqhist Created October 11, 2024 
\end{myreq}

\subsection{Audit Requirements}
Not applicable.
\subsection{Immunity Requirements}
Not applicable.

\section{Cultural Requirements}
\subsection{Cultural Requirements}

\begin{myreq}
  \threeinline
    {\reqno 37}
    {\reqtype 16}
    {\reqevent 1}
  \reqdesc The product shall provide an option to not specify gender.
  \reqrat Users may not associate themselves as male or female.
  \reqorig Nicholas Fabugais-Inaba
  \reqfit A user must be able to choose an option to not specify their
  gender.
  \twoinline
    {\reqsatis 5}
    {\reqdissat 5}
  \twoinline
  {\reqdep None}
  {\reqconf None}
  \reqmater None
  \reqhist Created October 11, 2024 
\end{myreq}

\section{Compliance Requirements}
\subsection{Legal Requirements}
Not applicable.
\subsection{Standards Compliance Requirements}
Not applicable.

\section{Open Issues}

\begin{itemize}

  \item There are no open issues

\end{itemize}

\section{Off-the-Shelf Solutions}
\subsection{Ready-Made Products}
Not applicable.
\subsection{Reusable Components}
Not applicable.
\subsection{Products That Can Be Copied}
Not applicable.

\section{New Problems}
\subsection{Effects on the Current Environment}
Not applicable.
\subsection{Effects on the Installed Systems}
Not applicable.
\subsection{Potential User Problems}

Existing users of the current platform may suffer from the migration
to the new platform as they may not be familiar with how to navigate
the new system and utilize certain functions/features. This could be
concerning considering the age group and lack of technical literacy
for some users. One precaution that can be taken is to create a user
guide to aid users in interacting with the new system and the various
features it has to offer.

\subsection{Limitations in the Anticipated Implementation Environment That May
Inhibit the New Product}
\lips
\subsection{Follow-Up Problems}
\lips

\section{Tasks}
\subsection{Project Planning}
\lips
\subsection{Planning of the Development Phases}
\lips

\section{Migration to the New Product}
\subsection{Requirements for Migration to the New Product}
\lips

\subsection{Data That Has to be Modified or Translated for the New System}
\lips

\section{Costs}
\lips
\section{User Documentation and Training}
\subsection{User Documentation Requirements}

\begin{myreq}
  \threeinline
    {\reqno xx}
    {\reqtype 0}
    {\reqevent xx}
  \reqdesc There must exist a user guide that informs users of the previous
  system and new users how to use the features of the new system.
  \reqrat There will be many new users of the system who might not be familiar
  with how to use the features of our solution, and will require guidance.
  \reqorig Alex Verity
  \reqfit There shall exist a user guide that explains all major features of
  the system.
  \twoinline
    {\reqsatis 3}
    {\reqdissat 1}
  \twoinline
  {\reqdep None}
  {\reqconf None}
  \reqmater
  \reqhist
\end{myreq}

\subsection{Training Requirements}
\lips

\section{Waiting Room}
\lips

\section{Ideas for Solution}
\lips

\newpage{}
\section*{Appendix --- Reflection}

The information in this section will be used to evaluate the team members on the
graduate attribute of Lifelong Learning.  Please answer the following questions:

\begin{enumerate}
  \item What knowledge and skills will the team collectively need to acquire to
  successfully complete this capstone project?  Examples of possible knowledge
  to acquire include domain specific knowledge from the domain of your
  application, or software engineering knowledge, mechatronics knowledge or
  computer science knowledge.  Skills may be related to technology, or writing,
  or presentation, or team management, etc.  You should look to identify at
  least one item for each team member.
  \item For each of the knowledge areas and skills identified in the previous
  question, what are at least two approaches to acquiring the knowledge or
  mastering the skill?  Of the identified approaches, which will each team
  member pursue, and why did they make this choice?
\end{enumerate}

\subsection*{Alex Verity -- Reflection}

\begin{enumerate}
  \item What went well while writing this deliverable?\\\\
  The parts that went well for me were the non-functional requirements, I felt
  the look and feel section gave me a much better idea of what the solution is
  going to look like. The TA meeting was also extremely helpful and gave me a
  much better perspective on the project and confidence in our solution.
  \item What pain points did you experience during this deliverable, and how
  did you resolve them?\\\\
  Some pain points experienced were trying to get full coverage when coming up
  with use cases and functional requirements, we have many more questions for
  the supervisor now and will almost certainly need to come back to this
  document and modify it with new information we recieve. I still don't know
  if the use cases provided are well put together or not and will be looking
  heavily at feedback in that area.
  \item How many of your requirements were inspired by speaking to your
  client(s) or their proxies (e.g. your peers, stakeholders, potential users)?
  \\\\
  Almost all requirements are inspired by speaking to our supervisor and
  primary stakeholder, as he defined the entire project during meetings with
  him. We were also heavily inspired by the TA meeting, where many of the
  look and feel requirements and user guide requirements came from, as well
  as advice on what to make into functional requirements.
  \item Which of the courses you have taken, or are currently taking, will help
  your team to be successful with your capstone project.\\\\
  The class 4HC3, whose subject is human computer interfaces and talks mainly
  about designing user interface. One of the main goals of the project is
  designing the UI to be more readable and better designed than the current
  solution, so any learning in that regard will be vital.

\end{enumerate}

\subsection*{Jung Woo Lee -- Reflection}

\begin{enumerate}
  \item What went well while writing this deliverable?\\\\
  What pain points did you experience during this deliverable, and how did you resolve them? 
  Going through and working on this document, made me realize how much more we as a team probably need to know about user needs. Sometimes we were left with questions on if something was a requirement or not. All of this indicated to the fact that we need to have more time to speak to our supervisor.
  
  \item What pain points did you experience during this deliverable, and how
  did you resolve them?\\\\
  Almost all of the requirements came from the supervisor, who is both a potential user and stakeholder. With the initial information he provided us on the problem and desired product, we could develop most of these requirements. A few of the requirements were not explicitly told to us, but could be inferred and others have been created by us to conform with ethical practices for example.
  \item How many of your requirements were inspired by speaking to your
  client(s) or their proxies (e.g. your peers, stakeholders, potential users)?
  \\\\
  Almost all requirements are inspired by speaking to our supervisor and
  primary stakeholder, as he defined the entire project during meetings with
  him. We were also heavily inspired by the TA meeting, where many of the
  look and feel requirements and user guide requirements came from, as well
  as advice on what to make into functional requirements.
  \item Which of the courses you have taken, or are currently taking, will help
  your team to be successful with your capstone project.\\\\
  All project based courses are definitely useful to understand project dynamics. Dealing with a team, and performing to certain standards has been taught to us in all of these various courses. Providing deliverables and project scheduling has also been taught to us, such that administrative duties are easier for us to tackle. The databases course will likely help us as our project will most likely be using this in our solution. The user-interface course will help us in creating a human-centred design that is intuitive and clean. This is one of the pitfalls of the current solution, and thus one of the important non-functional requirements. The requirements course will also help us interpret requirements going forward and make use of our work here. And late it will undoubtedly help us revise this document.

\end{enumerate}

\end{document}