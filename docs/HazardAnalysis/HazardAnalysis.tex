\documentclass{article}

\usepackage{booktabs}
\usepackage{tabularx}
\usepackage{hyperref}
\usepackage{multirow}
\usepackage{pdflscape}
\usepackage{geometry}
\usepackage{multicol}
\usepackage[dvipsnames]{xcolor}
\usepackage{enumitem}
\usepackage{tcolorbox}
\usepackage{array}
\usepackage{afterpage}
\usepackage{longtable}

\hypersetup{
    colorlinks=true,       % false: boxed links; true: colored links
    linkcolor=red,          % color of internal links (change box color with linkbordercolor)
    citecolor=green,        % color of links to bibliography
    filecolor=magenta,      % color of file links
    urlcolor=cyan           % color of external links
}

\newenvironment{myreq}[1]{%
\setlist[description]{font=\normalfont\color{darkgray}}%
\begin{tcolorbox}[colframe=black,colback=white, sharp corners, boxrule=1pt]%
\bfseries\color{blue}%
\begin{description}#1}%
{\end{description}\end{tcolorbox}}

\newcommand{\twoinline}[2]{\begin{multicols}{2}#1 #2\end{multicols}}

\newcommand{\reqno}{\item[Requirement \#:]}
\newcommand{\reqdesc}{\item[Description:]}
\newcommand{\reqrat}{\item[Rationale:]}
\newcommand{\reqorig}{\item[Originator:]}
\newcommand{\reqfit}{\item[Fit Criterion:]}
\newcommand{\reqsatis}{\item[Customer Satisfaction:]}
\newcommand{\reqdissat}{\item[Customer Dissatisfaction:]}
\newcommand{\reqdep}{\item[Dependencies:]}
\newcommand{\reqconf}{\item[Conflicts:]}
\newcommand{\reqmater}{\item[Materials:]}
\newcommand{\reqhist}{\item[History:]}

\title{Hazard Analysis\\\progname}

\author{\authname}

\date{}

\input{../Comments}
\input{../Common}

\begin{document}

\maketitle
\thispagestyle{empty}

~\newpage

\pagenumbering{roman}

\begin{table}[hp]
\caption{Revision History} \label{TblRevisionHistory}
\begin{tabularx}{\textwidth}{llX}
\toprule
\textbf{Date} & \textbf{Developer(s)} & \textbf{Change}\\
\midrule
Oct. 21, 2024 & NFI, JL, AV, CG & TA Feedback\\
Oct. 23, 2024 & NFI, JL, AV, CG & Rev0\\
Jan. 7, 2025 & JL, NFI & Made changes to reflect changes in SRS; added AS-6 to table.
Based on TA feedback: Adjustments made to weak assumptions.\\
Jan. 8, 2025 & NFI, JL & Based on TA feedback: Fixed spelling/grammatical errors.
Changed visible and readable to specifically highlight font size and color ratio. Added reference columns for hazards in FMEA table.\\
Apr. 4, 2025 & JL & Based on TA feedback: Added table number and caption; referenced in 
document body. Added security requirement, added security and data integrity parts to FMEA table. Also fixed page numbering.\\
\bottomrule
\end{tabularx}
\end{table}

~\newpage

\tableofcontents

~\newpage

\pagenumbering{arabic}

\section{Introduction}

Sandlot is intended to be the successor to the current McMaster GSA
softball league scheduling and management platform. This project
will implement the current functionality from the existing system including
additional features such as a new login system, commissioner admin
privileges, and an improvement to the robustness, ease of use, and
maintainability to the platform.

Hazards are conditions of the product system that could lead to undesirable outcomes. 
In Sandlot, security hazards, malfunctions, and usability will be analyzed.

\section{Scope and Purpose of Hazard Analysis}

Hazard analysis aims to identify potential hazards that may exist with the product, 
understand their causes and effects, and provide proactive strategies to mitigate
them.
The purpose of this document is to perform this analysis for Sandlot, investigate potential
hazards and give means of mitigating the issues identified. 
Hazards to Sandlot could take the form of user interface misinterpretations, misinputted data,
and database errors, leading to possible malfunctions with the product and/or user 
dissatisfaction.
This document will first identify the system components and boundaries, state the critical
assumptions, use a Failure Mode and Effects Analysis (FMEA) table (\hyperref[tab:FMEA]{Table 1}) to identify the hazards
and their causes, effects, and mitigation strategies, and lastly the requirements that
are discovered from the analysis.

\section{System Boundaries and Components}

\subsection{System Components}
The system can be divided into the following components:
\begin{itemize}
    \item Authentication
    \subitem This component authenticates user-inputted login information with the database
    and determines whether a login is successful.
    \item Scheduling
    \subitem This component generates a schedule based on user-inputted availability data. 
    Additionally, it allows for rescheduling events in these schedules after they 
    have been generated.
    \item Team Structure
    \subitem This component is a data structure of the system that allows accounts to be grouped 
    such that a group of accounts interact with the system, its data, and its components in the same way as 
    each other. Additionally, accounts can be added/removed from teams, users can request to join a 
    certain team, and teams can be created/deleted.
    \item Accounts
    \subitem This component allows users to create, delete, or edit accounts. Additionally, account types 
    determine how a user can interact with the system, its data, and its components.
    \item Scoring/Standings
    \subitem This component allows users to input score data associated with specific events and will 
    generate cumulative standings based on score data.
    \item Alerts
    \subitem This component alerts users of important information related to the system.
    \item Database
    \subitem This component stores the system's data.
\end{itemize}

\subsection{System Boundaries}
The system limits itself to the components mentioned before; the frontend and backend of the website.
External APIs connecting to our system are considered outside the system boundaries.
This would include the web server, and weather tracker, among others. 

\section{Critical Assumptions}

\begin{itemize}
  \item It is assumed that users will act in good faith and will not engage in activities such as
  exploiting vulnerabilities, unauthorized access, or other malicious actions against the system.
  
  \item The system assumes that the data provided by users is accurate and properly formatted, and
  that users will comply with any input validation guidelines. Any discrepancies or errors in data
  may affect the system's ability to function correctly.

  \item The system assumes that the underlying infrastructure (servers, network, etc.) will function
  reliably, with sufficient resources to support the expected load and usage patterns.
  
  \item It is assumed that the system will be used within the intended geographical and legal
  contexts, and that any legal or regulatory considerations regarding data privacy and security
  will be adhered to.
  
  \item It is assumed that any third-party services or APIs integrated into the system are secure
  and reliable, with proper data safety protocols in place.
  
  \item The system assumes that users will have a stable internet connection, as connectivity is
  required for full functionality.  
\end{itemize}

\section{Failure Mode and Effect Analysis}

\newgeometry{left=0.65in,right=0.25in,top=0.25in,bottom=0.20in}
\pagenumbering{gobble}
\begin{landscape}

% Longtable starts here
\def\arraystretch{1.2}
\begin{longtable}{|p{3cm}||p{4cm}|p{4cm}|p{4cm}|p{1cm}|p{5.75cm}|p{1cm}|}
\caption{Failure Mode and Effect Analysis Table} \label{tab:FMEA} \\
\hline
\textbf{Component} & \textbf{Failure Mode} & \textbf{Failure Effect} & \textbf{Failure Cause} & \textbf{SR} & \textbf{Recommended Actions} & \textbf{Ref} \\
\hline
\endfirsthead

\hline
\textbf{Component} & \textbf{Failure Mode} & \textbf{Failure Effect} & \textbf{Failure Cause} & \textbf{SR} & \textbf{Recommended Actions} & \textbf{Ref} \\
\hline
\endhead

\hline
\multicolumn{7}{r}{\textit{(Continued on next page)}} \\
\endfoot

\hline
\caption{Failure Mode and Effect Analysis Table} \label{tab:FMEA} \\
\endlastfoot

\multirow{2}{2cm}{Authentication} & User login information does not match database stored
login information. & User cannot access their account. & Misinputted user login information.
& EU-2 AS-6 & Misinputted user login information should give a warning to the user when
login information does not exist or is incorrect. & HA1\\
\hline

\multirow{3}{2cm}{Scheduling} & Captain inputs incorrect availability dates at
start of season & Schedule is generated incorrectly. & Captain's user error. &
FR-17 & Allow the captain to resubmit availability that overwrites previously
submitted availability if submitted before the due date. & HA2.1\\
\cline{2-7}
& User reads schedule incorrectly. & User may not attend a game by accident or
travel to a game at the wrong time. & Schedule is unclear, or unreadable. & AC-1 AC-2 & Ensure schedule data and structure are a readable
font size of at least 16 pixels and the colours have a contrast ratio of at least
4.5:1. & HA2.2\\
\cline{2-7}
& Team's availability data have scheduling conflicts. & System cannot make a
valid schedule. & Availability data has scheduling conflicts. & EU-3 & Teams
will be given a warning if their availability data has scheduling conflicts. & HA2.3\\
\cline{2-7}
& Captain doesn't respond to another captain's reschedule request. & Captain
who sent the reschedule request will not know if the opposing captain wants to
reschedule or not. & System doesn't adequately notify the opposing captain
about the reschedule request. & IG-4 & Make sure the captain who needs to
accept the request cannot avoid seeing the request notification. & HA2.4\\
\hline

\multirow{2}{2cm}{Team Structure} & User is assigned to or joins the wrong team. & User
will be shown the wrong schedule and team information. & Interface inputs are
unclear. & AP-1 AP-2 & Ensure that when assigning a team for a
user, the team options to select from are clear and the input section is
readable. & HA3\\
\hline

\multirow{3}{2cm}{Accounts} & Account is given the wrong permissions. & User 
has access to actions they shouldn't use. & Interface is unclear to user. & AS-5 & Giving an account additional permissions should give
a warning to the user when configuring accounts. & HA4.1\\
\cline{2-7}
& All commissioner level accounts are deleted. & There are no accounts left
that can give commissioner level permissions. & System allows for all
commissioner accounts to be deleted. & IG-2 & Do not let users delete a
commissioner account if there is only one left. & HA4.2\\
\cline{2-7}
& User logs in on an account that isn't theirs. & User may make actions that
are not desired by the account's owner. & Account owners are not careful with
their login data. & PV-2 & Remind users to keep their passwords secret and secure.
& HA4.3\\
\hline

\multirow{2}{2cm}{Scoring/ Standings} & Captain inputs the wrong score. &
Standings and the database have incorrect scores. & User inputs the wrong score. &
IG-3 & Allow the opposing team's captain to verify the score and contest it if
the score is wrong. & HA5\\
\hline

\multirow{2}{2cm}{Alerts} & User reads alert incorrectly. &
User may travel to a game that was postponed or cancelled, or miss out on
critical information. & Alert is unclear, unreadable, or the user makes an error. &
AC-3 & Ensure the alert message has a readable font size of at least 16 pixels
and colours with a contrast ratio of at least 4.5:1. & HA6\\
\hline

\multirow{3}{2cm}{Database} & Database fails. &
Data may be lost and unrecoverable. & Database fails. &
IG-5, IG-6 & Enforce database backups and implement audit logging of some form to diagnose issues. & HA7\\
\cline{2-7}
& User's data is stolen. & User's private information is compromised to a malicious party. & User's data is not protected. & PV-3 & Enforce data protection of some sort, such as hashing, for sensitive personal information such as emails, passwords, and contact info. & HA8\\
\hline

\end{longtable}

\end{landscape}
\restoregeometry{}


\pagenumbering{arabic}
\setcounter{page}{4}

\section{Safety and Security Requirements}

\begin{myreq}
    \reqno FR-17
    \reqdesc If the season start availability due date hasn't been reached, a
    captain should be able to resubmit availability data which overwrites
    previous availability data.
    \reqrat If the captain makes an error when submitting availability data
    they should be able to fix their error.
    \reqorig Alex Verity
    \reqfit If the captain has submitted data, they shall be able to overwrite
    it with new data.
    \twoinline
      {\reqsatis 4}
      {\reqdissat 3}
    \reqhist Created 2024-10-15
\end{myreq}

\begin{myreq}
    \reqno AS-5
    \reqdesc Giving permissions to users must be accompanied by an error that
    warns the user of the severity of the action.
    \reqrat Accidentally giving permissions to users who shouldn't have them
    could result in unexpected errors.
    \reqorig Alex Verity
    \reqfit If permissions are being changed, a warning shall be displayed to
    the user before updating the permissions.
    \twoinline
      {\reqsatis 5}
      {\reqdissat 5}
    \reqhist Created 2024-10-15
\end{myreq}

\begin{myreq}
    \reqno IG-2
    \reqdesc If there is only one commissioner level account, that account
    cannot be deleted.
    \reqrat Deleting this account would stop any more commissioner level
    accounts from being created, soft-locking the system.
    \reqorig Alex Verity
    \reqfit If there is only one commissioner level account it shall not be
    deleted.
    \twoinline
      {\reqsatis 3}
      {\reqdissat 3}
    \reqhist Created 2024-10-15
\end{myreq}

\begin{myreq}
    \reqno IG-3
    \reqdesc All match scores must be visible and contestable by other
    captains once recorded.
    \reqrat Match score correctness is extremely important for a competitive
    league, if a score is wrong, captains should be able to request that it be
    fixed.
    \reqorig Alex Verity
    \reqfit Any match scores shall have the option to be viewed and contested
    by captains.
    \twoinline
      {\reqsatis 3}
      {\reqdissat 3}
    \reqhist Created 2024-10-15
\end{myreq}

\begin{myreq}
  \reqno IG-4
  \reqdesc Captains are adequately notified when they receive an alert or a
  reschedule request.
  \reqrat Important information may be shared in alerts, and reschedule
  requests that go unanswered may be frustrating for captains. Notifications
  should reach their intended targets if possible.
  \reqorig Alex Verity
  \reqfit Assuming the user hasn't interfered, alerts and reschedule requests
  shall always be sent to a place where the user receives notifications.
  \twoinline
    {\reqsatis 3}
    {\reqdissat 3}
  \reqhist Created 2024-10-15
\end{myreq}

\begin{myreq}
  \reqno AS-6
  \reqdesc Users can only access an account with correctly inputted login
  information that matches the database stored login information.
  \reqrat Login information stored in the database should correspond to a
  specific account. The account should only be accessed by the correctly
  inputted login information details.
  \reqorig Nicholas Fabugais-Inaba
  \reqfit Assuming the user has correctly inputted the login details for
  an account stored in the database, they should be granted access into
  the corresponding account.
  \twoinline
    {\reqsatis 3}
    {\reqdissat 3}
  \reqhist Created 2024-10-25
\end{myreq}

\begin{myreq}
  \reqno EU-2
  \reqdesc Misinputted user login information shall provide a warning to the
  user, if login information does not exist or does not match the database
  stored login information.
  \reqrat Login information stored in the database should match the user
  inputted login information. Feedback should be provided for the user to
  understand an error has occurred when accessing an account with incorrect
  login details.
  \reqorig Nicholas Fabugais-Inaba
  \reqfit Assuming the user has misinputted the login details for
  an account stored in the database, they should be given a warning that
  notifies them about the login information being incorrect or not existing
  in the database.
  \twoinline
    {\reqsatis 3}
    {\reqdissat 3}
  \reqhist Created 2024-10-25
\end{myreq}

\begin{myreq}
  \reqno EU-3
  \reqdesc Teams will be given a warning if their availability data has
  scheduling conflicts.
  \reqrat The system should be able to create a valid schedule, in which
  teams do not have conflicting availability data that schedules games for
  the same dates and times as other teams.
  \reqorig Nicholas Fabugais-Inaba
  \reqfit Teams should receive a warning about their availability data
  conflicting on the schedule.
  \twoinline
    {\reqsatis 3}
    {\reqdissat 3}
  \reqhist Created 2024-10-25
\end{myreq}

\begin{myreq}
  \reqno AC-3
  \reqdesc Alerts sent to users must be a readable font size of at least 16 pixels
  and the colours must have a contrast ratio of at least 4.5:1.
  \reqrat User may travel to a game that was postponed or cancelled, or miss
  out on critical information.
  \reqorig Nicholas Fabugais-Inaba
  \reqfit User receives an alert that is readable and clear enough for them
  to understand.
  \twoinline
    {\reqsatis 3}
    {\reqdissat 3}
  \reqhist Created 2024-10-25
\end{myreq}

\begin{myreq}
  \reqno PV-2
  \reqdesc Users are reminded to keep their passwords secure.
  \reqrat If a user accesses another user's account they may do actions
  without the account owner's permission or that the account owner is unaware
  of.
  \reqorig Alex Verity
  \reqfit A reminder telling users to keep their passwords secret and secure
  is displayed to the user.
  \twoinline
    {\reqsatis 1}
    {\reqdissat 1}
  \reqhist Created 2024-10-25
\end{myreq}

\begin{myreq}
  \reqno IG-5
  \reqdesc The system should implement audit logging for system actions.
  \reqrat The system should log system activities to help troubleshoot issues
  within the system and make sure the system runs as intended.
  \reqorig Nicholas Fabugais-Inaba
  \reqfit Logs are generated after each system action.
  \twoinline
    {\reqsatis 3}
    {\reqdissat 3}
  \reqhist Created 2025-01-08
\end{myreq}

\begin{myreq}
  \reqno IG-6
  \reqdesc The system should have data backups at least every BACKUP\_WEEKS weeks.
  \reqrat The system should complete regularly scheduled data backups in the case
  where the platform shuts down, making sure data is not lost.
  \reqorig Nicholas Fabugais-Inaba
  \reqfit Data backups are completed at least every
  \\ BACKUP\_WEEKS weeks.
  \twoinline
    {\reqsatis 3}
    {\reqdissat 3}
  \reqhist Created 2025-01-08
\end{myreq}

\begin{myreq}
  \reqno PV-3
  \reqdesc The system should enforce data protection for sensitive user data.
  \reqrat The system should protect sensitive user data from being stolen or otherwise accessed by a possibly malicious party.
  \reqorig Jung Woo Lee
  \reqfit Sensitive user data should be protected by hashing or other means of data protection.
  \twoinline
    {\reqsatis 3}
    {\reqdissat 5}
  \reqhist Created 2025-04-04
\end{myreq}

\section{Roadmap}

The hazard analysis resulted in new safety and security requirements being
introduced to the project's already existing requirements located in the
SRS. Provided the time constraints and current project deadlines, high-priority
requirements will be initially addressed with lower-priority requirements
being investigated in the future. The explanation of these priorities can be viewed
in the SRS section 21.1. The hazard analysis will be used as a guideline
to ensure the safety and security of the Sandlot platform's users.\\\\

\textbf{Proof of Concept Demonstration (November 11-22):} \\\\
This phase focuses on demonstrating the system's core functionalities to ensure
feasibility and alignment with initial requirements. 
\begin{itemize}
  \item Authentication (AS-6 and EU-2): Validating user login information and ensuring
  that users can access their respective accounts.
  \item Scheduling (FR-17, AC-1, AC-2, and EU-3):  Verifying schedule data is correct
  and adheres to the readable requirements listed in the recommended actions.\\
\end{itemize}

\textbf{Final Demonstration (Revision 1) (March 24-30):} \\\\
The final demonstration phase evaluates the full implementation of the system,
ensuring it meets all specified functional and non-functional requirements.
\begin{itemize}
  \item Scheduling (IG-4): System adequately notifies the user of a rescheduling request.
  \item Teams (AP-1 and AP-2): User is appropriately assigned to/correctly joins their
    team.
  \item Accounts (AS-5, IG-2, and PV-2): Accounts are correctly given their respective 
  permissions.
  \item Scoring/Standings (IG-3): Displaying of scores are verified to be accurate.
  \item Alerts (AC-3): Displaying of alerts adhere to the readable requirements listed
    in the recommended actions.
\end{itemize}

\newpage{}

\section*{Appendix --- Reflection}

\input{../Reflection.tex}

\begin{enumerate}
    \item What went well while writing this deliverable? 
    \item What pain points did you experience during this deliverable, and how
    did you resolve them?
    \item Which of your listed risks had your team thought of before this
    deliverable, and which did you think of while doing this deliverable? For
    the latter ones (ones you thought of while doing the Hazard Analysis), how
    did they come about?
    \item Other than the risk of physical harm (some projects may not have any
    appreciable risks of this form), list at least 2 other types of risk in
    software products. Why are they important to consider?
\end{enumerate}

\subsection*{Nicholas Fabugais-Inaba -- Reflection}

\begin{enumerate}
  \item What went well while writing this deliverable?\\\\
  When writing this deliverable, outlining the different failure modes and effects 
  helped in understanding what hazards could impact our project from more than just a
  surface level view. More specifically, content in components such as scheduling and
  alerts, aided in my understanding of how each user may be impacted by a type of failure
  that could occur within the system.
  \item What pain points did you experience during this deliverable, and how
  did you resolve them?\\\\
  Some of the pain points from this deliverable had to deal with brainstorming
  appropriate hazards that may impact a user in the system. With many working components,
  there are a lot of hazards that could occur or that could be easily handled. 
  Understanding which hazards could impact the users the most, which the team would then
  focus on to address primarily before other secondary hazards, was really crucial
  in the completion of this deliverable. These pain points were resolved
  by brainstorming together as a team our own failure modes that we would then review
  amongst each other, if a member needed a second opinion on the component they were
  working on. This would help all of us to understand if the hazard was crucial to
  address for the sake of the system users or if it could be dealt with in an alternative
  way, and therefore not need to be listed.
  \item Which of your listed risks had your team thought of before this
  deliverable, and which did you think of while doing this deliverable? For
  the latter ones (ones you thought of while doing the Hazard Analysis), how
  did they come about?
  \\\\
  Before this deliverable, one of the bigger risks our team had thought of before
  this deliverable was what we would need to do to address scheduling conflicts.
  We knew the scheduling portion of the system was an important aspect of this project
  and would need to be dealt with appropriately to have a robust system rather than the
  current platform that is prone to many errors. Some of the latter risks such as the
  alerts or the accounts risks came about as the team understood alerts could be misread,
  so to make it as easy as possible for users, the platform would need to make sure any
  announcements were to be clear and readable. Additionally, for account risks,
  on a rare occasion where all commissioner level accounts are deleted, we thought of
  making sure to put a safeguard so at the very least there could be one account, since
  users could make the mistake of deleting accounts by accident.
  \item Other than the risk of physical harm (some projects may not have any
  appreciable risks of this form), list at least 2 other types of risk in
  software products. Why are they important to consider?\\\\
  In software products, 2 other risks could be associated with a system shutdown from
  loss of battery life or some type of malware that could impact the software in
  a negative way. These would be important to consider as in the case for a failure
  from the system, there should be a failsafe to deal with such a mode as system integrity
  could be impacted as well as the user's device itself. In terms of malware, it could
  also impact the system's integrity leaving it vulnerable to impact users with faulty
  links that could damage a user's device. Malware could also crash the software system
  itself, which would be detrimental.

\end{enumerate}

\subsection*{Casra Ghazanfari -- Reflection}

\begin{enumerate}
  \item What went well while writing this deliverable?\\\\
  Writing the “System Components” section of this deliverable went well for our 
  team. This is because going into this deliverable we already had a concrete 
  idea of the structure of our overall system. This allowed us to quickly 
  partition the system into components based on our ideas for the overall 
  structure. Moreover, due to our existing understanding of the overall system 
  we were able to easily envision and determine how the derived components of 
  the system would interact with each other and the system boundaries.
  \item What pain points did you experience during this deliverable, and how
  did you resolve them?\\\\
  The biggest pain point during this deliverable was connecting hazards in our 
  FMEA table to security requirements (SRs). This is because our SRS deliverable 
  was incomplete at the time of writing this deliverable and did not have a 
  proper set of security requirements to connect to any of the hazards 
  identified in the FMEA. This significantly increased the amount of work 
  needed for us to properly complete this deliverable and generally slowed down 
  the process of creating the FMEA table as well.
  \item Which of your listed risks had your team thought of before this
  deliverable, and which did you think of while doing this deliverable? For
  the latter ones (ones you thought of while doing the Hazard Analysis), how
  did they come about?\\\\
  The team didn’t concretely identify any risks prior to working on this 
  deliverable, but had loose ideas of more obvious potential risks to include 
  in the deliverable. “The user incorrectly inputting their login information” 
  or “the user joining the wrong team” are examples of risks we had thought of 
  before working on this deliverable. Risks like “all commissioner accounts are 
  deleted” and “account is given wrong permissions” are examples of risks we 
  thought of while working through this deliverable. Generally, the idea to 
  include these risks came from carefully evaluating each individual component 
  defined previously in the document and asking ourselves the question: 
  “if anything could go wrong, what would go wrong?”. From this line of thinking 
  we were able to envision and identify uncommon risk scenarios for the system 
  that were tied to the system’s more complex components.
  \item Other than the risk of physical harm (some projects may not have any
  appreciable risks of this form), list at least 2 other types of risk in
  software products. Why are they important to consider?\\\\
  \begin{enumerate}
    \item Security risks\\
    Software products that handle sensitive data (i.e. personal / financial / 
    health information) must consider the risks of handling sensitive information. 
    Data can be stolen through unauthorized access and data breaches and can cause 
    financial and reputational harm to both the users and owners of the product. 
    Therefore, security risks such as data breaches are important to consider when 
    handling sensitive data, especially when trying to comply with data protection 
    regulations.
    \item Operational risks\\
    Software products that require high availability must consider the risks of 
    relying on services outside the product’s system boundary. For example, when 
    using a hosting service provided by another organization to host the product’s 
    website or database, There’s the inherent risk that if the organization hosting 
    the service faces issues, so will your product. Issues like these can cause 
    delays or system downtime for your product which is especially harmful if the 
    product requires high availability as downtime negatively impacts the user 
    satisfaction and experience of users of the product.
  \end{enumerate}

\end{enumerate}

\subsection*{Alexander Verity -- Reflection}

\begin{enumerate}
  \item What went well while writing this deliverable?\\\\
  The new requirements added needed coverage to the SRS document. Thinking
  about hazards helped a lot in adding requirements, and I am much more
  confident about the requirements document after the hazard analysis. Once
  again the TA meeting was extremely helpful, many good questions answered
  insightfully. The TA meetings help team confidence above all else. The
  supervisor meeting also was very helpful and will lead to many other changes
  to the SRS.

  \item What pain points did you experience during this deliverable, and how
  did you resolve them?\\\\
  What to add as a critical assumption was something the team debated over a
  long time. I feel like we ended up with a good critical assumption section
  but it took a lot of thinking and it is probably the part of the document
  we are least confident. The system boundary section was also difficult and
  required a lot of thinking.

  \item Which of your listed risks had your team thought of before this
  deliverable, and which did you think of while doing this deliverable? For
  the latter ones (ones you thought of while doing the Hazard Analysis), how
  did they come about?\\\\
  We had thought of the coaches inputting incorrect scoring, as that is the
  part of the scoring process that is completely reliant on users to work
  correctly (however our supervisor has told us that the current system as
  never had an issue with scoring). Most of the risks we thought of while
  doing the deliverable. They mostly came about by analyzing the different
  parts of our system in section 3.

  \item Other than the risk of physical harm (some projects may not have any
  appreciable risks of this form), list at least 2 other types of risk in
  software products. Why are they important to consider?\\\\
  In our solution specifically, lack of fairness in a competitive competition.
  The league, although not every team needs to be of a high skill level,
  relies on teams taking it somewhat seriously for the most fun when playing
  games. Fairness is an important element of this. Without fairness players
  would be frustrated and likely not participate in the league. Another type
  of risk is privacy, with users personal information being leaked and used in
  ways the user didn't intend. This is important as privacy is a growing
  concern and something people value. It can also be financially dangerous or
  even physically dangerous, although that specifically is more rare.

\end{enumerate}

\subsection*{Jung Woo Lee -- Reflection}

\begin{enumerate}
  \item What went well while writing this deliverable?\\\\
  I think coming up with components went very smoothly. There was not much difficulty in figuring out the different sections of our system. As an added bonus, I found it very helpful to map out the future design of our system, and its major parts. This lets us begin to think about how these parts interact with the users and each other.
  \item What pain points did you experience during this deliverable, and how
  did you resolve them?\\\\
  There were significant issues with figuring out if certain issues counted as hazards. One example was about scheduling conflicts, and initially we had thought that two captains entering conflicting schedules at the same time should have been the hazard, however, further discussion made us realize this was becoming too implementation specific. After a long discussion, and even debating whether it was a hazard at all, we had come to an agreement on the current hazard in the FMEA table. 
  \item Which of your listed risks had your team thought of before this
  deliverable, and which did you think of while doing this deliverable? For
  the latter ones (ones you thought of while doing the Hazard Analysis), how
  did they come about?
  \\\\
  We had thought about the ‘captains inputting wrong scores’ and ‘users joining the wrong team’. Most of the others we have come up with during the hazard analysis process. How these came about was as the team drew up the system components, we were more clearly able to see how users would interact with each component, or how components would function independently from other components. This allowed us to more readily think of potential problems to the components and make our FMEA table. For example, ‘all commissioner level accounts are deleted’ was able to be thought of as we had thought to make ‘Accounts’ a component of our system. As we thought about how accounts could mess up somehow, it led us to think about the account hierarchy in our system, then to the hazard.
  \item Other than the risk of physical harm (some projects may not have any
  appreciable risks of this form), list at least 2 other types of risk in
  software products. Why are they important to consider?\\\\
  One kind of risk would be the risks associated with the project. I mean this as an umbrella that includes design, time, compliance, among others that occur during the creation of the software product. These are crucial to consider, as this can often make or break a product just as much as a bad product going to deployment. For example, a product that was made and misses the standards laid out for it would result in a large headache to go back and change, or a subpar system.

Another type would be risks associated with the post-deployment era of a product. More so in terms of maintainability, adaptability, and portability. These can be important to consider if a product is slated to function for a long period of time, and receive adequate support for a long time period, it should be made as risk free as possible in these aspects.


\end{enumerate}

\end{document}