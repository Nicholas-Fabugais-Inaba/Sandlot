\documentclass{article}

\usepackage{booktabs}
\usepackage{tabularx}
\usepackage{hyperref}
\usepackage{multirow}
\usepackage{pdflscape}
\usepackage{multicol}
\usepackage[dvipsnames]{xcolor}
\usepackage{enumitem}
\usepackage{tcolorbox}
\usepackage{array}
\usepackage{afterpage}

\hypersetup{
    colorlinks=true,       % false: boxed links; true: colored links
    linkcolor=red,          % color of internal links (change box color with linkbordercolor)
    citecolor=green,        % color of links to bibliography
    filecolor=magenta,      % color of file links
    urlcolor=cyan           % color of external links
}

\newenvironment{myreq}[1]{%
\setlist[description]{font=\normalfont\color{darkgray}}%
\begin{tcolorbox}[colframe=black,colback=white, sharp corners, boxrule=1pt]%
\bfseries\color{blue}%
\begin{description}#1}%
{\end{description}\end{tcolorbox}}

\newcommand{\twoinline}[2]{\begin{multicols}{2}#1 #2\end{multicols}}

\newcommand{\reqno}{\item[Requirement \#:]}
\newcommand{\reqdesc}{\item[Description:]}
\newcommand{\reqrat}{\item[Rationale:]}
\newcommand{\reqorig}{\item[Originator:]}
\newcommand{\reqfit}{\item[Fit Criterion:]}
\newcommand{\reqsatis}{\item[Customer Satisfaction:]}
\newcommand{\reqdissat}{\item[Customer Dissatisfaction:]}
\newcommand{\reqdep}{\item[Dependencies:]}
\newcommand{\reqconf}{\item[Conflicts:]}
\newcommand{\reqmater}{\item[Materials:]}
\newcommand{\reqhist}{\item[History:]}

\title{Hazard Analysis\\\progname}

\author{\authname}

\date{}

\input{../Comments}
\input{../Common}

\begin{document}

\maketitle
\thispagestyle{empty}

~\newpage

\pagenumbering{roman}

\begin{table}[hp]
\caption{Revision History} \label{TblRevisionHistory}
\begin{tabularx}{\textwidth}{llX}
\toprule
\textbf{Date} & \textbf{Developer(s)} & \textbf{Change}\\
\midrule
Oct. 21, 2024 & NFI, JL, AV, CG & TA Feedback\\
Oct. 23, 2024 & NFI, JL, AV, CG & Rev0\\
... & ... & ...\\
\bottomrule
\end{tabularx}
\end{table}

~\newpage

\tableofcontents

~\newpage

\pagenumbering{arabic}

\section{Introduction}

Sandlot is intended to be the successor for the current McMaster GSA
softball league scheduling and management platform. This project
will implement the current functionality from the existing system including
additional features such as a new login system, commissioner admin
privileges, and an improvement to the robustness, ease of use, and
maintainability to the platform.

Hazards are conditions of the product system that could lead to undesirable outcomes. 
In Sandlot, hazards of security, malfunctions, usability, among others will be analyzed.

\section{Scope and Purpose of Hazard Analysis}

Hazard analysis aims to identify potential hazards that may exist with the product, 
understand their causes and effects, and provide proactive strategies to mitigate
them.
The purpose of this document is to perform this analysis for Sandlot, investigate potential
hazards and give means of mitigating the issues identified. 
Hazards to Sandlot could take the form of user interface misinterpretations, misinputted data,
and database errors, leading to possible malfunctions with the product and/or user 
dissatisfaction.
This document will first identify the system components and boundaries, state the critical
assumptions, use a Failure Mode and Effects Analysis (FMEA) table to identify the hazards
and their causes, effects, and mitigation strategies, and lastly the requirements that
are discovered from the analysis.

\section{System Boundaries and Components}

\subsection{System Components}
The system can be divided into the following components:
\begin{itemize}
    \item Authentication
    \subitem This component authenticates user inputted login information with the database
    and determines whether a login is successful.
    \item Scheduling
    \subitem This component generates a schedule based on user inputted availability data. 
    Additionally, it allows for rescheduling events in these schedules after they 
    have been generated.
    \item Accounts
    \subitem This component allows users to create, delete or edit accounts. Additionally, account types 
    determine the ways in which a user can interact with the system, its data, and its components.
    \item Team Structure
    \subitem This component is a data structure of the system which allows accounts to be grouped together 
    such that a group of accounts interact with the system, its data, and its components in the same way as 
    each other. Additionally, accounts can be added/removed from teams, users can request to join a 
    certain team, and teams can be created/deleted.
    \item Scoring/Standings
    \subitem This component allows users to input score data associated with specific events and will 
    generate cumulative standings based on score data.
    \item Alerts
    \subitem This component alerts users of important information related to the system.
    \item Database
    \subitem This component stores the system's data.
    \item User Interface
    \subitem This component displays the system's data to the user.
\end{itemize}

\subsection{System Boundaries}
The system limits itself to the components mentioned before; the front end and backend of the website.
External APIs that connect to our system are considered as outside the system boundaries.
This would include the webserver, and weather tracker, among others. 

\section{Critical Assumptions}

\begin{itemize}
    \item It is assumed that users will not attempt to maliciously attack the system.
\end{itemize}

\section{Failure Mode and Effect Analysis}

\begin{landscape}
% breaks margins
\vspace*{-3.25cm}
\hspace*{-3.25cm}
% adds vertical padding to table
\def\arraystretch{1.3}
\begin{tabularx}{504pt}{|p{3cm}||p{4cm}|p{4cm}|p{4cm}|p{1cm}|p{5.75cm}|}
\cline{1-6}
\textbf{Component} & \textbf{Failure Mode} & \textbf{Failure Effect} & \textbf{Failure Cause} & \textbf{SR} & \textbf{Recommended Actions}\\
\cline{1-6}
\multirow{2}{2cm}{Authentication} & User login information does not match database stored
login information. & User cannot access their account. & Misinputted user login information.
& SR-7 & Misinputted user login information should give a warning to the user when
login information does not exist or is incorrect.\\
\cline{1-6}
\multirow{2}{2cm}{Scheduling} & Captain inputs incorrect availability dates at
start of season & Schedule is generated incorrectly. & Captain's user error. &
SR-1 & Allow captain to resubmit availability that overwrites previously
submitted availability if submitted before the due date.\\
\cline{2-6}
& User reads schedule incorrectly. & User may not attend a game by accident or
travel to a game at the wrong time. & Schedule is unclear, unreadable, or user
makes an error. & 22 & Ensure schedule data and structure is visible and
readable.\\
\cline{2-6}
& Teams availability data have scheduling conflicts. & System cannot make a
valid schedule. & Availability data has scheduling conflicts. & SR-8 & Teams
will be given a warning if their availability data has scheduling conflicts.\\
\cline{2-6}
& Captain doesn't respond to another captain's reschedule request. & Captain
who sent the reschedule request will not know if the opposing captain wants to
reschedule or not. & System doesn't adequately notify the opposing captain
about the reschedule request. & SR-5 & Make sure the captain who needs to
accept the request cannot avoid seeing the request notification.\\
\cline{1-6}
\multirow{2}{2cm}{Teams} & User is assigned to or joins the wrong team. & User
will be shown the wrong schedule and team information. & Interface inputs are
unclear or user makes an error. & 17 & Ensure that when assigning a team for a
user, the team options to select from are clear and the input section is
readable.\\
\cline{1-6}
\multirow{2}{2cm}{Accounts} & Account is given the wrong permissions. & User 
has access to actions they shouldn't use. & Interface lacks security or user
makes an error. & SR-2 & Giving an account additional permissions should give
a warning to the user when configuring accounts.\\
\cline{2-6}
& All commissioner level accounts are deleted. & There are no accounts left
that can give commissioner level permissions. & System allows for all
commissioner accounts to be deleted. & SR-3 & Do not let users delete a
commissioner account if there is only one left.\\
\cline{1-6}
\multirow{3}{2cm}{Scoring/ Standings} & Captain inputs the wrong score. &
Standings and database has incorrect scores. & User inputs the wrong score. &
SR-4 & Allow the opposing team's captain to verify the score and contest it if
the score is wrong.\\
\cline{2-6}
\cline{1-6}
\multirow{3}{2cm}{Alerts} & User reads alert incorrectly. &
User may travel to a game that was postponed or cancelled, or miss out on
critical information. & Alert is unclear, unreadable, or user makes an error. &
SR-9 & Ensure alert message is visible and readable.\\
\cline{2-6}
\hline
\end{tabularx}

\end{landscape}

\section{Safety and Security Requirements}

\wss{Newly discovered requirements.  These should also be added to the SRS.  (A
rationale design process how and why to fake it.)}

\begin{myreq}
    \reqno SR-1
    \reqdesc If the season start availability due date hasn't been reached, a
    captain should be able to resubmit availability data which overwrites
    previous availability data.
    \reqrat If the captain makes an error when submitting availability data
    they should be able to fix their error.
    \reqorig Alex Verity
    \reqfit If the captain has submitted data, they shall be able to overwrite
    it with new data.
    \twoinline
      {\reqsatis 4}
      {\reqdissat 3}
    \reqhist Created 2024-10-15
\end{myreq}

\begin{myreq}
    \reqno SR-2
    \reqdesc Giving permissions to users must be accompanied by an error that
    warns the user of the severity of the action.
    \reqrat Accidentally giving permissions to users who shouldn't have them
    could result in unexpected errors.
    \reqorig Alex Verity
    \reqfit If permissions are being changed, a warning shall be displayed to
    the user before updating the permissions.
    \twoinline
      {\reqsatis 5}
      {\reqdissat 5}
    \reqhist Created 2024-10-15
\end{myreq}

\begin{myreq}
    \reqno SR-3
    \reqdesc If there is only one commissioner level account, that account
    cannot be deleted.
    \reqrat Deleting this account would stop any more commissioner level
    accounts from being created, soft-locking the system.
    \reqorig Alex Verity
    \reqfit If there is only one commissioner level account it shall not be
    deleted.
    \twoinline
      {\reqsatis 3}
      {\reqdissat 3}
    \reqhist Created 2024-10-15
\end{myreq}

\begin{myreq}
    \reqno SR-4
    \reqdesc All match scores must be visible and contestable by other
    captains once recorded.
    \reqrat Match score correctness is extremely important for a competitive
    league, if a score is wrong, captains should be able to request that it be
    fixed.
    \reqorig Alex Verity
    \reqfit Any match scores shall have the option to be viewed and contested
    by captains.
    \twoinline
      {\reqsatis 3}
      {\reqdissat 3}
    \reqhist Created 2024-10-15
\end{myreq}

\begin{myreq}
  \reqno SR-5
  \reqdesc Captains are adequately notified when they receive an alert or a
  reschedule request.
  \reqrat Important information may be shared in alerts, and reschedule
  requests that go unanswered may be frustrating for captains. Notifications
  should reach their intended targets if possible.
  \reqorig Alex Verity
  \reqfit Assuming the user hasn't interfered, alerts and reschedule requests
  shall always be sent to a place the user receives notifications.
  \twoinline
    {\reqsatis 3}
    {\reqdissat 3}
  \reqhist Created 2024-10-15
\end{myreq}

\begin{myreq}
  \reqno SR-6
  \reqdesc Users can only access an account with correctly inputted login
  information that matches the database stored login information.
  \reqrat Login information stored in the database should correspond to a
  specific account. The account should only be accessed by the correctly
  inputted login information details.
  \reqorig Nicholas Fabugais-Inaba
  \reqfit Assuming the user has correctly inputted the login details for
  an account stored in the database, they should be granted access into
  the corresponding account.
  \twoinline
    {\reqsatis 3}
    {\reqdissat 3}
  \reqhist Created 2024-10-25
\end{myreq}

\begin{myreq}
  \reqno SR-7
  \reqdesc Misinputted user login information shall provide a warning to the
  user, if login information does not exist or does not match the database
  stored login information.
  \reqrat Login information stored in the database should match the user
  inputted login information. Feedback should be provided for the user to
  understand an error has occured when accessing an account with incorrect
  login details.
  \reqorig Nicholas Fabugais-Inaba
  \reqfit Assuming the user has misinputted the login details for
  an account stored in the database, they should be given a warning that
  notifies them about the login information being incorrect or not existing
  in the database.
  \twoinline
    {\reqsatis 3}
    {\reqdissat 3}
  \reqhist Created 2024-10-25
\end{myreq}

\begin{myreq}
  \reqno SR-8
  \reqdesc Teams will be given a warning if their availability data has
  scheduling conflicts.
  \reqrat The system should be able to create a valid schedule, in which
  teams do not have conflicting availability data that schedules games for
  the same dates and times as other teams.
  \reqorig Nicholas Fabugais-Inaba
  \reqfit Teams should receive a warning about their availability data
  conflicting on the schedule.
  \twoinline
    {\reqsatis 3}
    {\reqdissat 3}
  \reqhist Created 2024-10-25
\end{myreq}

\begin{myreq}
  \reqno SR-9
  \reqdesc Alerts sent to users must be visible and readable.
  \reqrat User may travel to a game that was postponed or cancelled, or miss
  out on critical information.
  \reqorig Nicholas Fabugais-Inaba
  \reqfit User receives an alert that is readable and clear enough for them
  to understand.
  \twoinline
    {\reqsatis 3}
    {\reqdissat 3}
  \reqhist Created 2024-10-25
\end{myreq}

\section{Roadmap}

The hazard analysis resulted in new safety and security requirements being
introduced to the project's already existing requirements located in the
SRS. Provided the time constraints and current project deadlines, critical
requirements will be initially addressed with lower priority requirements
being investigated in the future. The hazard analysis will be used as a guideline
to ensure the safety and security of the Sandlot platform's users.\\\\

Proof of Concept Demonstration (November 11-22):
\begin{itemize}
  \item Authentication: SR-6 and SR-7
  \item Scheduling: SR-1, SRS(22), SR-8
\end{itemize}

Final Demonstration (Revision 1) (March 24-30):
\begin{itemize}
  \item Scheduling: SR-5
  \item Teams: SRS(17)
  \item Accounts: SR-2 and SR-3
  \item Scoring/Standings: SR-4
  \item Alerts: SR-9
\end{itemize}

\newpage{}

\section*{Appendix --- Reflection}

\input{../Reflection.tex}

\begin{enumerate}
    \item What went well while writing this deliverable? 
    \item What pain points did you experience during this deliverable, and how
    did you resolve them?
    \item Which of your listed risks had your team thought of before this
    deliverable, and which did you think of while doing this deliverable? For
    the latter ones (ones you thought of while doing the Hazard Analysis), how
    did they come about?
    \item Other than the risk of physical harm (some projects may not have any
    appreciable risks of this form), list at least 2 other types of risk in
    software products. Why are they important to consider?
\end{enumerate}

\subsection*{Nicholas Fabugais-Inaba -- Reflection}

\begin{enumerate}
  \item What went well while writing this deliverable?\\\\
  When writing this deliverable, outlining the different failure modes and effects 
  helped in understanding what hazards could impact our project from more than just a
  surface level view. More specifically, content in components such as scheduling and
  alerts, aided in my understanding of how each user may be impacted by a type of failure
  that could occur within the system.
  \item What pain points did you experience during this deliverable, and how
  did you resolve them?\\\\
  Some of the pain points from this deliverable had to deal with brainstorming
  appropriate hazards that may impact a user in the system. With many working components,
  there are a lot of hazards that could occur or that could be easily handled. 
  Understanding which hazards could impact the users the most, which the team would then
  focus on to address primarily before other secondary hazards, was really crucial
  in the completion of this deliverable. These pain points were resolved
  by brainstorming together as a team our own failure modes that we would then review
  amongst each other, if a member needed a second opinion on the component they were
  working on. This would help all of us to understand if the hazard was crucial to
  address for the sake of the system users or if it could be dealt with in an alternative
  way, and therefore not need to be listed.
  \item Which of your listed risks had your team thought of before this
  deliverable, and which did you think of while doing this deliverable? For
  the latter ones (ones you thought of while doing the Hazard Analysis), how
  did they come about?
  \\\\
  Before this deliverable, one of the bigger risks our team had thought of before
  this deliverable was what we would need to do to address scheduling conflicts.
  We knew the scheduling portion of the system was an important aspect of this project
  and would need to be dealt with appropriately to have a robust system rather than the
  current platform that is prone to many errors. Some of the latter risks such as the
  alerts or the accounts risks came about as the team understood alerts could be misread,
  so to make it as easy as possible for users, the platform would need to make sure any
  announcements were to be clear and readable. Additionally, for account risks,
  on a rare occasion where all commissioner level accounts are deleted, we thought of
  making sure to put a safeguard so at the very least there could be one account, since
  users could make the mistake of deleting accounts by accident.
  \item Other than the risk of physical harm (some projects may not have any
  appreciable risks of this form), list at least 2 other types of risk in
  software products. Why are they important to consider?\\\\
  In software products, 2 other risks could be associated with a system shutdown from
  loss of battery life or some type of malware that could impact the software in
  a negative way. These would be important to consider as in the case for a failure
  from the system, there should be a failsafe to deal with such a mode as system integrity
  could be impacted as well as the user's device itself. In terms of malware, it could
  also impact the system's integrity leaving it vulnerable to impact users with faulty
  links that could damage a user's device. Malware could also crash the software system
  itself, which would be detrimental.

\end{enumerate}

\subsection*{Casra Ghazanfari -- Reflection}

\begin{enumerate}
  \item What went well while writing this deliverable?\\\\
  Writing the “System Components” section of this deliverable went well for our 
  team. This is because going into this deliverable we already had a concrete 
  idea of the structure of our overall system. This allowed us to quickly 
  partition the system into components based on our ideas for the overall 
  structure. Moreover, due to our existing understanding of the overall system 
  we were able to easily envision and determine how the derived components of 
  the system would interact with each other and the system boundaries.
  \item What pain points did you experience during this deliverable, and how
  did you resolve them?\\\\
  The biggest pain point during this deliverable was connecting hazards in our 
  FMEA table to security requirements (SRs). This is because our SRS deliverable 
  was incomplete at the time of writing this deliverable and did not have a 
  proper set of security requirements to connect to any of the hazards 
  identified in the FMEA. This significantly increased the amount of work 
  needed for us to properly complete this deliverable and generally slowed down 
  the process of creating the FMEA table as well.
  \item Which of your listed risks had your team thought of before this
  deliverable, and which did you think of while doing this deliverable? For
  the latter ones (ones you thought of while doing the Hazard Analysis), how
  did they come about?\\\\
  The team didn’t concretely identify any risks prior to working on this 
  deliverable, but had loose ideas of more obvious potential risks to include 
  in the deliverable. “The user incorrectly inputting their login information” 
  or “the user joining the wrong team” are examples of risks we had thought of 
  before working on this deliverable. Risks like “all commissioner accounts are 
  deleted” and “account is given wrong permissions” are examples of risks we 
  thought of while working through this deliverable. Generally, the idea to 
  include these risks came from carefully evaluating each individual component 
  defined previously in the document and asking ourselves the question: 
  “if anything could go wrong, what would go wrong?”. From this line of thinking 
  we were able to envision and identify uncommon risk scenarios for the system 
  that were tied to the system’s more complex components.
  \item Other than the risk of physical harm (some projects may not have any
  appreciable risks of this form), list at least 2 other types of risk in
  software products. Why are they important to consider?\\\\
  \begin{enumerate}
    \item Security risks\\
    Software products that handle sensitive data (i.e. personal / financial / 
    health information) must consider the risks of handling sensitive information. 
    Data can be stolen through unauthorized access and data breaches and can cause 
    financial and reputational harm to both the users and owners of the product. 
    Therefore, security risks such as data breaches are important to consider when 
    handling sensitive data, especially when trying to comply with data protection 
    regulations.
    \item Operational risks\\
    Software products that require high availability must consider the risks of 
    relying on services outside the product’s system boundary. For example, when 
    using a hosting service provided by another organization to host the product’s 
    website or database, There’s the inherent risk that if the organization hosting 
    the service faces issues, so will your product. Issues like these can cause 
    delays or system downtime for your product which is especially harmful if the 
    product requires high availability as downtime negatively impacts the user 
    satisfaction and experience of users of the product.
  \end{enumerate}

\end{enumerate}

\end{document}