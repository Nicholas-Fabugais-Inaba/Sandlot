\documentclass[12pt, titlepage]{article}

\usepackage{booktabs}
\usepackage{tabularx}
\usepackage{hyperref}
\hypersetup{
    colorlinks,
    citecolor=blue,
    filecolor=black,
    linkcolor=red,
    urlcolor=blue
}
\usepackage[numbers, round]{natbib}

\input{../Comments}
\input{../Common}

\begin{document}

\title{System Verification and Validation Plan for \progname{}} 
\author{\authname}
\date{\today}
	
\maketitle

\pagenumbering{roman}

\section*{Revision History}

\begin{tabularx}{\textwidth}{p{3cm}p{2cm}X}
\toprule {\bf Date} & {\bf Version} & {\bf Notes}\\
\midrule
Oct. 28, 2024 & 1.0 & TA Feedback\\
Nov. 1, 2024 & 1.1 & Rev0\\
Jan. 11, 2025 & 1.2 & Based on TA feedback: Fixed spelling/grammatical errors.
Quantified a good survey result in the description above the questions.\\
Apr. 1, 2025 & 1.3 & Based on TA feedback: Specified how data will be analyzed
and used to improve the product. Specified meaning of user documentation and
how user documentation will be validated.\\
Apr. 2, 2025 & 1.4 & Based on TA feedback: Added descriptions for functional
and non-functional tests.\\
Apr. 3, 2025 & 1.5 & Based on TA feedback: Added traceability matrix.\\
\bottomrule
\end{tabularx}

\newpage

\tableofcontents

\newpage

\section{Symbols, Abbreviations, and Acronyms}

\renewcommand{\arraystretch}{1.2}
\begin{tabular}{l l} 
  \toprule		
  \textbf{symbol} & \textbf{description}\\
  \midrule 
  SRS & Software Requirements Specification\\
  GSA & Graduate Students Association\\
  VnV & Verification and Validation\\
  CI & Continuous Integration\\
  T & Test\\
  \bottomrule
\end{tabular}\\

\newpage

\pagenumbering{arabic}

This document will outline the various verification and validation (VnV) plans
for the Sandlot project including general information about the system and a
verification plans for existing documents. System tests and unit tests will
also be included that will be used to validate the requirements created in the
SRS document.

\section{General Information}

\subsection{Summary}

The software being tested is the Sandlot project. Sandlot is a scheduling and
management platform for the McMaster GSA softball league. Users of the system
will include players, captains, commissioners, and other general users who will
not need to make an account to access the system. Sandlot is intended to be an
upgrade to the current platform that is outdated and lacks maintainability. This
project will build off the current platform's existing features such as game
scheduling, viewing of the scoring and standings, and team creation. Sandlot will
also add new features including account creation and commissioner specific
permissions like league-wide alerts.

\subsection{Objectives}

The main objectives intended to be accomplished are to build confidence in the software
correctness, demonstrate the requirements created for this project are correctly
implemented, and demonstrate adequate usability for functionalities of the
system. When demonstrating the correct implementations of requirements for the
project, these requirements are referring to the ones outlined in both the
SRS and Hazard Analysis documents.

Objectives that are out of scope due to the limitations in our resources for
verification and validation include the assumption that the database and the web
server the platform utilizes and the platform runs on, respectively, have already
been verified by its implementation teams.

\subsection{Challenge Level and Extras}

The challenge level of the project is general. The extras that are being used
are user documentation and usability testing. The challenge level and both extras
have been approved by an instructor.

User documentation will be a guide for users to understand the functions
of the system and how to properly interact with them. There will be step-by-step
procedures that show users how to perform certain actions as well as answers to any
frequently asked questions that may arise. Additionally, this user documentation
will be verified by users of the system including the supervisor and testers that
may request assistance on actions they may not fully understand.

\subsection{Relevant Documentation}

\begin{thebibliography}{9}
  \bibitem{SRS} Software Requirements Specification Document (2024)
  \href{../SRS-Volere/SRS.pdf}{SRS.pdf}
  \bibitem{DP} Development Plan Document (2024)
  \href{../DevelopmentPlan/DevelopmentPlan.pdf}{DevelopmentPlan.pdf}
  \bibitem{MG} Module Guide Document (2024)
  \href{../Design/SoftArchitecture/MG.pdf}{MG.pdf}
  \bibitem{MIS} Module Interface Specification Document (2024)
  \href{../Design/SoftDetailedDes/MIS.pdf}{MIS.pdf}
  \\
\end{thebibliography}

The SRS document has requirements for our project that must be verified, it is
the most complete list of functional and non-functional requirements that our
solution needs to achieve.\\

The Development Plan Document contains information about our plans on how to use 
automated testing and verification tools for the project. Specifically, it discusses
linting tools, unit testing frameworks, code coverage measuring tools, performance
measuring tools, and our plans for CI. All of which are important tools used to 
help us verify and validate our tests.\\

The MG and MIS documents also refer heavily to the requirements being tested
in our VnV plan.

\section{Plan}

In this section, our plan for verifying all important documents will be
recorded as well as our plan for verifying the software itself. Documents to
be verified include the SRS, MIS, and MG design documents and the VnV itself.

\subsection{Verification and Validation Team}

\begin{enumerate}
  \item Casra: Automation Tester\\
  Will focus on automation testing using GitHub Actions.
  \item Jung Woo: Non-Functional Tester\\
  Will focus on non-functional tests listed in section 4.2.
  \item Alex: Functional Tester\\
  Will focus on functional tests listed in section 4.1.
  \item Nicholas: Survey Tester\\
  Will be in charge of performing surveys needed for any non-functional tests.
  \item Dr. Jake Nease: Supervisor\\
  Will help give feedback on all functionality of the solution and gather
  stakeholders to help test the solution.
\end{enumerate}

\subsection{SRS Verification Plan}

A meeting with the team and supervisor will be held to verify the SRS document covers all of
the necessary requirements desired for the system. Each requirement outlined in the SRS will
be reviewed by the team and supervisor to ensure the requirement appropriately addresses the
requested functionality.

The SRS document will also be reviewed by another capstone group who will give
at least six points of feedback on our SRS document. Our team will review the
feedback and make any changes needed.

\subsection{Design Verification Plan}

A meeting with the team and supervisor will be held to review the design documents created
for the Sandlot project. Both parties will then verify anything outlined in the design
documents are correctly implemented in the system.

An additional meeting with the team will be held to verify any and all reviews created by
classmates are either implemented in the system or appropriately addressed by a team member
with a justification for why a suggestion should not be implemented.

\subsection{Verification and Validation Plan Verification Plan}

The verification and validation plan will be reviewed by another capstone
group who will give at least six points of feedback on our VnV document. Our
team will review the feedback and make any changes needed. 

During the writing of the verification and validation plan, any questions or
concerns for the supervisor will be recorded and asked during our next
scheduled supervisor meeting.

During testing, we will use mutation testing to verify test case coverage, and
any holes found in our coverage will be patched by adding more test cases.

\subsection{Implementation Verification Plan}

Code reviews with the team will be held for all functionalities of the system, utilizing the
system tests and unit tests outlined in the VnV plan. The team will attempt to identify any
errors within the system and correct any faults that may occur.

Furthermore, usability testing and user documentation will be done by the team and
reviewed by the supervisor. Both the usability test and user documentation must highlight
all functionalities of the system and how they are implemented. The supervisor will then
verify if the usability test and user documentation adequately supply the necessary
information for an admin or developer to understand the full solution. Their understanding
should allow them to sufficiently maintain the system and add any new functionalities as
required.

\subsection{Automated Testing and Verification Tools}

This was discussed in our development plan document \cite{DP} 
in the expected technology section.

\subsection{Software Validation Plan}

% The system will be provided to the supervisor and external testers, gathered by the supervisor,
% to enter given inputs into the system. The given inputs shall produce expected outputs according to
% the system tests created from the SRS. For specific system tests, a usability survey will be
% provided along with the test that must be answered by the individual conducting the test.

The system will be provided to external testers to enter given inputs into the system. 
The given inputs shall produce expected outputs according to the system tests created 
from the SRS. For specific system tests, a usability survey will be provided along 
with the test that must be answered by the individual conducting the test.

The data from the usability survey will then be analyzed for certain system actions such as
scheduling, rescheduling, system performance, ease of use, and more. This analysis will account for
metrics not deemed a "satisfactory result", see section 6.2 for details, considering them
an issue that should be dealt with. These issues should then be elaborated upon in the
survey specifically in the additional feedback section under each issue's question or at the end
of the survey. These suggestions will then be implemented in the system to improve the product.

\section{System Tests}

This section will list all test cases for both functional and non-functional
requirements, split into distinct areas of testing. The goal of the test cases
is to have full coverage over all possible errors. This section will likely be
added to in the future, as holes in coverage are found and new test cases are
made.

\subsection{Tests for Functional Requirements}

This section covers tests verifying all functional requirements in our SRS
document \cite{SRS}. Areas of testing include scheduling, accounts, team
structure, scoring/standings, and alerts. These areas cover all major
functionality of our solution.

\subsubsection{Scheduling}

This section includes all functional tests related to viewing, creating, and
modifying the league schedule. The schedule includes all future and past
matches created using each team's availability at the start of the season and
can be modified with the reschedule feature.

\begin{enumerate}

  \item{test-FR8\\}
  
  Description: Before the season starts, captains must have the
  option to give their team’s availability for the season.

  Control: Manual

  Initial State: The system has provided the option for captains to enter their team
  availability data and is ready to take in the user's input.

  Input: Non-conflicting team availability data.

  Output: Captain inputted team availability data is stored in the system.

  Test Case Derivation: The team availability data inputted by the captain has been
  determined to not conflict with other availability data already stored in the system.

  How test will be performed: The tester will select the option to enter in team 
  availability data and be provided non-conflicting team availability
  data that they will input into the system. They will then observe if the system successfully
  accepts the inputted team availability data and if any errors occur within the system.

  % \item{test-FR9\\}

  % Control: Automatic

  % Initial State: The current season's due date for inputting availability
  % has been reached.

  % Input: The current season's inputted team availability data.

  % Output: A season schedule is generated by the system and a visual
  % representation of the schedule is generated and displayed on the website.

  % Test Case Derivation: Players should be able to quickly know when and where
  % they're playing their games shortly after they've submitted their availability.

  % How test will be performed: A wide set of availability data sets will be
  % submitted to the schedule generation feature, with the expected results
  % compared to the results given by the program.

  \item{test-FR10\\}
  
  Description: Captains must be able to request a rescheduling
  for their team’s upcoming games as long as the game is at
  least 24 hours in the future. Rescheduling involves giving
  a list of alternate dates which are sent to the opposing
  team’s captain to choose a date and accept.

  Control: Manual

  Initial State: There are two captains whose teams are scheduled to play a
  game in the future.

  Input: One captain submits a reschedule request.

  Output: The other captain receives a reschedule request.

  Test Case Derivation: A captain should receive a reschedule request if another
  captain submits a request on a game both captains will be playing.

  How test will be performed: The tester will check if a reschedule request is
  successfully sent when a captain requests a reschedule. This will be checked
  for at least 3 different dates and times.

  \item{test-FR11-1\\}
  
  Description: When a captain receives a rescheduling request,
  the system should prompt the captain to accept
  the request and choose a date from the list of alternate
  dates.

  Control: Manual

  Initial State: A reschedule request has been sent to a captain.

  Input: A captain accepts a reschedule request.

  Output: The game is rescheduled to that time.

  Test Case Derivation: A captain should be able to accept a reschedule
  request.

  How test will be performed: The tester will accept multiple reschedule
  requests on different dates and times and verify the correct output is made by
  the system.

  \item{test-FR11-2\\}
  
  Description: When a captain receives a rescheduling request,
  the system should prompt the captain to deny the request.

  Control: Manual

  Initial State: A reschedule request has been sent to a captain.

  Input: A captain denies a reschedule request.

  Output: The captain who made the request is notified of the denial.

  Test Case Derivation: A captain should be able to deny a reschedule
  request.

  How test will be performed: The tester will deny multiple reschedule
  requests on different dates and times and verify the correct output is made by
  the system.

  \item{test-FR12\\}
  
  Description: When a captain’s rescheduling request is either
  accepted or denied the system should notify them about
  the outcome.

  Control: Manual

  Initial State: The system has received a captain's reschedule request and is ready to
  notify them about the outcome of their request.

  Input: Captain reschedule request.

  Output: A notification is sent by the system to the captain about the status of their
  reschedule request.

  Test Case Derivation: The notification about the status of a captain's reschedule
  request is immediately sent to the captain once the status is confirmed to be either
  accepted or denied by the system.

  How test will be performed: The tester will be provided both a valid and invalid captain
  reschedule request to submit into the system and wait to observe the notification sent
  by the system of the status for the reschedule request. They will then observe if the
  system has both accepted and denied the submitted reschedule requests and if any errors
  occur within the system. 

  \item{test-FR18\\}
  
  Description: Commissioners should be able to edit the league
  schedule.

  Control: Manual

  Initial State: The system has created a league schedule and is ready to
  accept new schedule data.

  Input: New schedule data.

  Output: The updated league schedule according to the new schedule data inputted.

  Test Case Derivation: The league schedule is immediately updated and shows the new changes
  once new schedule data is inputted into the system.

  How test will be performed: The tester will be provided admin level permissions and sample
  schedule data to submit into the system and wait to observe the updated league schedule
  by the system once the new schedule data is received. They will then observe if the
  system has displayed the correct updated league schedule and if any errors had
  occurred within the system.

  \item{test-FR20\\}
  
  Description: The system must display a schedule that includes
  all past and future games from all teams on a calendar.

  Control: Manual

  Initial State: The system has created a season schedule.

  Input: Season schedule.

  Output: Schedule for all the games of all the teams during the season in a calendar. 

  Test Case Derivation: The overall season schedule should display all games in a calendar.

  How test will be performed: The season schedule should resemble a calendar and display all games of the season.

  \item{test-FR21\\}
  
  Description: The system must display a team schedule for each
  team that includes all the team’s past and future games.

  Control: Manual

  Initial State: The system has created a schedule.

  Input: User navigating to a team's schedule section.

  Output: The system will display all games of the specified team's schedule.

  Test Case Derivation: Each team should have a schedule that lists all of
  that team's games. This should be accessible by users who navigate to this
  section of the system.

  How test will be performed: Each team's schedule section will be navigated to
  and compared with the full league schedule. All games on the main schedule
  that include the specified team should be on the team's schedule. There should
  not exist any games on the team's schedule where the specified team is not
  playing.

  % \item{test-FR22\\}

  % Control: Manual

  % Initial State: The system has created a schedule and is ready to display upcoming games from
  % an accepted time interval.

  % Input: Time interval.

  % Output: The upcoming games in a schedule according to the time interval inputted.

  % Test Case Derivation: The schedule shows all upcoming games within the inputted time interval
  % that is received by the system.

  % How test will be performed: The tester will be provided a sample time interval to input
  % into the system and wait to observe all of the upcoming games from the displayed schedule.
  % They will then observe if the system has displayed the correct schedule based on the
  % inputted time interval and if any errors had occurred within the system. 

\end{enumerate}

\subsubsection{Accounts}

This section includes all functional tests related to the creation, use and
modification of accounts.

\begin{enumerate}

  \item{test-FR1-1\\}
  
  Description: System must display the season schedule.

  Control: Manual

  Initial State: System is open on the user's browser.

  Input: User requests to display the season schedule.

  Output: Season schedule is displayed.

  Test Case Derivation: The season schedule should be displayed to all users
  regardless of their access level.

  How test will be performed: Users will attempt to display
  the season schedule.

  \item{test-FR1-2\\}
  
  Description: System must display the standings.

  Control: Manual

  Initial State: System is open on the user's browser.

  Input: User requests to display the standings.

  Output: Standings are displayed.

  Test Case Derivation: Standings should be displayed to all users
  regardless of their access level.

  How test will be performed: Users will attempt to display
  the standings.

  \item{test-FR3-1\\}
  
  Description: Users should be able to create a new account by
  providing valid account information.

  Control: Manual

  Initial State: The system is not logged in to an account.

  Input: User navigates to create an account and enters valid account creation
  data.

  Output: The system adds an account to the database and logs in to the new
  account.

  Test Case Derivation: If valid account information is given a new account
  should be created.

  How test will be performed: Multiple accounts will be added to the system
  with differing valid account data covering all input fields.

  \item{test-FR3-2\\}
  
  Description: Users should not be able to create a new account by
  providing invalid account information.

  Control: Manual

  Initial State: The system is not logged in to an account.

  Input: User navigates to create an account and enters invalid account
  creation data.

  Output: The system does not create a new account and the user is informed of
  which data is invalid.

  Test Case Derivation: If invalid account information is given a new account
  should not be created and the user should be notified of which data is
  invalid.

  How test will be performed: Multiple attempts to create accounts with
  differing invalid account data will be made, with invalid data each attempt
  covering different input fields.

  \item{test-FR4\\}
  
  Description: Users should be able to change their account in-
  formation by providing the necessary information.

  Control: Manual

  Initial State: The system is set up and ready to take in the user's input.

  Input: Valid account information.

  Output: User inputted account information has replaced the previously displayed
  account information.

  Test Case Derivation: The account information inputted by the user has already been
  determined to be valid and should not cause the system to run into any errors. The
  user inputted account information, although should be correct, is not required by
  the system to be correct to change the user's previous account information.

  How test will be performed: The tester will be provided valid account information
  that they will input into the system and observe if the system successfully
  accepts the inputted account information. The tester will also observe if the previously
  existing account information has been changed to the information that had been entered
  at the start of the test. At any point during the test, the tester will also observe if
  any errors occur within the system.

  \item{test-FR5-1\\}
  
  Description: Users should be able to delete their account by
  providing valid login information for that account.

  Control: Manual

  Initial State: User is logged into an account.

  Input: User requests to delete their account and provides valid login
  information for that account.

  Output: The user is logged out, and their account is deleted.

  Test Case Derivation: If valid login information is submitted then the
  account should be deleted if the user requests to do so because if the user
  wishes to leave the league they should be able to delete their information
  from the system.

  How test will be performed: Users of all account types will attempt to delete
  their account using valid login information.

  \item{test-FR5-2\\}
  
  Description: Users should not be able to delete their account by
  providing invalid login information for that account.

  Control: Manual

  Initial State: User is logged into an account.

  Input: User requests to delete their account and provides invalid login
  information for that account.

  Output: The user is not logged out, their account is not deleted, and
  the user is told why their request was unsuccessful.

  Test Case Derivation: If invalid login information is submitted then the
  account should not be deleted because accounts should only be able to be
  deleted if the security protections in place are satisfied. Additionally,
  the user should be informed about why the request failed because if the
  user requesting account deletion is the proper owner of the account but is
  facing issues they should be provided information to help them troubleshoot
  the issues.

  How test will be performed: Users of all account types will attempt to delete
  their account using invalid login information.

  \item{test-FR16-1\\}
  
  Description: Account holders should be able to log in to Sand-
  lot with valid login data.

  Control: Manual

  Initial State: The system is not logged in to an account.

  Input: User navigates to the login section and enters valid login data for
  an account that has already been made.

  Output: The system logs in to the valid account.

  Test Case Derivation: If valid account information is given when logging in,
  the system should log in as that account.

  How test will be performed: Multiple accounts will be logged into each with
  different valid account information and permission levels.

  \item{test-FR16-2\\}
  
  Description: Account holders should not be able to log in to Sand-
  lot with invalid login data.

  Control: Manual

  Initial State: The system is not logged in to an account.

  Input: User navigates to the login section and enters invalid login data
  for an account login.

  Output: The system warns the user the data used is invalid.

  Test Case Derivation: If invalid account information is given when logging
  in, the system should not log in to any account and warn the user the data
  isn't valid.

  How test will be performed: Multiple invalid logins will be attempted each
  with different invalid account information.

  % \item{test-FR19\\}

  % Control: Manual

  % Initial State: The User is logged into a commissioner account.

  % Input: The user requests to assign captain level permissions to a 
  % certain user.

  % Output: The system assigns captain level permissions to the user.

  % Test Case Derivation: commissioners should be able to assign captain level
  % permissions to users because they control which players are captains.

  % How test will be performed: Users logged into commissioner accounts will attempt
  % to assign captain level permissions to a player level account.

\end{enumerate}

\subsubsection{Team Structure}

This section includes all functional tests related to teams and team
information. This covers team creation, users joining teams, and team
information being modified.

\begin{enumerate}

  % \item{test-FR2-1\\}

  % Control: Manual

  % Initial State: Captain user is logged in and the subject team has not been 
  % created in the system.

  % Input: Captain user navigates to team creation and inputs valid team 
  % information then submits. 

  % Output: Submitted team is added to the database.

  % Test Case Derivation: If valid team information is submitted, the team
  % should be created and the information should be reflected in the database.

  % How test will be performed: Captain accounts will create multiple
  % valid teams with differing data that covers all input fields.

  % \item{test-FR2-2\\}

  % Control: Manual

  % Initial State: Captain user is logged in.

  % Input: Captain user navigates to team creation and inputs invalid team 
  % information then submits. 

  % Output: No team is added to the database and the user is given informative 
  % feedback as to why the team submission had failed.

  % Test Case Derivation: If invalid team information is submitted, the team
  % should not be created and the user should be told the reason as to why.

  % How test will be performed: Captain accounts will attempt to create 
  % multiple invalid teams with differing data that covers all input fields. 

  % \item{test-FR7\\}

  % Control: Automatic

  % Initial State: The system is logged into a commissioner level account.

  % Input: New team data to replace a current team's data.

  % Output: If data is valid, the system should replace the old team data with the new
  % team data. If the data is invalid, the system should not change the old team
  % data and inform the commissioner the data is invalid.

  % Test Case Derivation: A commissioner level account should be able to change
  % any team's data, including player list and scores.

  % How test will be performed: A set of valid and invalid test cases will be
  % submitted to the commissioner's replace team data feature, with the expected
  % results compared to the results given by the program.

  % \item{test-FR13-1\\}

  % Control: Manual

  % Initial State: The User is logged into a captain account.

  % Input: The user requests to update the team information of the team they are
  % the captain of.

  % Output: The system provides the user with a menu to change the team's
  % information and will save any changes made by the user.

  % Test Case Derivation: Captains of teams should be able to update the team
  % information of their team because certain team information may have changed
  % since the team was created.

  % How test will be performed: Users logged into captain accounts will attempt
  % to request to update the team information of the team they are the captain of.

  % \item{test-FR13-2\\}

  % Control: Manual

  % Initial State: The User is logged into a captain account.

  % Input: The user requests to update the team information of a team they are
  % not the captain of.

  % Output: The system informs the user that they are not the captain of this team
  % and therefore cannot update this team's information.

  % Test Case Derivation: The team information of a team should only be able to be
  % updated by the captain of that team, not captains of other teams.

  % How test will be performed: Users logged into captain accounts will attempt
  % to request to update the team information of a team they are not the captain
  % of.

  \item{test-FR15\\}
  
  Description: Players should be able to join a team.

  Control: Manual

  Initial State: The system is logged into a player level account.

  Input: Join team interaction.

  Output: Player is added to the team. This is reflected in team composition
  and alerts sent to that team.

  Test Case Derivation: A player who has joined a team should be shown to be a
  member of that team in the system.

  How test will be performed: Multiple player accounts will attempt to join 
  different teams. Team compositions will be inspected to see if the player
  is shown to be a member. 

  % \item{test-FR23\\}

  % Control: Manual

  % Initial State: At least one team has been created by a captain.

  % Input: A captain level user is made the associated captain of a team that
  % already has an associated captain.

  % Output: Feedback to the user that the team already has a captain in charge.

  % Test Case Derivation: Only one captain is allowed to be associated with a
  % team due to league rules.

  % How test will be performed: A team with an associated captain will attempt
  % to have another captain be associated with it. It is important to note that
  % a captain playing for a different team than the one they are in charge of
  % does not count as being associated, they are just playing for the team.

\end{enumerate}

% \subsubsection{Scoring/Standings}

% This section includes all functional tests related to the scoring and
% league standings. It mainly focuses on captains recording the scores of
% matches their team plays.

% \begin{enumerate}

  % \item{test-FR14-1\\}

  % Control: Manual

  % Initial State: The User is logged into a captain account.

  % Input: The user requests to submit the score for a game that has been
  % completed.

  % Output: The system provides the user with a menu to submit the game's
  % score and will save the score inputted by the user.

  % Test Case Derivation: Captains of teams should be able to submit the score
  % for a game that has been completed because they're responsible for the
  % score.

  % How test will be performed: Users logged into captain accounts will attempt
  % to submit a score for a game that has been marked as completed by the system.

  % \item{test-FR14-2\\}

  % Control: Manual

  % Initial State: The User is logged into a captain account.

  % Input: The user requests to submit the score for a game that has not
  % been completed.

  % Output: The system informs the user that they cannot submit a score
  % for this game because it has not been completed.

  % Test Case Derivation: Captains of teams should not be able to 
  % submit the score for a game that has not been completed because 
  % the final score for that game has not been determined yet.

  % How test will be performed: Users logged into captain accounts will attempt
  % to submit a score for a game that has not been marked as completed by 
  % the system.

  % \item{test-FR14-3\\}

  % Control: Manual

  % Initial State: The User is logged into a player account.

  % Input: The user requests to submit the score for a game that has
  % been completed.

  % Output: The system informs the user that they are not the captain 
  % of this team and therefore cannot update the score for this game.

  % Test Case Derivation: regular players of a team should not be able to 
  % submit the score for a game because that is not their responsibility.

  % How test will be performed: Users logged into player accounts will 
  % attempt to submit a score for a game that has been marked as completed 
  % by the system.

% \end{enumerate}

% \subsubsection{Alerts}

% This section includes all functional tests related to the creation of alerts.
% Alerts can be created by commissioners and send information to any specified
% group of users of the solution.

% \begin{enumerate}

  % \item{test-FR6-1\\}

  % Control: 

  % Initial State: The system is logged into a commissioner level account.

  % Input: Valid alert data and chosen target user(s) entered.

  % Output: Alerts are stored in the system. Alerts are viewable by target users.

  % Test Case Derivation: Alerts sent to target user should be able to be 
  % seen by those target users.

  % How test will be performed: Different commissioner level accounts input
  % various valid alert data and submit them. Target user accounts should be checked
  % if the alerts are received.

  % \item{test-FR6-2\\}

  % Control: 

  % Initial State: The system is logged into a commissioner level account.

  % Input: Invalid alert data or invalid chosen target user(s) entered.

  % Output: User is given informative feedback as to why the alert had failed to submit.

  % Test Case Derivation: Invalid alerts should not be sent and the user should
  % be told as to why they could not be sent.

  % How test will be performed: Commissioner level account inputs various invalid
  % alert data covering all input fields.

% \end{enumerate}

\subsection{Tests for Nonfunctional Requirements}

This section covers tests verifying all non-functional requirements in our SRS
document \cite{SRS}. Most prominent areas of testing include usability,
performance, maintainability, security, and cultural.

\subsubsection{Look and Feel Requirements}

\begin{enumerate}

  % \item{test-AP1\\}

  % Type: Non-Functional, Dynamic, Manual

  % Initial State: The solution is open on the user's browser.

  % Input: The user navigates through the website. 

  % Output: All input elements displayed have a minimum size of 44x44 pixels, 
  % maintain and maintain at least a 3:1 contrast ratio with the background.

  % How test will be performed: Each possible user input element will be checked by 
  % the user to have a minimum size of 44x44 pixels, maintain and maintain at 
  % least a 3:1 contrast ratio with the background.

  % \item{test-AP2\\}

  % Type: Non-Functional, Dynamic, Manual

  % Initial State: The system is logged in.

  % Input: Inputs associated with each different type of input field in the
  % solution.

  % Output: Feedback from each input field.

  % How test will be performed: Each possible type of input will be used by the
  % tester and they will monitor the result for feedback. Each input field will
  % be tested with valid and invalid data if applicable, and invalid data should
  % receive feedback that lets the user know the input was invalid.

\item{test-AP3\\}

Description: All images and visuals made for/by Sandlot should
be high quality.

Type: Non-Functional, Dynamic, Manual

Initial State: The solution is opened on a user's web browser.

Input/Condition: The user will be asked if the images made for/by Sandlot are viewed at
a high quality containing no pixelations or blurring at their displayed size.

Output/Result: The supervisor will state that Sandlot's images are displayed at a high
quality.

How test will be performed: The supervisor will be provided the solution and a set of
sample inputs for Sandlot. They will then enter in the sample inputs and observe the
generated outputs from the system. After their observations, the supervisor will be given
a usability survey to fill out that is located in section 6.2 of this VnV plan document.

\item{test-AP4\\}

Description: Navigation should be straightforward, with menus
and links easily accessible and readable.

Type: Non-Functional, Dynamic, Manual
					
Initial State: The solution is opened on a user's web browser.
					
Input/Condition: The user will be asked if the navigation of Sandlot is straightforward 
and if menus and links are easily accessible and readable.
					
Output/Result: The user will state that Sandlot's navigation is straightforward.
					
How test will be performed: The user will be provided the solution. They will then 
navigate through the system and observe its simplicity. After their observations, 
the supervisor will be given a usability survey to fill out that is located in 
section 6.2 of this VnV plan document.

\item{test-STY1\\}

Description: The solution should use the same colours, fonts
and buttons across the entire user interface.

Type: Non-Functional, Dynamic, Manual
					
Initial State: The system is running and viewable.
					
Input/Condition: User will be asked if there are inconsistent colours,
fonts, or buttons across the interface of the system.
					
Output/Result: Supervisor will state that the style is consistent.
					
How test will be performed: Reviewer will navigate through all parts
of the interface and inspect text, colours, and buttons. Any 
inconsistent interface elements will be recorded. 

\end{enumerate}

\subsubsection{Usability and Humanity Requirements}

\begin{enumerate}

  \item{test-EU1\\}
  
  Description: All users must be able to easily find the season
  schedule.

  Type: Non-Functional, Dynamic, Manual

  Initial State: The system is not logged in.

  Input: Navigation inputs leading to the full season schedule.

  Output: The season schedule displayed on the screen.

  How test will be performed: At least 5 testers unfamiliar with the system will
  attempt to navigate to the season schedule. Their number of clicks used and
  time taken to get to the season schedule will be recorded. The test passes
  if on average testers take less than 2 clicks and less than one minute to
  find the schedule.

  \item{test-EU2\\}
  
  Description: Misinputted user login information shall provide
  a warning to the user, if login information does not exist
  or does not match the database stored login information.

  Type: Non-Functional, Dynamic, Manual

  Initial State: User is located on the system's login page and the system is ready
  for the user's inputs.

  Input/Condition: Misinputted login information.

  Output/Result: The system will provide a warning to the user for login information that
  does not exist or does not match any database stored login information.

  How test will be performed: The tester will be provided login information that does not
  currently exist in the database and they will input the provided information into the
  system. The tester will observe the output or any errors that may occur in the system.

  % \item{test-EU3\\}
  
  % Type: Non-Functional, Dynamic, Manual
            
  % Initial State: User is logged into a captain account.
            
  % Input/Condition: User inputs availability data which causes a schedule conflict.
            
  % Output/Result: The system will provide a warning to the user informing them that
  % the availability data they've inputted has caused conflicts with another team's
  % availability data that the scheduler cannot resolve.
            
  % How test will be performed: The user will be provided a captain account with a set 
  % of availability data. The system will have stored some set of existing availability 
  % data that will cause an unresolvable conflict with the availability data of the user.
  % The user will input the provided availability data to the system and observe the 
  % output or any errors that may occur in the system.

  % \item{test-LR1\\}

  % Type: Non-Functional, Dynamic, Manual

  % Initial State: The system is logged in to a commissioner level account.

  % Input/Condition: Inputs for all features available to commissioner 
  % level accounts.

  % Output/Result: The results of using each commissioner level account 
  % feature.

  % How test will be performed: A tester will be given a list of tasks to
  % complete that involve commissioner specific actions including sending 
  % alerts and editing team composition. 
  % The tester will also be given the user manual. If the user can
  % complete all tasks in less than one hour the test is successful.

  % \item{test-LR2\\}

  % Type: Non-Functional, Dynamic, Manual

  % Initial State: The system is logged in to a captain level account.

  % Input: Inputs for all features available to captain level accounts.

  % Output: The results of using each captain level account feature.

  % How test will be performed: A tester will be given a list of tasks to
  % complete including creating a team, modifying team data, and submitting a
  % score. The tester will also be given the user manual. If the user can
  % complete all tasks in less than one hour the test is successful.

  \item{test-LR3\\}
  
  Description: A new user should be able to use the basic fea-
  tures (such as viewing the league schedule) of the prod-
  uct without having prior knowledge of the product.

  Type: Non-Functional, Dynamic, Manual

  Initial State: The solution is opened on a user's web browser.

  Input/Condition: A new user will be asked to navigate to the season schedule on their first
  time interacting with the solution.

  Output/Result: A new user is able to successfully navigate to the season schedule on
  their first time interacting with the solution.

  How test will be performed: A new user of the system will be provided the solution and
  will be asked to navigate and view the season schedule.

  \item{test-UP1\\}
  
  Description: Any terminology or symbols used are the same as
  ones used in the past solution.

  Type: Non-Functional, Dynamic, Manual

  Initial State: The system is running.

  Input/Condition: User will be asked to navigate through the system to 
  selected areas of text for them to read.

  Output/Result: Users will be able to understand terminology used in
  the selected tests without difficulty. If 90 percent of users do understand,
  the test is considered a success.

  How test will be performed: A set of testers will read selected text
  in the system and record whether they generally do or do not 
  understand the terminology used. 

  % \item{test-AC1\\}

  % Type: Non-Functional, Dynamic, Manual

  % Initial State: The solution is open on the user's browser.

  % Input: The user navigates through the website. 

  % Output: All body text displayed has a minimum font size of 16 
  % pixels, and a line length between 45 and 75 characters.

  % How test will be performed: All possible body text will be checked by 
  % the user to have a minimum font size of 16 pixels, and a line length 
  % between 45 and 75 characters.

  % \item{test-AC2\\}

  % Type: Non-Functional, Dynamic, Manual

  % Initial State: The system is logged in to a commissioner level account.

  % Input: Navigation to each page of the system.

  % Output: The colours on the screen for each page of the system.

  % How test will be performed: Each page of the system will be viewed, and any
  % adjacent colours will be confirmed to have a ratio of at least 4.5:1.
  % Particular attention will be given to input fields and critical information
  % such as schedules.

  % \item{test-AC3\\}

  % Type: Non-Functional, Dynamic, Manual

  % Initial State: The solution is opened on a user's web browser and a commissioner's web
  % browser.

  % Input/Condition: A user will be sent an alert created by a commissioner and will be asked
  % if the alert is visible and readable.

  % Output/Result: A user is able to successfully see the alert sent by a commissioner and
  % the alert is readable and clear enough for the user to understand.

  % How test will be performed: The supervisor and a tester will be provided the solution and
  % will be asked to view/send an alert. Once the alert is sent/received, the recipient will
  % be asked to observe the alert. After their observations, the supervisor will be given
  % a usability survey to fill out that is located in section 6.2 of this VnV plan document.

\end{enumerate}

% \subsubsection{Performance Requirements}

% \begin{enumerate}

  % \item{test-SL1\\}

  % Type: Non-Functional, Dynamic, Manual

  % Initial State: The solution is not loaded, internet connection is stable.

  % Input/Condition: The system is started.

  % Output/Result: Sandlot loads and displays the homepage within 3 seconds.

  % How test will be performed: The system will be timed from not loaded to a
  % fully loaded homepage. This will be recorded 10 times. If the average load
  % time is less than 3 seconds the test is successful.

  % \item{test-CR1\\}

  % Type: Non-Functional, Dynamic, Manual

  % Initial State: The solution is opened on a user's web browser.

  % Input/Condition: Sandlot should have 60 teams and 1500 players stored in the system at
  % once.

  % Output/Result: Sandlot should continue to successfully function with no faults when
  % 60 teams have 25 players each stored in the system.

  % How test will be performed: The system will be provided sample inputs for 60 teams and
  % 25 players on each team. The tester will then observe the system as they navigate the
  % solution and utilize functionalities such as rescheduling games, and viewing the standings
  % and season schedule. After their observations, the tester will be given
  % a usability survey to fill out that is located in section 6.2 of this VnV plan document.

  % \item{test-CR2\\}

  % Type: Non-Functional, Dynamic, Manual

  % Initial State: The system is online.

  % Input/Condition: Sandlot will have 500 users using the system at the same
  % time.

  % Output/Result: Sandlot should continue to successfully function with no
  % faults when 500 users are using the system at once.

  % How test will be performed: 500 users will be mimicked using a solution like
  % JMeter. All functions on the system would be run through to make sure no
  % faults occur.

% \end{enumerate}

\subsubsection{Operational and Environmental Requirements}

\begin{enumerate}

  % \item{test-IAS1\\}

  % Type: Non-Functional, Dynamic, Manual

  % Initial State: Sandlot is running.

  % Input/Condition: User will attempt to access the system from multiple 
  % browsers with each of the latest four releases of the product.

  % Output/Result: User is able to access and use the system from multiple
  % browsers for the latest four releases of the product.

  % How test will be performed: Tester will attempt to use the product on
  % the five most popular web browsers for each of the latest four 
  % releases of the product. The user should be able to use the product
  % with no issues on each browser for each version of the product.

  \item{test-RR1\\}

  No test needed for requirement.

\end{enumerate}

% \subsubsection{Maintainability and Support Requirements}

% \begin{enumerate}

  % \item{test-MR1\\}

  % Type: Non-Functional, Dynamic, Manual

  % Initial State: The solution is opened on a user's web browser.

  % Input/Condition: A tester will be asked to start a new season within a time constraint
  % of one hour.

  % Output/Result: A new season is successfully started within one hour of the solution
  % being initially opened in the tester's web browser.

  % How test will be performed: A tester will be provided the solution and asked to start
  % a new season by following a set of instructions. A timer will begin at the same time
  % the solution is opened on the tester's web browser, in which, the tester should be
  % able to start a new season before one hour has passed.
  
% \end{enumerate}

\subsubsection{Security Requirements}

\begin{enumerate}

  \item{test-AS1-1\\}
  
  Description: Viewing the league schedule is accessible by any users of
  the system, logged in or not.

  Control: Manual

  Initial State: System is not logged in to an account.

  Input: User requests to display the season schedule.

  Output: Season schedule is displayed.

  How test will be performed: Users without accounts will attempt to display
  the season schedule.

  \item{test-AS1-2\\}
  
  Description: Viewing the league standings is accessible by any users of
  the system, logged in or not.

  Control: Manual

  Initial State: System is not logged in to an account.

  Input: User requests to display the standings.

  Output: Standings are displayed.

  How test will be performed: Users without accounts will attempt to display
  the standings.

  % \item{test-AS2-1\\}

  % Type: Non-Functional, Dynamic, Manual

  % Initial State: The system is logged in to a player level account.

  % Input/Condition: Player attempts to access the contact information of their 
  % team captain.

  % Output/Result: Captain's contact information is displayed.

  % How test will be performed: Various player level accounts from various teams 
  % will attempt to access the contact information of their respective team
  % captains. Players should be able to access the information.

  % \item{test-AS2-2\\}

  % Type: Non-Functional, Dynamic, Manual

  % Initial State: The system is logged in to a player level account.

  % Input/Condition: Player attempts to access the contact information of a 
  % team captain not in their team.

  % Output/Result: Captain's contact information is not displayed.

  % How test will be performed: Various player level accounts from various teams 
  % will attempt to access the contact information team captains who are not 
  % in their respective teams. Players should not be able to access the information.

  % \item{test-AS3-1\\}

  % Type: Non-Functional, Dynamic, Manual

  % Initial State: The system is logged in to a captain level account.

  % Input/Condition: The captain will attempt to access the contact information
  % of different players on their team and other captains in the league.

  % Output/Result: The system will display their contact information.

  % How test will be performed: The user will attempt to access the contact
  % information of multiple players on their team and other captains in the
  % league. They should be able to access it.

  % \item{test-AS3-2\\}

  % Type: Non-Functional, Dynamic, Manual

  % Initial State: The system is logged in to a captain level account.

  % Input/Condition: The captain will attempt to access the contact information
  % of different players that are not on their team.

  % Output/Result: The system will not display their contact information.

  % How test will be performed: The user will attempt to access contact
  % information of multiple players who are not on their team. They should not
  % be able to access it.

  % \item{test-AS4\\}

  % Type: Non-Functional, Dynamic, Manual

  % Initial State: The system is logged in to a commissioner level account.

  % Input/Condition: The commissioner will attempt to access the contact
  % information of many different players and captains on different teams.

  % Output/Result: The system will display their contact information.

  % How test will be performed: The user will attempt to access contact
  % information of multiple players and captains on different teams. They should
  % be able to access it.

  % \item{test-AS5\\}

  % Type: Non-Functional, Dynamic, Manual

  % Initial State: The solution is opened on a user's web browser.

  % Input/Condition: A user's account is provided permissions that are different from the
  % the account's current permissions.

  % Output/Result: The system provides a warning to the user changing an account's
  % permissions before the permissions are successfully updated.

  % How test will be performed: A tester will be provided 2 accounts. Account 1 will have its
  % permissions changed and account 2 will have the ability to change permissions. They will then
  % use account 2 to update account 1's permissions and observe the warning given by the system
  % or if any errors had occurred. The tester will then be asked to log in to account 1 and try
  % out new functionalities that should have been provided with the permission update or if any
  % errors had occurred.

  \item{test-AS6-1\\}
  
  Description: Users can only access an account with correctly
  inputted login information that matches the database
  stored login information.
  
  Type: Non-Functional, Dynamic, Manual

  Initial State: The system is not logged in to an account.

  Input: User navigates to the login section and enters valid login data for
  an account that has already been made.

  Output: The system logs in to the valid account.

  Test Case Derivation: If valid account information is given when logging in,
  the system should log in as that account.

  How test will be performed: Multiple accounts will be logged into each with
  different valid account information and permission levels.

  \item{test-AS6-2\\}
  
  Description: Users cannot access an account with incorrectly
  inputted login information that does not match the database
  stored login information.

  Type: Non-Functional, Dynamic, Manual

  Initial State: The system is not logged in to an account.

  Input: User navigates to the login section and enters invalid login data
  for an account login.

  Output: The system warns the user the data used is invalid.

  Test Case Derivation: If invalid account information is given when logging
  in the system should not log in to any account and warn the user the data
  isn't valid.

  How test will be performed: Multiple invalid logins will be attempted each
  with different invalid account information.

  % \item{test-AS7\\}
  
  % Type: Non-Functional, Dynamic, Manual
            
  % Initial State: The system is logged in as a captain of a team. 
            
  % Input/Condition: Captain user attempts to join a team that they are not a part
  % of.
            
  % Output/Result: The captain user will be able to join the team.
            
  % How test will be performed: A captain of a team will attempt to join a
  % team that is not their own. It should be checked if the captain user is
  % successfully added to the team list.

  % \item{test-AS8\\}

  % Type: Non-Functional, Dynamic, Manual

  % Initial State: The system is logged in as a captain of a team and is also a
  % player on another team.

  % Input/Condition: The captain attempts to change team information and submit
  % scores.

  % Output/Result: The system should not make any changes.

  % How test will be performed: A captain who is a player on a team they are not
  % in charge of will try to change team information and submit scores for the
  % team they are not in charge of. If they cannot do these actions, the test
  % succeeds.

  \item{test-IG1-1\\}
  
  Description: Website must not create conflicts when scheduling
  games.

  Type: Non-Functional, Dynamic, Manual

  Initial State: The system has received scheduling data.

  Input/Condition: The system is instructed to create the schedule with non-conflicting
  scheduling data.

  Output: The system creates a schedule without conflicting scheduling data.

  How test will be performed: Sample scheduling data will be provided to the system. The test
  succeeds if a schedule is created without conflicting scheduling data.

  \item{test-IG1-2\\}
  
  Description: Website must not create conflicts when rescheduling
  games.

  Type: Non-Functional, Dynamic, Manual

  Initial State: The system has scheduling data and awaits any reschedule requests.

  Input/Condition: The system is sent a reschedule request.

  Output: The system accepts or denies a reschedule request and either changes the scheduling
  data or remains the same with no conflicting scheduling data.

  How test will be performed: Sample scheduling data and a sample reschedule request
  will be provided to the system. The test succeeds if a reschedule request is either accepted
  or denied and the schedule is updated or remains the same with no conflicting scheduling
  data.

  % \item{test-IG2\\}

  % Type: Non-Functional, Dynamic, Manual

  % Initial State: The system is logged in to a commissioner level account.

  % Input: A user attempts to delete their account.

  % Output: A warning that you cannot delete the only commissioner level
  % account.

  % How test will be performed: A commissioner level account will attempt to
  % delete their account. The test succeeds if they get a warning that their
  % account cannot be deleted since it is the only commissioner level account
  % and the account is not deleted.

  % \item{test-IG3\\}

  % Type: Non-Functional, Dynamic, Manual

  % Initial State: The system is logged in to a captain level account.

  % Input: A user attempts to contest a match score.

  % Output: An option is visible to the user to contest a recorded match score.

  % How test will be performed: A captain level account will attempt to
  % contest a match score. The test succeeds if they are able to view an option to contest
  % a recorded match score since they are a captain level account and they can view and contest
  % a recorded match score.

  % \item{test-IG4-1\\}

  % Type: Non-Functional, Dynamic, Manual

  % Initial State: The system is logged in to a captain level account.

  % Input/Condition: A user awaits a notification from an alert.

  % Output: A notification is received by the user about an alert.

  % How test will be performed: A captain level account will be sent a sample alert. The test
  % succeeds if they receive a notification about the alert and they should observe if any
  % errors or failures occur.

  % \item{test-IG4-2\\}

  % Type: Non-Functional, Dynamic, Manual

  % Initial State: The system is logged in to a captain level account.

  % Input/Condition: A user awaits a notification from a reschedule request.

  % Output: A notification is received by the user about a reschedule request.

  % How test will be performed: A captain level account will be sent a sample reschedule request.
  % The test succeeds if they receive a notification about the reschedule request and they
  % should observe if any errors or failures occur.

  \item{test-PV1\\}
  
  Description: The product shall not reveal contact information
  to any parties without the necessary access.

  Type: Non-Functional, Dynamic, Manual

  Initial State: The system is not logged in.

  Input/Condition: A user is not logged into an account.

  Output: No contact information is displayed to the user.

  How test will be performed: A user will not be logged into an account and will
  observe if any contact information can be seen or accessed. If no contact information
  is revealed to the user, the test succeeds.

  % \item{test-PV2\\}

  % Type: Non-Functional, Dynamic, Manual

  % Initial State: The system is not logged in.

  % Input: A user creates an account.

  % Output: A warning to keep your password secure, and a new account is made.

  % How test will be performed: A new account will be created. If the user 
  % receives a warning to keep their password secure, the test succeeds.

\end{enumerate}

% \subsubsection{Cultural Requirements}

% \begin{enumerate}

  % \item{test-CL1\\}

  % Type: Non-Functional, Dynamic, Manual

  % Initial State: A user is creating an account for themselves on the solution.

  % Input/Condition: A user selects a gender option when creating their account.

  % Output/Result: A user is able to select a gender option that does not specify their
  % gender when creating their account.

  % How test will be performed: A tester will be provided the solution and will be asked
  % to create a new account. They will then be asked to select the gender option that allows
  % them to not specify their gender and observe if their information is displayed correctly
  % when they are finished creating their account or if any errors had occurred.

% \end{enumerate}

\subsubsection{Migration to the New Product}

\begin{enumerate}

  \item{test-DMT1\\}
  
  No test needed for this requirement.

\end{enumerate}

\subsubsection{User Documentation and Training}

\begin{enumerate}

  \item{test-TR1\\}

  No test needed for this requirement.

\end{enumerate}

\subsection{Traceability Between Test Cases and Requirements}

The name of our test cases are each stamped with the requirement number which
they address. Each requirement in our SRS \cite{SRS} has an associated test to
ensure full coverage. If a requirement doesn't require a test it will be
noted in section 4 above.

\section{Unit Test Description}

% \wss{This section should not be filled in until after the MIS (detailed design
%   document) has been completed.}

% \wss{Reference your MIS (detailed design document) and explain your overall
% philosophy for test case selection.}  

% \wss{To save space and time, it may be an option to provide less detail in this section.  
% For the unit tests, you can potentially layout your testing strategy here.  That is, you 
% can explain how tests will be selected for each module.  For instance, your test building 
% approach could be test cases for each access program, including one test for normal behaviour 
% and as many tests as needed for edge cases.  Rather than create the details of the input 
% and output here, you could point to the unit testing code.  For this to work, you code 
% needs to be well-documented, with meaningful names for all of the tests.}

\subsection{Unit Testing Scope}

% \wss{What modules are outside of the scope.  If there are modules that are
%   developed by someone else, then you would say here if you aren't planning on
%   verifying them.  There may also be modules that are part of your software, but
%   have a lower priority for verification than others.  If this is the case,
%   explain your rationale for the ranking of module importance.}

The modules that will primary use unit testing are the Reschedule Module,
Database Module and Season Scheduler Module. This is beacuse the other modules
(Player Module, Team Module, Commissioner Module, Account Module) have a focus
on user interface rather than functions and mostly use functions defined in
the Database Module. Their user interface interactions are covered in tests in
section 4 and useability tests.

\subsection{Tests for Functional Requirements}

% \wss{Most of the verification will be through automated unit testing.  If
%   appropriate specific modules can be verified by a non-testing based
%   technique.  That can also be documented in this section.}

This section will cover testing all data accessing and manipulation functions
in our system using unit tests. The modules covering these functions are the
Reschedule Module, Database Module and Season Scheduler Module.

\subsubsection{Reschedule Module}

% \wss{Include a blurb here to explain why the subsections below cover the module.
%   References to the MIS would be good.  You will want tests from a black box
%   perspective and from a white box perspective.  Explain to the reader how the
%   tests were selected.}

\begin{enumerate}

\item{test-URS1, Reschedule date variation.\\}

Description: Rescheduling of a game should update the database to remove
the original date of the game and replace the same game with a new date.

Type: Automatic

Initial State: Schedule generated.

Input: Variations of an original date of a game and a new date for the game.

Output: The database is updated to remove the game from the original date and
add the same game to the new date.

Test Case Derivation: Rescheduler must update the database to reflect the new
date of the game.

How test will be performed: Unit test in test\_rescheduler.py found at
src/backend/app/functions.

\end{enumerate}

\subsubsection{Database Module}

% \wss{Include a blurb here to explain why the subsections below cover the module.
%   References to the MIS would be good.  You will want tests from a black box
%   perspective and from a white box perspective.  Explain to the reader how the
%   tests were selected.}

\begin{enumerate}

\item{test\_get\_schedule\_games\\}
  Located in: \href{https://github.com/Nicholas-Fabugais-Inaba/Sandlot/blob/pytest-workflow/test/test\_schedule.py}{test\_schedule.py}
  
  Tests if the backend code properly retrieves all games in the accepted schedule. The 
  database is a testing database separate from the one used in the actual system. 
  Additionally, the game data being retrieved is from a set of placeholder testing data
  such that the list of games retrieved is consistent and can be tested against. (This 
  information about the test database and placeholder data is true for all database module
  tests and will not be repeated for each test, but will apply)

  \item{test\_team\_games\\}
  Located in: \href{https://github.com/Nicholas-Fabugais-Inaba/Sandlot/blob/pytest-workflow/test/test\_schedule.py}{test\_schedule.py}
  
  Tests if the backend code properly retrieves all games played by a certain team based on 
  the id of the team passed to the function.

  \item{test\_create\_RR\\}
  Located in: \href{https://github.com/Nicholas-Fabugais-Inaba/Sandlot/blob/pytest-workflow/test/test\_schedule.py}{test\_schedule.py}
  
  Tests if the backend code properly creates and stores a reschedule request between 2 teams
  passed to the function. The reschedule request created only contains 1 reschedule option.

  \item{test\_get\_team\_RRs\\}
  Located in: \href{https://github.com/Nicholas-Fabugais-Inaba/Sandlot/blob/pytest-workflow/test/test\_schedule.py}{test\_schedule.py}
  
  Tests if the backend code properly retrieves all reschedule requests sent to a team 
  based on the id of the team passed to the function.

  \item{test\_delete\_reschedule\_request\\}
  Located in: \href{https://github.com/Nicholas-Fabugais-Inaba/Sandlot/blob/pytest-workflow/test/test\_schedule.py}{test\_schedule.py}
  
  Tests if the backend code properly deletes a reschedule request when passed its 
  corresponding id in the database.

  \item{test\_get\_standings\_data\\}
  Located in: \href{https://github.com/Nicholas-Fabugais-Inaba/Sandlot/blob/pytest-workflow/test/test_standings.py}{test\_standings.py}
  
  Tests if the backend code properly retrieves the score data for all completed games
  in the schedule and calculates the standings for all teams based on the scores.
  
  \item{test\_get\_player\_data\\}
  Located in: \href{https://github.com/Nicholas-Fabugais-Inaba/Sandlot/blob/pytest-workflow/test/test_teams.py}{test\_teams.py}
  
  Tests if the backend code properly retrieves the list of all players that belong to a 
  certain team and their corresponding information based on the id of the teanm passed to
  the function.

  \item{test\_get\_team\_data\\}
  Located in: \href{https://github.com/Nicholas-Fabugais-Inaba/Sandlot/blob/pytest-workflow/test/test_teams.py}{test\_teams.py}
  
  Tests if the backend code properly retrieves the list of all teams in the database and 
  their divisions.

  \item{test\_create\_player\_account\\}
  Located in: \href{https://github.com/Nicholas-Fabugais-Inaba/Sandlot/blob/pytest-workflow/test/test_users.py}{test\_users.py}
  
  Tests if the backend code properly creates and stores player account in the database 
  when passed a name, email, and password.

  \item{test\_create\_team\_account\\}
  Located in: \href{https://github.com/Nicholas-Fabugais-Inaba/Sandlot/blob/pytest-workflow/test/test_users.py}{test\_users.py}
  
  Tests if the backend code properly creates and stores team account in the database 
  when passed a team name, username, password, and preference data.

  \item{test\_get\_player\_account\\}
  Located in: \href{https://github.com/Nicholas-Fabugais-Inaba/Sandlot/blob/pytest-workflow/test/test_users.py}{test\_users.py}
  
  Tests if the backend code properly retrieves a player account from the database 
  based on the email of the player passed to the function.

  \item{test\_get\_team\_account\\}
  Located in: \href{https://github.com/Nicholas-Fabugais-Inaba/Sandlot/blob/pytest-workflow/test/test_users.py}{test\_users.py}
  
  Tests if the backend code properly retrieves a team account from the database 
  based on the username of the team passed to the function.

  item{test\_delete\_player\\}
  Located in: \href{https://github.com/Nicholas-Fabugais-Inaba/Sandlot/blob/pytest-workflow/test/test_users.py}{test\_users.py}
  
  Tests if the backend code properly deletes a player account request when passed its 
  corresponding id in the database.

  item{test\_delete\_team\\}
  Located in: \href{https://github.com/Nicholas-Fabugais-Inaba/Sandlot/blob/pytest-workflow/test/test_users.py}{test\_users.py}
  
  Tests if the backend code properly deletes a team account request when passed its 
  corresponding id in the database.


\end{enumerate}

\subsubsection{Season Scheduler Module}

To cover testing the Season Scheduler Module, the following tests will be
perfromed. They will test each input of the scheduler and validate that
whenever possible, a valid schedule is generated.

\begin{enumerate}

\item{test-USS1, Start and end dates.\\}

Description: Multiple variations of season start and end dates, number of games
per team, teams, and division data will be used to generate a valid schedule
for each given variation, ensuring all games are within the start and end date.

Type: Automatic

Initial State: Schedule not yet generated.

Input: Variations of season start and end dates, standard number of games per
team, team and division data.

Output: A valid schedule is generated, all games within start and end date
range.

Test Case Derivation: Scheduler must generate a valid schedule based on
inputs.

How test will be performed: Unit test in test\_scheduler.py found at
src/backend/app/functions.

\item{test-USS2, Number of games played per team.\\}

Description: Multiple variations of season start and end dates, number of games
per team, teams, and division data will be used to generate a valid schedule
for each given variation, ensuring all teams play the same number of games
plus or minus one game.

Type: Automatic

Initial State: Schedule not yet generated.

Input: Variations of number of games played per team, standard dates, team and
division data.

Output: A valid schedule is generated, all teams play the same number of games
in the season plus or minus one game.

Test Case Derivation: Scheduler must generate a valid schedule based on
inputs.

How test will be performed: Unit test in test\_scheduler.py found at
src/backend/app/functions.

\item{test-USS3, Team data variations.\\}

Description: Multiple variations of season start and end dates, number of games
per team, teams, and division data will be used to generate a valid schedule
for each given variation, ensuring all teams do not play on their offday and
only play teams in their division.

Type: Automatic

Initial State: Schedule not yet generated.

Input: Variations of team's data and division data, standard dates and number
of games played by each team.

Output: A valid schedule is generated, all teams do not play on their offday
and only play teams in their division.

Test Case Derivation: Scheduler must generate a valid schedule based on
inputs.

How test will be performed: Unit test in test\_scheduler.py found at
src/backend/app/functions.

\end{enumerate}

\subsection{Tests for Nonfunctional Requirements}

% \wss{If there is a module that needs to be independently assessed for
%   performance, those test cases can go here.  In some projects, planning for
%   nonfunctional tests of units will not be that relevant.}

% \wss{These tests may involve collecting performance data from previously
%   mentioned functional tests.}

Some modules require tasks to be done in a reasonable amount of time. These
tests will define what a reasonable amount of time is and test if the modules
meet these requirements.

\subsubsection{Season Scheduler Module}

To cover testing the Season Scheduler Module, the following tests will be
perfromed. They will test each input of the scheduler and validate that
whenever possible, a valid schedule is generated.

\begin{enumerate}

\item{test-USS4, Time to generate schedule.\\}

Description: Multiple variations of season start and end dates, number of games
per team, teams, and division data will be used to generate a valid schedule
for each given variation, ensuring a valid schedule is generated within 10 minutes.

Type: Automatic

Initial State: Schedule not yet generated.

Input: Variations of season start and end dates, number of games per
team, team and division data.

Output: A valid schedule is generated within 10 minutes.

Test Case Derivation: Scheduler must generate a valid schedule within the
supervisors time constraints.

How test will be performed: Unit test in test\_scheduler.py found at
src/backend/app/functions.

\end{enumerate}

\subsection{Traceability Between Test Cases and Modules}

% \wss{Provide evidence that all of the modules have been considered.}

Each module that requires unit testing has been named in the above sections.
The section name matches the module name in the Module Decomposition table in
the MG and MIS documents \cite{MG} \cite{MIS}.

\bibliographystyle{plainnat}

\bibliography{../../refs/References}

\newpage

\section{Appendix}

\subsection{Traceability Matrix}

\begin{table}
  \centering
  \begin{tabular}{p{0.2\textwidth} p{0.6\textwidth}}
  \toprule
  \textbf{Requirement} & \textbf{Test}\\
  \midrule
  FR-1 & test-FR1-1, test-FR1-2\\
  FR-3 & test-FR3-1, test-FR3-2\\
  FR-4 & test-FR4\\
  FR-5 & test-FR5-1, test-FR5-2\\
  FR-8 & test-FR8\\
  FR-10 & test-FR10\\
  FR-11 & test-FR11-1, test-FR11-2\\
  FR-12 & test-FR12\\
  FR-15 & test-FR15\\
  FR-16 & test-FR16-1, test-FR16-2\\
  FR-18 & test-FR18\\
  FR-20 & test-FR20\\
  FR-21 & test-FR21\\
  AP-3 & test-AP3\\
  AP-4 & test-AP4\\
  STY-1 & test-STY1\\
  EU-1 & test-EU1\\
  EU-2 & test-EU2\\
  LR-3 & test-LR3\\
  UP-1 & test-UP1\\
  RR-1 & test-RR1\\
  AS-1 & test-AS1-1, test-AS1-2\\
  AS-6 & test-AS6-1, test-AS6-2\\
  IG-1 & test-IG1-1, test-IG1-2\\
  PV-1 & test-PV1\\
  URS-1 & test-URS1\\
  USS-1 & test-USS1\\
  USS-2 & test-USS2\\
  USS-3 & test-USS3\\
  USS-4 & test-USS4\\

  \bottomrule
  \end{tabular}
  \caption{Trace Between Functional Requirements and Tests}
  \label{TblRT}
  \end{table}

\subsection{Symbolic Parameters}

The definition of the test cases will call for SYMBOLIC\_CONSTANTS.
Their values are defined in this section for easy maintenance.

\subsection{Usability Survey Questions}

A Microsoft Form will be provided to testers of the system to validate specific
system tests outlined in the VnV plan. Answers for each question will range from either a
rating between 1 (Strongly Disagree) to 5 (Strongly Agree), 1 (Very Difficult) to 5 (Very Easy), or
1 (No, the system was difficult to use due to accessibility issues) to 3 (Yes, the system was accessible).
A satisfactory result for each question should be a 3 (Neutral/Yes, the system was accessible) or above.
Additionally, the team should address any issues the user has with the system, which will be received
from optional questions located below each multiple choice question of the survey.
\\\\Questions in the Microsoft Form can be accessed at the following link:\\

\href{https://forms.office.com/Pages/ResponsePage.aspx?id=B2M3RCm0rUKMJSjNSW9HcodvkeIlB8lOjrmyIWuVT7dUQ0hBNFRVTjFHWVhITDIzSklZRDRYTVZRMi4u}{Sandlot Usability Survey}

\newpage{}
\section*{Appendix --- Reflection}

The information in this section will be used to evaluate the team members on the
graduate attribute of Lifelong Learning.

\input{../Reflection.tex}

\begin{enumerate}
  \item What went well while writing this deliverable? 
  \item What pain points did you experience during this deliverable, and how
    did you resolve them?
  \item What knowledge and skills will the team collectively need to acquire to
  successfully complete the verification and validation of your project?
  Examples of possible knowledge and skills include dynamic testing knowledge,
  static testing knowledge, specific tool usage, Valgrind etc.  You should look to
  identify at least one item for each team member.
  \item For each of the knowledge areas and skills identified in the previous
  question, what are at least two approaches to acquiring the knowledge or
  mastering the skill?  Of the identified approaches, which will each team
  member pursue, and why did they make this choice?
\end{enumerate}

\subsection*{Team Reflection}

  \begin{enumerate}
    \item What knowledge and skills will the team collectively need to acquire to
    successfully complete the verification and validation of your project?
    Examples of possible knowledge and skills include dynamic testing knowledge,
    static testing knowledge, specific tool usage, Valgrind etc.  You should look to
    identify at least one item for each team member.\\

    Our team will need to collectively acquire a great deal of knowledge and skills 
    to successfully complete the verification and validation of our project. In 
    particular, knowledge on how to implement continuous integration using Github 
    actions will be key to automating unit tests to run whenever changes are made 
    to the code. Furthermore, the skills and knowledge needed to implement effective 
    unit tests using pytest is the core of the verification and validation of the 
    project. Additionally, understanding how to implement and properly analyze code 
    coverage using tools like Coverage.py for Python and Jest for React will be key 
    to ensuring our unit tests are robust and properly test all states of our code. 
    Finally, learning how to properly setup and use linters like flake8 for Python 
    and ESLint for React will be extremely helpful for catching bugs before we test, 
    hopefully reducing the amount of time we spend debugging. Casra will tackle 
    Github actions, Jung-woo will learn about pytest, Nicholas will look into code 
    coverage tools, and Alex will handle linters.\\

    \item For each of the knowledge areas and skills identified in the previous
    question, what are at least two approaches to acquiring the knowledge or
    mastering the skill?  Of the identified approaches, which will each team
    member pursue, and why did they make this choice?\\

    Regarding the knowledge area of CI and Github actions, 2 possible approaches 
    to learning the skills required would be through official documentation from 
    Github or online video tutorials. Casra will tackle learning about CI and 
    Github actions from the official Github documentation because he believes that 
    learning about a tool from the people that made it will include more thorough 
    information than the learning material of a 3rd party (ie. a video tutorial). 
    In terms of learning the knowledge and skills needed to create unit tests via 
    pytest, 2 possible approaches to gaining the knowledge and skills required would 
    be through the official pytest documentation or more general articles discussing 
    the basics and best practices for unit testing. Jung-woo will learn about unit 
    testing through the more general articles because he believes that starting with 
    the more basic knowledge and working his way up to more framework specific unit 
    testing knowledge will allow him to have a better and deeper understanding of 
    how to make proper unit tests. Regarding the knowledge area of code coverage and 
    related tools like Coverage.py and Jest, 2 possible approaches to learning the 
    skills required would be through online video tutorials, or testing out the 
    tools yourself. Nicholas will learn about code coverage and the related tools 
    by testing out the tools himself on sample code. This is because he believes 
    that he will learn more quickly by trying out the tools himself rather than 
    reading about how they work. Finally, in terms of learning the knowledge and 
    skills needed to setup and work with linters like flake8 and ESLint, 2 possible 
    approaches to gaining the knowledge and skills required would be through 
    official tool documentation or referencing previous projects which used the 
    tools. Alex will learn about linters and the related tools by referencing a 
    previous project in which he used flake8. This is because he believes that by 
    revisiting an old project he’s worked on in the past it will help jog his memory 
    on how he used linters previously.
\end{enumerate}

\subsection*{Casra Ghazanfari -- Reflection}

\begin{enumerate}
  \item What went well while writing this deliverable?
  
  Section 3.6: Automated Testing and Verification Tools went well while writing 
  this deliverable because we had already discussed all these topics in the 
  development plan’s expected technology section. This allowed us to simply 
  reference the development plan document and provide a high level overview of the 
  information we previously discussed rather than spending time researching and 
  writing in detail about the topic from scratch for the VnV document.

  \item What pain points did you experience during this deliverable, and how
    did you resolve them?
  
  The main pain point we experienced during this deliverable was having to spend a 
  large chunk of our time on a major revision on our requirements document. We did 
  this revision because we felt our functional and access requirements were still 
  incomplete as we worked through some initial tests. Even though we had already 
  done a major revision on our requirements document previously when working on our 
  hazard analysis document, that revision was very NFR focused. Therefore we felt 
  another revision was necessary to target our FRs and ARs. Ultimately, this process 
  took a good chunk of our time working on the VnV document but was worth it to 
  ensure the robustness of our requirements and their respective tests.
\end{enumerate}

\subsection*{Nicholas Fabugais-Inaba -- Reflection}

\begin{enumerate}
  \item What went well while writing this deliverable?\\\\
  When writing this deliverable, the things that went well were the organization of tasks
  amongst each team member. Particularly with the system tests, each team member was given
  a group of requirements to write system tests for regarding both the functional and
  non-functional requirements located in the SRS. This allowed us to accomplish our own
  individual work, while as a team, we were able to discuss the various plans outlined in the
  Plan section of the VnV document. With the SRS requirements having an adequate amount of
  details to work from, writing the system tests became very easy to trace back to the SRS
  requirements.
  \item What pain points did you experience during this deliverable, and how
  did you resolve them?\\\\
  Some of the pain points from this deliverable were figuring out which tests were not
  covered by the SRS requirements. This was an issue as certain tests we realized we should
  include in our VnV plan were not addressed by the previously created SRS requirements.
  This meant the team had to go back and adjust or add new requirements to the SRS document,
  where we could then add system tests to appropriately address certain functionalities that
  would require sufficient testing by a user of the system.
\end{enumerate}

\subsection*{Jung Woo Lee -- Reflection}

\begin{enumerate}
  \item What went well while writing this deliverable?\\\\
  Writing the test cases helped me see how the product would shape to 
  be, and allowed the team to make crucial changes to the requirements. 
  They helped me verify if requirements were atomic and self-contained 
  or not. It also helped understand the nature of some requirements as 
  I considered the test types: manual vs. automatic and dynamic vs. 
  static.  
  \item What pain points did you experience during this deliverable, and how
  did you resolve them?\\\\
  There trouble coming up with initial states of some tests in terms of 
  what to include or exclude. I found that it was easy to overlook a 
  crucial but obvious detail in this section of the tests. I also had 
  trouble with thinking of inputs for the tests. One issue was trying 
  to avoid as much implementation detail as possible. Another was 
  when tests seemingly had no obvious input. Lastly, considering 
  invalid cases caused some difficulty, as these are not explicitly 
  mentioned by the requirements document, so this was the first time 
  thinking about what should not happen instead of what should happen.
  These were all resolved either by more thorough thought, discussion 
  with team members, or looking at examples to gauge how they could be 
  written.
\end{enumerate}

\subsection*{Alex Verity -- Reflection}

\begin{enumerate}

  \item What went well while writing this deliverable?\\\\
  Writing section 4, especially the functional tests, helped fill in a lot of
  holes in the coverage of the SRS \cite{SRS}. I'm always glad when this
  happens as I believe it leads to a better final SRS document and I hope a
  smoother development workload as we have a better grasp on what our final
  solution will look like.
  \item What pain points did you experience during this deliverable, and how
  did you resolve them?\\\\
  Writing so many tests was definitely a pain point, as we needed to go back
  to our SRS and update our requirements. Particularly the functional
  requirements and access requirements. It is similar to what went well while
  writing this deliverable but finding more requirements is a double edged
  sword. We resolved it by filling out the SRS and fixing requirements that
  needed updating.

\end{enumerate}

\end{document}