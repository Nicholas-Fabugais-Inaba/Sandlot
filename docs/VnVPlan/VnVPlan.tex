\documentclass[12pt, titlepage]{article}

\usepackage{booktabs}
\usepackage{tabularx}
\usepackage{hyperref}
\hypersetup{
    colorlinks,
    citecolor=blue,
    filecolor=black,
    linkcolor=red,
    urlcolor=blue
}
\usepackage[numbers, round]{natbib}

\input{../Comments}
\input{../Common}

\begin{document}

\title{System Verification and Validation Plan for \progname{}} 
\author{\authname}
\date{\today}
	
\maketitle

\pagenumbering{roman}

\section*{Revision History}

\begin{tabularx}{\textwidth}{p{3cm}p{2cm}X}
\toprule {\bf Date} & {\bf Version} & {\bf Notes}\\
\midrule
Oct. 28, 2024 & 1.0 & TA Feedback\\
Nov. 1, 2024 & 1.1 & Rev0\\
\bottomrule
\end{tabularx}

~\\
\wss{The intention of the VnV plan is to increase confidence in the software.
However, this does not mean listing every verification and validation technique
that has ever been devised.  The VnV plan should also be a \textbf{feasible}
plan. Execution of the plan should be possible with the time and team available.
If the full plan cannot be completed during the time available, it can either be
modified to ``fake it'', or a better solution is to add a section describing
what work has been completed and what work is still planned for the future.}

\wss{The VnV plan is typically started after the requirements stage, but before
the design stage.  This means that the sections related to unit testing cannot
initially be completed.  The sections will be filled in after the design stage
is complete.  the final version of the VnV plan should have all sections filled
in.}

\newpage

\tableofcontents

\listoftables
\wss{Remove this section if it isn't needed}

\listoffigures
\wss{Remove this section if it isn't needed}

\newpage

\section{Symbols, Abbreviations, and Acronyms}

\renewcommand{\arraystretch}{1.2}
\begin{tabular}{l l} 
  \toprule		
  \textbf{symbol} & \textbf{description}\\
  \midrule 
  SRS & Software Requirements Specification\\
  GSA & Graduate Students Association\\
  T & Test\\
  \bottomrule
\end{tabular}\\

\wss{symbols, abbreviations, or acronyms --- you can simply reference the SRS
  \citep{SRS} tables, if appropriate}

\wss{Remove this section if it isn't needed}

\newpage

\pagenumbering{arabic}

This document ... \wss{provide an introductory blurb and roadmap of the
  Verification and Validation plan}

\section{General Information}

\subsection{Summary}

\wss{Say what software is being tested.  Give its name and a brief overview of
  its general functions.}

The software being tested is the Sandlot project. Sandlot is a scheduling and
management platform for the McMaster GSA softball league. Users of the system
will include players, captains, commissioners, and other general users that will
not need to make an account to access the system. Sandlot is intended to be an
upgrade to the current platform that is outdated and lacks maintainability. This
project will build off the current platform's existing features such as game
scheduling, viewing of the scoring and standings, and team creation. Sandlot will
also add new features including account creation and commissioner specific
permissions like league-wide alerts.

\subsection{Objectives}

\wss{State what is intended to be accomplished.  The objective will be around
  the qualities that are most important for your project.  You might have
  something like: ``build confidence in the software correctness,''
  ``demonstrate adequate usability.'' etc.  You won't list all of the qualities,
  just those that are most important.}

\wss{You should also list the objectives that are out of scope.  You don't have 
the resources to do everything, so what will you be leaving out.  For instance, 
if you are not going to verify the quality of usability, state this.  It is also 
worthwhile to justify why the objectives are left out.}

\wss{The objectives are important because they highlight that you are aware of 
limitations in your resources for verification and validation.  You can't do everything, 
so what are you going to prioritize?  As an example, if your system depends on an 
external library, you can explicitly state that you will assume that external library 
has already been verified by its implementation team.}

The main objectives intended to be accomplished are to build confidence in the software
correctness, demonstrate the requirements created for this project are correctly
implemented, and demonstrate adequate usability for functionalities of the
system. When demonstrating the correct implementations of requirements for the
project, these requirements are referring to the ones outlined in both the
SRS and Hazard Analysis documents.

Objectives that are out of scope due to the limitations in our resources for
verification and validation include the assumption that the database and the web
server the platform utilizes and the platform runs on, respectively, have already
been verified by its implementation teams.

\subsection{Challenge Level and Extras}

\wss{State the challenge level (advanced, general, basic) for your project.
Your challenge level should exactly match what is included in your problem
statement.  This should be the challenge level agreed on between you and the
course instructor.  You can use a pull request to update your challenge level
(in TeamComposition.csv or Repos.csv) if your plan changes as a result of the
VnV planning exercise.}

\wss{Summarize the extras (if any) that were tackled by this project.  Extras
can include usability testing, code walkthroughs, user documentation, formal
proof, GenderMag personas, Design Thinking, etc.  Extras should have already
been approved by the course instructor as included in your problem statement.
You can use a pull request to update your extras (in TeamComposition.csv or
Repos.csv) if your plan changes as a result of the VnV planning exercise.}

The challenge level of the project is general. The extras that are being used
are user documentation and a code walkthrough. The challenge level and both extras
have been approved by an instructor.

\subsection{Relevant Documentation}

\wss{Reference relevant documentation.  This will definitely include your SRS
  and your other project documents (design documents, like MG, MIS, etc).  You
  can include these even before they are written, since by the time the project
  is done, they will be written.  You can create BibTeX entries for your
  documents and within those entries include a hyperlink to the documents.}

\begin{thebibliography}{9}
  \bibitem{SRS} Software Requirements Specification Document (2024)
  \href{../SRS-Volere/SRS.pdf}{SRS.pdf}
  \bibitem{MG} Module Guide Document (2024)
  \href{../Design/SoftArchitecture/MG.pdf}{MG.pdf}
  \bibitem{MIS} Module Interface Specification Document (2024)
  \href{../Design/SoftDetailedDes/MIS.pdf}{MIS.pdf}
  \\
\end{thebibliography}

\wss{Don't just list the other documents.  You should explain why they are relevant and 
how they relate to your VnV efforts.}\\
The SRS document has requirements for our project that must be verified, it is
the most complete list of functional and non-functional requirements that our
solution needs to achieve.

\section{Plan}

\wss{Introduce this section.  You can provide a roadmap of the sections to
  come.}

\subsection{Verification and Validation Team}

\wss{Your teammates.  Maybe your supervisor.
  You should do more than list names.  You should say what each person's role is
  for the project's verification.  A table is a good way to summarize this information.}

\begin{enumerate}
  \item Casra: Automation Tester
  Will focus on automation testing using GitHub Actions.
  \item Jung Woo: Non-Functional Tester
  Will focus on non-functinoal tests listed in section 4.2.
  \item Alex: Functional Tester
  Will focus on functional tests listed in section 4.1.
  \item Nicholas: Survey Tester
  Will be in charge of performing surveys needed for any non-functional tests.
  \item Dr. Jake Nease: Supervisor
  Will help give feedback on all functionality of the solution and gather
  stakeholders to help test the solution.
\end{enumerate}

\subsection{SRS Verification Plan}

\wss{List any approaches you intend to use for SRS verification.  This may
  include ad hoc feedback from reviewers, like your classmates (like your
  primary reviewer), or you may plan for something more rigorous/systematic.}

\wss{If you have a supervisor for the project, you shouldn't just say they will
read over the SRS.  You should explain your structured approach to the review.
Will you have a meeting?  What will you present?  What questions will you ask?
Will you give them instructions for a task-based inspection?  Will you use your
issue tracker?}

\wss{Maybe create an SRS checklist?}

A meeting with the team and supervisor will be held to verify the SRS document covers all of
the necessary requirements desired for the system. Each requirement outlined in the SRS will
be reviewed by the team and supervisor to ensure the requirement appropriately addresses the
requested functionality.

\subsection{Design Verification Plan}

\wss{Plans for design verification}

\wss{The review will include reviews by your classmates}

\wss{Create a checklists?}

\subsection{Verification and Validation Plan Verification Plan}

\wss{The verification and validation plan is an artifact that should also be
verified.  Techniques for this include review and mutation testing.}

\wss{The review will include reviews by your classmates}

\wss{Create a checklists?}

The verification and validation plan will be reviewed by another capstone
group who will give at least six points of feedback on our VnV document. Our
team will review the feedback and make any changes needed. 

During the writing of the verification and validation plan, any questions or
concerns for the superviso will be recorded and asked during our next
scheduled supervisor meeting.

During testing we will use mutation testing to verify test case coverage, and
any holes found in our coverage will be patched by adding more test cases.

\subsection{Implementation Verification Plan}

\wss{You should at least point to the tests listed in this document and the unit
  testing plan.}

\wss{In this section you would also give any details of any plans for static
  verification of the implementation.  Potential techniques include code
  walkthroughs, code inspection, static analyzers, etc.}

\wss{The final class presentation in CAS 741 could be used as a code
walkthrough.  There is also a possibility of using the final presentation (in
CAS741) for a partial usability survey.}

\subsection{Automated Testing and Verification Tools}

\wss{What tools are you using for automated testing.  Likely a unit testing
  framework and maybe a profiling tool, like ValGrind.  Other possible tools
  include a static analyzer, make, continuous integration tools, test coverage
  tools, etc.  Explain your plans for summarizing code coverage metrics.
  Linters are another important class of tools.  For the programming language
  you select, you should look at the available linters.  There may also be tools
  that verify that coding standards have been respected, like flake9 for
  Python.}

\wss{If you have already done this in the development plan, you can point to
that document.}

\wss{The details of this section will likely evolve as you get closer to the
  implementation.}

\subsection{Software Validation Plan}

\wss{If there is any external data that can be used for validation, you should
  point to it here.  If there are no plans for validation, you should state that
  here.}

\wss{You might want to use review sessions with the stakeholder to check that
the requirements document captures the right requirements.  Maybe task based
inspection?}

\wss{For those capstone teams with an external supervisor, the Rev 0 demo should 
be used as an opportunity to validate the requirements.  You should plan on 
demonstrating your project to your supervisor shortly after the scheduled Rev 0 demo.  
The feedback from your supervisor will be very useful for improving your project.}

\wss{For teams without an external supervisor, user testing can serve the same purpose 
as a Rev 0 demo for the supervisor.}

\wss{This section might reference back to the SRS verification section.}

The system will be provided to the supervisor and external testers, gathered by the supervisor,
to enter given inputs into the system. The given inputs shall produce expected outputs according to
the system tests created from the SRS. For specific system tests, a usability survey will be
provided along with the test that must be answered by the individual conducting the test.

\section{System Tests}

\wss{There should be text between all headings, even if it is just a roadmap of
the contents of the subsections.}

\subsection{Tests for Functional Requirements}

\wss{Subsets of the tests may be in related, so this section is divided into
  different areas.  If there are no identifiable subsets for the tests, this
  level of document structure can be removed.}

\wss{Include a blurb here to explain why the subsections below
  cover the requirements.  References to the SRS would be good here.}

\subsubsection{Area of Testing1}

\wss{It would be nice to have a blurb here to explain why the subsections below
  cover the requirements.  References to the SRS would be good here.  If a section
  covers tests for input constraints, you should reference the data constraints
  table in the SRS.}
		
\paragraph{Title for Test}

\begin{enumerate}

  \item{test-FR2-1\\}

  Control: Manual
            
  Initial State: Captain user is logged in and subject team has not been 
  created in system.
            
  Input: Captain user navigates to team creation and inputs valid team 
  information then submits. 
            
  Output: Submitted team is added to the database.
  
  Test Case Derivation: If valid team information is submitted, team
  should be created and information reflected in the database.
  
  How test will be performed: Captain accounts will create multiple
  valid teams with differing data that covers all input fields.

  \item{test-FR2-2\\}

  Control: Manual
            
  Initial State: Captain user is logged in.
            
  Input: Captain user navigates to team creation and inputs invalid team 
  information then submits. 
            
  Output: No team is added to the database and user is given informative 
  feedback as to why team submission failed.
  
  Test Case Derivation: If invalid team information is submitted, team
  should not be created and user should be told the reason as to why.
  
  How test will be performed: Captain accounts will attempt to create 
  multiple invalid teams with differing data that covers all input fields. 

  \item{test-FR3-1\\}

  Control: Manual
            
  Initial State: The system is not logged in to an account.
            
  Input: User navigates to create an account and enters valid account creation
  data.
            
  Output: The system adds an account to the database and logs in to the new
  account.
  
  Test Case Derivation: If valid account information is given a new account
  should be created.
  
  How test will be performed: Multiple accounts will be added to the system
  with differing valid account data covering all input fields.

  \item{test-FR3-2\\}

  Control: Manual
            
  Initial State: The system is not logged in to an account.
            
  Input: User navigates to create an account and enters invalid account
  creation data.
            
  Output: The system does not create a new account and the user is informed of
  which data is invalid.
  
  Test Case Derivation: If invalid account information is given a new account
  should not be created and the user should be notified of which data is
  invalid.
  
  How test will be performed: Multiple attempts to create accounts with
  differing invalid account data will be made, with invalid data each attempt
  covering different input fields.

\item{test-FR4\\}

Control: Manual
					
Initial State: The system is set up and ready to take in the user's input.
					
Input: Valid account information.
					
Output: User inputted account information has replaced the previously displayed
account information.

Test Case Derivation: The account information inputted by the user has already been
determined to be valid and should not cause the system to run into any errors. The
user inputted account information, although should be correct, is not required by
the system to be correct to change the user's previous account information.
					
How test will be performed: The tester will be provided valid account information
that they will input into the system and observe if the system successfully
accepts the inputted account information. The tester will also observe if the previously
existing account information has been changed to the information that had been entered
at the start of the test. At any point during the test, the tester will also observe if
any errors occur within the system.

\item{test-FR6-1\\}

Control: 

Initial State: The system is logged into a commissioner level account.
          
Input: Valid alert data and chosen target user(s) entered.
          
Output: Alerts are stored in the system. Alerts are viewable by target users.

Test Case Derivation: Alerts sent to target user should be able to be 
seen by those target users.

How test will be performed: Different commissioner level accounts input
various valid alert data and submit. Target user accounts should be checked
if the alerts are received.

\item{test-FR6-2\\}

Control: 

Initial State: The system is logged into a commissioner level account.
          
Input: Invalid alert data or invalid chosen target user(s) entered.
          
Output: User is given informative feedback as to why alert failed to submit.

Test Case Derivation: Invalid alerts should not be sent and user should
be told as to why they could not be sent.

How test will be performed: Commissioner level account inputs various invalid
alert data covering all input fields.

\item{test-FR7\\}

Control: Automatic

Initial State: The system is logged into a commissioner level account.
          
Input: New team data to replace a current team's data.
          
Output: If data is valid, system should replace the old team data with the new
team data. If the data is invalid, the system should not change the old team
data and inform the commissioner the data is invalid.

Test Case Derivation: A commissioner level account should be able to change
any team's data, including player list and scores.

How test will be performed: A set of valid and invalid test cases will be
submitted to the commissioner's replace team data feature, with the expected
results compared to the results given by the program.

\item{test-FR8\\}

Control: Manual
					
Initial State: The system has provided the option for captains to enter their team
availability data and is ready to take in the user's input.
					
Input: Non-conflicting team availability data.
					
Output: Captain inputted team availability data is stored in the system.

Test Case Derivation: The team availability data inputted by the captain has been
determined to not conflict with other availability data already stored in the system.
					
How test will be performed: The tester will select the option to enter in team 
availability data and be provided non-conflicting team availability
data that they will input into the system. They will then observe if the system successfully
accepts the inputted team availability data and if any errors occur within the system. 

\item{test-FR10\\}

Control: Manual

Initial State: There are two captains whose teams are scheduled to play a
game in the future.
          
Input: One captain submits a reschedule request.

Output: The other captain receives a reschedule request.

Test Case Derivation: A captain should receive a reschedule request if another
captain submits a request on a game both captains will be playing.

How test will be performed: The tester will check if a reschedule request is
successfully sent when a captain requests a reschedule. This will be checked
for at least 3 different dates and times.

\item{test-FR11\\}

Control: Manual

Initial State: A reschedule request has been sent to a captain.
          
Input: A captain accepts or denies a reschedule request.

Output: If accepted, the game is rescheduled to that time. If denied, the
captain who made the request is notified of the denial.

Test Case Derivation: A captain should be able to accept or deny a reschedule
request.

How test will be performed: The tester will accept or deny multiple reschedule
requests on different dates and times and verify the correct output is made by
the system.

\item{test-FR12\\}

Control: Manual
					
Initial State: The system has received a captain's reschedule request and is ready to
notify them about the outcome of their request.
					
Input: Captain reschedule request.
					
Output: A notification is sent by the system to the captain about the status of their
reschedule request.

Test Case Derivation: The notification about the status of a captain's reschedule
request is immediately sent to the captain once the status is confirmed to be either
accepted or denied by the system.
					
How test will be performed: The tester will be provided both a valid and invalid captain
reschedule request to submit into the system and wait to observe the notification sent
by the system of the status for the reschedule request. They will then observe if the
system has both accepted and denied the submitted reschedule requests and if any errors
occur within the system. 

\item{test-FR15\\}

Control: Manual

Initial State: The system will have a game in the schedule that will be played
in the future. The system is logged in as a captain.
          
Input: A score submitted for a game.

Output: No output.

Test Case Derivation: A score should not be entered by a captain until a game
is played.

How test will be performed: A captain level account will attempt to submit a
score for a game that has not been played yet. It should not modify the
standings.

\item{test-FR16-1\\}

Control: Manual
					
Initial State: A captain has created a team.
					
Input: Interaction with create a team functionality.
					
Output: A prompt displayed by the system to the captain that they have already created
a team.

Test Case Derivation: Because the system has determined the captain has already created a
team, the system should not allow the captain to create more than one team.
					
How test will be performed: The tester will be provided a captain account that has already
created a team. They will then attempt to create a team in the system and observe if the
system prevents them from creating another team or if any errors occur within the system.

\item{test-FR16-2\\}

Control: Manual
					
Initial State: A player has joined a team.
					
Input: Interaction with join a team functionality.
					
Output: A prompt displayed by the system to the player that they have already joined
a team.

Test Case Derivation: Because the system has determined the player has already joined a
team, the system should not allow the player to join more than one team.
					
How test will be performed: The tester will be provided a player account that has already
joined a team. They will then attempt to join a team in the system and observe if the
system prevents them from joining another team or if any errors occur within the system.

\item{test-id2\\}

Control: Manual versus Automatic
					
Initial State: 
					
Input: 
					
Output: \wss{The expected result for the given inputs.  Output is not how you
are going to return the results of the test.  The output is the expected
result.}

Test Case Derivation: \wss{Justify the expected value given in the Output field}

How test will be performed: 

\end{enumerate}

\subsubsection{Area of Testing2}

...

\subsection{Tests for Nonfunctional Requirements}

\wss{The nonfunctional requirements for accuracy will likely just reference the
  appropriate functional tests from above.  The test cases should mention
  reporting the relative error for these tests.  Not all projects will
  necessarily have nonfunctional requirements related to accuracy.}

\wss{For some nonfunctional tests, you won't be setting a target threshold for
passing the test, but rather describing the experiment you will do to measure
the quality for different inputs.  For instance, you could measure speed versus
the problem size.  The output of the test isn't pass/fail, but rather a summary
table or graph.}

\wss{Tests related to usability could include conducting a usability test and
  survey.  The survey will be in the Appendix.}

\wss{Static tests, review, inspections, and walkthroughs, will not follow the
format for the tests given below.}

\wss{If you introduce static tests in your plan, you need to provide details.
How will they be done?  In cases like code (or document) walkthroughs, who will
be involved? Be specific.}

\subsubsection{Look and Feel Requirements}
		
\paragraph{Title for Test}

\begin{enumerate}

  \item{test-AP2\\}

  Type: Non-Functional, Dynamic, Manual

  Initial State: The system is logged in.

  Input: Inputs associated with each different type of input field in the
  solution.

  Output: Feedback from each input field.

  How test will be performed: Each possible type of input will be used by the
  tester and they will monitor the result for feedback. Each input field will
  be tested with valid and invalid data if applicable, and invalid data should
  receive feedback that lets the user know the input was invalid.

\item{test-AP3\\}

Type: Non-Functional, Dynamic, Manual
					
Initial State: The solution is opened on a user's web browser.
					
Input/Condition: The user will be asked if the images made for/by Sandlot are viewed at
a high quality containing no pixelations or blurring at their displayed size.
					
Output/Result: The supervisor will state that Sandlot's images are displayed at a high
quality.
					
How test will be performed: The supervisor will be provided the solution and a set of
sample inputs for Sandlot. They will then enter in the sample inputs and observe the
generated outputs from the system.

\wss{Remember to add a usability survey for the supervisor located in the Appendix.}
					
\item{test-id2\\}

Type: Functional, Dynamic, Manual, Static etc.
					
Initial State: 
					
Input: 
					
Output: 
					
How test will be performed: 

\end{enumerate}

\subsubsection{Usability and Humanity Requirements}

\begin{enumerate}

  \item{test-EU1\\}

  Type: Non-Functional, Dynamic, Manual

  Initial State: The system is not logged in.

  Input: Navigation inputs leading to the full season schedule.

  Output: The season schedule displayed on the screen.

  How test will be performed: At least 5 testers unfamiliar to the system will
  attempt to navigate to the season schedule. Their number of clicks used and
  time taken to get to the season schedule will be recorded. The test passes
  if on average testers take less than 2 clicks and less than one minutes to
  find the schedule.

  \item{test-EU2\\}
  
  Type: Non-Functional, Dynamic, Manual
            
  Initial State: User is located on the system's login page and the system is ready
  for the user's inputs.
            
  Input/Condition: Misinputted login information.
            
  Output/Result: The system will provide a warning to the user for login information that
  does not exist or does not match any database stored login information.
            
  How test will be performed: The tester will be provided login information that does not
  currently exist in the database and they will input the provided information into the
  system. The tester will observe the output or any errors that may occur in the system.

  \item{test-LR2\\}

  Type: Non-Functional, Dynamic, Manual

  Initial State: The system is logged in to a captain level account.

  Input: Inputs for all features available to captain level accounts.

  Output: The results of using each captain level account feature.

  How test will be performed: A tester will be given a list of tasks to
  complete including creating a team, modifying team data and submitting a
  score. The tester will also be given the user manual. If the user can
  complete all tasks in less than one hour the test is successful.
            
  \item{test-LR3\\}
  
  Type: Non-Functional, Dynamic, Manual
            
  Initial State: The solution is opened on a user's web browser.
            
  Input/Condition: A new user will be asked to navigate to the season schedule on their first
  time interacting with the solution.
            
  Output/Result: A new user is able to successfully navigate to the season schedule on
  their first time interacting with the solution.
            
  How test will be performed: A new user of the system will be provided the solution and
  will be asked to navigate and view the season schedule.

  \item{test-AC2\\}

  Type: Non-Functional, Dynamic, Manual

  Initial State: The system is logged in to a commissioner level accuont.

  Input: Navigation to each page of the system.

  Output: The colours on screen of each page of the system.

  How test will be performed: Each page of the system will be viewed, and any
  adjacent colours will be confirmed to have a ratio of at least 4.5:1.
  Particular attention will be given to input fields and critical information
  such as schedules.

  \item{test-AC3\\}
  
  Type: Non-Functional, Dynamic, Manual
            
  Initial State: The solution is opened on a user's web browser and a commissioner's web
  browser.
            
  Input/Condition: A user will be sent an alert created by a commissioner and will be asked
  if the alert is visible and readable.
            
  Output/Result: A user is able to successfully see the alert sent by a commissioner and
  the alert is readable and clear enough for the user to understand.
            
  How test will be performed: The supervisor and a tester will be provided the solution and
  will be asked to view/send an alert. Once the alert is sent/received, the recipient will
  be asked to observe the alert.

  \wss{Remember to add a usability survey for the supervisor located in the Appendix.}

  \item{test-id2\\}
  
  Type: Functional, Dynamic, Manual, Static etc.
            
  Initial State: 
            
  Input: 
            
  Output: 
            
  How test will be performed:
  
  \end{enumerate}

\subsubsection{Performance Requirements}

\begin{enumerate}

  \item{test-CR1\\}
  
  Type: Non-Functional, Dynamic, Manual
            
  Initial State: The solution is opened on a user's web browser.
            
  Input/Condition: Sandlot should have 60 teams and 1500 players stored in the system at
  once.
            
  Output/Result: Sandlot should continue to successfully function with no faults when
  60 teams have 25 players each stored in the system.
            
  How test will be performed: The system will be provided sample inputs for 60 teams and
  25 players on each team. The tester will then observe the system as they navigate the
  solution and utilize functionalities such as rescheduling games, and viewing the standings
  and season schedule.

  \wss{Remember to add a usability survey for the supervisor located in the Appendix.}
            
  \item{test-id2\\}
  
  Type: Functional, Dynamic, Manual, Static etc.
            
  Initial State: 
            
  Input: 
            
  Output: 
            
  How test will be performed: 
  
  \end{enumerate}

\subsubsection{Operational and Environmental Requirements}
\begin{enumerate}

  \item{test-id1\\}
  
  Type: Functional, Dynamic, Manual, Static etc.
            
  Initial State: 
            
  Input/Condition: 
            
  Output/Result: 
            
  How test will be performed: 
            
  \item{test-id2\\}
  
  Type: Functional, Dynamic, Manual, Static etc.
            
  Initial State: 
            
  Input: 
            
  Output: 
            
  How test will be performed: 

  \item{test-RR1\\}

  No test needed for requirement.
  
  \end{enumerate}

\subsubsection{Maintainability and Support Requirements}

\begin{enumerate}

  \item{test-MR1\\}
  
  Type: Non-Functional, Dynamic, Manual
            
  Initial State: The solution is opened on a user's web browser.
            
  Input/Condition: A tester will be asked to start a new season within a time constraint
  of one hour.
            
  Output/Result: A new season is successfully started within one hour of the solution
  being initially opened in the tester's web browser.
            
  How test will be performed: A tester will be provided the solution and asked to start
  a new season by following a set of instructions. A timer will begin at the same time
  the solution is opened on the tester's web browser, in which, the tester should be
  able to start a new season before one hour has passed.
            
  \item{test-id2\\}
  
  Type: Functional, Dynamic, Manual, Static etc.
            
  Initial State: 
            
  Input: 
            
  Output: 
            
  How test will be performed: 
  
  \end{enumerate}

\subsubsection{Security Requirements}

\begin{enumerate}

  \item{test-AS3\\}

  Type: Non-Functional, Dynamic, Manual

  Initial State: The system is logged in to a player level account.

  Input: A reschedule request is attempted.

  Output: No output.

  How test will be performed: A player level account will attempt to perform
  the same actions a captain level account would do to reschedule an upcoming
  game. If they cannot perform these actions, the test succeeds.

  \item{test-AS4\\}
  
  Type: Non-Functional, Dynamic, Manual
            
  Initial State: The solution is opened on a user's web browser.
            
  Input/Condition: A user's account with commissioner-level permissions will be asked to
  send a league-wide alert to all users on the platform.
            
  Output/Result: A commissioner is able to successfully send an alert to the entire league
  that any user is able to view themselves when they have the solution opened.
            
  How test will be performed: A tester will be provided a commissioner account, a
  non-commissioner account, and a sample alert message to send to the entire league
  through the system. They will then be asked to send the alert via the commissioner
  account, login to the non-commissioner account, and observe that the alert message
  was successfully sent to any league users or if any errors had occured. When the tester
  is logged into the non-commissioner account, they will also be asked if they are
  able to send an alert or not or if any other errors occur.

  \item{test-AS7\\}

  Type: Non-Functional, Dynamic, Manual

  Initial State: The system is logged in to a player level account.

  Input: A team data change input is attempted.

  Output: No output.

  How test will be performed: A player level account will attempt to perform
  the same actions a captain level account would do to change the player list
  for their team. If they cannot perform these actions, the test succeeds.

  \item{test-AS8\\}
  
  Type: Non-Functional, Dynamic, Manual
            
  Initial State: The solution is opened on a user's web browser.
            
  Input/Condition: A user's account with commissioner-level permissions will be asked to
  change the season schedule on the platform.
            
  Output/Result: A commissioner is able to successfully change the season schedule
  when they have the solution opened.
            
  How test will be performed: A tester will be provided a commissioner account, a
  non-commissioner account, and sample schedule data they will use to change the season
  schedule. They will then be asked to use the provided schedule data to change the season
  schedule and observe that the season schedule was successfully changed or if any errors
  had occured. The tester will then be asked to login to the non-commissioner account and
  observe if they can change the season schedule or not or if any other errors occur.

  \item{test-AS11-1\\}

  Type: Non-Functional, Dynamic, Manual

  Initial State: The system is logged in to a player level account.

  Input: A user attempts to view the contact information of a player and
  a captain not on their team.

  Output: No output.

  How test will be performed: A player level account will attempt to view the
  contact information of a player and a captain on another team. If they
  cannot perform these actions, the test succeeds.

  \item{test-AS11-2\\}

  Type: Non-Functional, Dynamic, Manual

  Initial State: The system is logged in to a captain level account.

  Input: A user attempts to view the contact information of a player not on
  their team.

  Output: No output.

  How test will be performed: A captain level account will attempt to view the
  contact information of a player on another team. If they cannot perform
  these actions, the test succeeds.

  \item{test-AS12\\}
  
  Type: Non-Functional, Dynamic, Manual
            
  Initial State: The solution is opened on a user's web browser.
            
  Input/Condition: A user's account is provided permissions that are different from the
  the account's current permissions.
            
  Output/Result: The system provides a warning to the user changing an account's
  permissions before the permissions are successfully updated.
            
  How test will be performed: A tester will be provided an account to change the
  permissions for and the account itself. They will then be asked to update the account
  that requires an update to its permissions and observe the warning given by the system
  or if any errors had occured. The tester will then be asked to login to the account
  that recently had its permissions change and try out new functionalities that should
  have been provided with the permission update or if any errors had occured.

  \item{test-IG2\\}

  Type: Non-Functional, Dynamic, Manual

  Initial State: The system is logged in to a commissioner level account.

  Input: A user attempts to delete their account.

  Output: A warning that you cannot delete the only commissioner level
  account.

  How test will be performed: A commissioner level account will attempt to
  delete their account. The test succeeds if they get a warning that their
  account cannot be deleted since it is the only commissioner level account
  and the account is not deleted.

  \item{test-PV2\\}

  Type: Non-Functional, Dynamic, Manual

  Initial State: The system is not logged in.

  Input: A user creates an account.

  Output: A warning to keep your password secure, and a new account is made.

  How test will be performed: A new account will be created. If the user 
  receives a warning to keep their password secure, the test succeeds.

  \item{test-id2\\}
  
  Type: Functional, Dynamic, Manual, Static etc.
            
  Initial State: 
            
  Input: 
            
  Output: 
            
  How test will be performed: 
  
\end{enumerate}

\subsubsection{Cultural Requirements}

\begin{enumerate}

  \item{test-CL1\\}
  
  Type: Non-Functional, Dynamic, Manual
            
  Initial State: A user is creating an account for themselves on the solution.
            
  Input/Condition: A user selects a gender option when creating their account.
            
  Output/Result: A user is able to select a gender option that does not specify their
  gender when creating their account.
            
  How test will be performed: A tester will be provided the solution and will be asked
  to create a new account. They will then be asked to select the gender option that allows
  them to not specify their gender and observe if their information is displayed correctly
  when they are finished creating their account or if any errors had occured.
  
\end{enumerate}

\subsubsection{Migration to the New Product}

\begin{enumerate}

  \item{test-id2\\}
  
  Type: Functional, Dynamic, Manual, Static etc.
            
  Initial State: 
            
  Input: 
            
  Output: 
            
  How test will be performed: 

\end{enumerate}

\subsubsection{User Documentation and Training}

\begin{enumerate}

  \item{test-TR1\\}

  No test needed for this requirement.

\end{enumerate}

\subsection{Traceability Between Test Cases and Requirements}

\wss{Provide a table that shows which test cases are supporting which
  requirements.}

\section{Unit Test Description}

\wss{This section should not be filled in until after the MIS (detailed design
  document) has been completed.}

\wss{Reference your MIS (detailed design document) and explain your overall
philosophy for test case selection.}  

\wss{To save space and time, it may be an option to provide less detail in this section.  
For the unit tests you can potentially layout your testing strategy here.  That is, you 
can explain how tests will be selected for each module.  For instance, your test building 
approach could be test cases for each access program, including one test for normal behaviour 
and as many tests as needed for edge cases.  Rather than create the details of the input 
and output here, you could point to the unit testing code.  For this to work, you code 
needs to be well-documented, with meaningful names for all of the tests.}

\subsection{Unit Testing Scope}

\wss{What modules are outside of the scope.  If there are modules that are
  developed by someone else, then you would say here if you aren't planning on
  verifying them.  There may also be modules that are part of your software, but
  have a lower priority for verification than others.  If this is the case,
  explain your rationale for the ranking of module importance.}

\subsection{Tests for Functional Requirements}

\wss{Most of the verification will be through automated unit testing.  If
  appropriate specific modules can be verified by a non-testing based
  technique.  That can also be documented in this section.}

\subsubsection{Module 1}

\wss{Include a blurb here to explain why the subsections below cover the module.
  References to the MIS would be good.  You will want tests from a black box
  perspective and from a white box perspective.  Explain to the reader how the
  tests were selected.}

\begin{enumerate}

\item{test-id1\\}

Type: \wss{Functional, Dynamic, Manual, Automatic, Static etc. Most will
  be automatic}
					
Initial State: 
					
Input: 
					
Output: \wss{The expected result for the given inputs}

Test Case Derivation: \wss{Justify the expected value given in the Output field}

How test will be performed: 
					
\item{test-id2\\}

Type: \wss{Functional, Dynamic, Manual, Automatic, Static etc. Most will
  be automatic}
					
Initial State: 
					
Input: 
					
Output: \wss{The expected result for the given inputs}

Test Case Derivation: \wss{Justify the expected value given in the Output field}

How test will be performed: 

\item{...\\}
    
\end{enumerate}

\subsubsection{Module 2}

...

\subsection{Tests for Nonfunctional Requirements}

\wss{If there is a module that needs to be independently assessed for
  performance, those test cases can go here.  In some projects, planning for
  nonfunctional tests of units will not be that relevant.}

\wss{These tests may involve collecting performance data from previously
  mentioned functional tests.}

\subsubsection{Module ?}
		
\begin{enumerate}

\item{test-id1\\}

Type: \wss{Functional, Dynamic, Manual, Automatic, Static etc. Most will
  be automatic}
					
Initial State: 
					
Input/Condition: 
					
Output/Result: 
					
How test will be performed: 
					
\item{test-id2\\}

Type: Functional, Dynamic, Manual, Static etc.
					
Initial State: 
					
Input: 
					
Output: 
					
How test will be performed: 

\end{enumerate}

\subsubsection{Module ?}

...

\subsection{Traceability Between Test Cases and Modules}

\wss{Provide evidence that all of the modules have been considered.}
				
\bibliographystyle{plainnat}

\bibliography{../../refs/References}

\newpage

\section{Appendix}

This is where you can place additional information.

\subsection{Symbolic Parameters}

The definition of the test cases will call for SYMBOLIC\_CONSTANTS.
Their values are defined in this section for easy maintenance.

\subsection{Usability Survey Questions?}

\wss{This is a section that would be appropriate for some projects.}

\newpage{}
\section*{Appendix --- Reflection}

\wss{This section is not required for CAS 741}

The information in this section will be used to evaluate the team members on the
graduate attribute of Lifelong Learning.

\input{../Reflection.tex}

\begin{enumerate}
  \item What went well while writing this deliverable? 
  \item What pain points did you experience during this deliverable, and how
    did you resolve them?
  \item What knowledge and skills will the team collectively need to acquire to
  successfully complete the verification and validation of your project?
  Examples of possible knowledge and skills include dynamic testing knowledge,
  static testing knowledge, specific tool usage, Valgrind etc.  You should look to
  identify at least one item for each team member.
  \item For each of the knowledge areas and skills identified in the previous
  question, what are at least two approaches to acquiring the knowledge or
  mastering the skill?  Of the identified approaches, which will each team
  member pursue, and why did they make this choice?
\end{enumerate}

\end{document}