\documentclass{article}

\usepackage{booktabs}
\usepackage{tabularx}

\title{Development Plan\\\progname}

\author{\authname}

\date{}

\input{../Comments}
\input{../Common}

\begin{document}

\maketitle

\begin{table}[hp]
\caption{Revision History} \label{TblRevisionHistory}
\begin{tabularx}{\textwidth}{llX}
\toprule
\textbf{Date} & \textbf{Developer(s)} & \textbf{Change}\\
\midrule
September 24, 2024 & NFI, JL, CG, AV & Initial Draft\\
October 13 & JL & Updated GitHub Project Link, fixed typos, updated section 6 based on TA feedback\\
November 6 & CG & Scaled back POC demo plans to remove menstions of system architecture and advanced UI based on TA feedback\\
November 11 & JL & Updated POC demo plan with specific plan\\
January 6, 2025 & JL & Updated sections 7 and 8 with additional procedures including when to branch for issues, how to connect issues to commits, and adding information on items for the project board.\\
\bottomrule
\end{tabularx}
\end{table}

\newpage{}

This document contains our development plan for the 'Sandlot' project, a
softball league scheduler. It includes team members, team meetings, workflows,
standards, expected technologies, reflections and our team charter.

\section{Confidential Information?}

This project has no confidential information to protect from industry.

\section{IP to Protect}

This project has no intellectual property to protect.

\section{Copyright License}

We are adopting the GNU General Public License v3.0. It is located 
in the root of this repository: \href{https://github.com/Nicholas-Fabugais-Inaba/Sandlot/blob/main/LICENSE}{Sandlot}

\section{Team Meeting Plan}

The team will have two scrum meetings a week on Tuesdays and Thursdays every week
between lectures. We ask every team member to be in attendance unless a valid
excuse is given. These meetings will be in person if possible.\newline

During scrum meetings, the scrum master will ask each team member to describe
work they have done since the last meeting, what they plan to work on, and any
pain points or blockers the have ran into while working.\newline

We will also have flexible meetings if a specific issue arrises. These meetings can be in
person or online. We ask all group members to attend these meetings, but
flexiblity is allowed if a member cannot attend.\newline

Supervisor meetings will be held biweekly in the supervisor's office. If the
team and supervisor agree a meeting is not necessary, it can be cancelled.
\newline

The meeting chair will head non-scrum meetings and will present the weekly
agenda at the beginning of each meeting. The meeting chair will also create an
issue for the meeting with the agenda on GitHub. The assigned team scribe will
record the meeting minutes, and add them to the meeting issue.

\section{Team Communication Plan}

All source code and technical documentation will be managed in a shared GitHub 
repository. Issue tracking will be used to communicate technical information 
about bugs, feature requests, and tasks between group members. By leveraging 
GitHub and issue tracking, the team ensures that technical details and 
contributions are documented, and accessible. This approach promotes 
collaboration and traceability, while minimizing miscommunication.\newline

Discord will be used as the main group communication channel for the team. A majority 
of group communication will be held through Discord including online meetings, admin details, 
and general unplanned discussions. Call and text will be used for direct 
communication and will generally be reserved for high priority and urgent 
communication or campus location coordination. 

\section{Team Member Roles}

\begin{itemize}
	\item Jung Woo Lee
	\begin{itemize}
	  \item Scrum Master: Leads scrum meetings
	  \item Developer
  \end{itemize}
	\item Alex Verity
  \begin{itemize}
    \item Scribe: Takes meeting minutes of any team meetings
	  \item Developer
  \end{itemize}
  \item Nicholas Fabugais-Inaba
  \begin{itemize}
    \item Meeting Chair: Provides agenda for any team meetings
	  \item Developer
  \end{itemize}
  \item Casra Ghazanfari
  \begin{itemize}
    \item Domain Expert: Provides expertise on solution architecture
	  \item Developer
  \end{itemize}
\end{itemize}

If a team member is unable fulfill their role for a time, the rest of the members will
unanimously decide on a person to take their place. 

\section{Workflow Plan}

Git is the most important part in managing issues, editing documentation, and developing features.
\newline

We are using two central branches named documentation and development for writing documentation
and developing software respectively.
The main branch will be merged and updated for milestones, proof of concepts, and
final deployments.
For new features, bug fixes or documents, a branch can be made from the documenation or development branch.
For documentation, each section of the document is to be given a branch and issue pertaining to the completion of that section.
Before any changes are be made, the local repository should be updated to the latest changes on
the remote branch.\newline

Continuous integration will be implemented via GitHub Actions. When a pull request is made, automated
tests will run against the code being tested.
Each pull request will be assigned a GitHub label (i.e. documentation, bug). At least one
approval will be required from a reviewer for each pull request.
All pull requests to the main branch should have no obvious issues. Changes should be made
to the development or documentation branch, and once quality is confirmed, the branch will
be merged with main.\newline

For each issue, a team member should be assigned to the issue on Github. The descriptions
of the issues should be explained in detail and labeled appropriately with a Git label. New labels
should be created, if none of the existing labels match the description of the issue in question.
Issue templates should be used where possible for maintaining the quality and consistency of the
issues. Commits made to solve an issue should reference the specific issue using GitHub's keywords in the commit message.
This allows for better tracking and automatic closing of the issues when the commit is merged to the main branch. Otherwise,
issues should be closed when a pull request is made for the branch.\newline

Milestones and checklists will be used to help keep the team on track. The team will
utilize these from the already existing templates within the Github repository.

\section{Project Decomposition and Scheduling}

GitHub Projects will be used to keep track of the team’s tasks. The ‘board’ feature will be primarily used for tracking issues during scrum meetings. This will ensure the team stays organized and understands what tasks to complete next.
The board will contain the sections ’Todo’, ‘Backlog’, ‘In Progress’, and ‘Done’, with subsections open to be added. ‘Todo’ will contain planned tasks that have not been started. ‘Backlog’ will contain the previous sprint’s items that carry over as well as any items that are currently on hold. ‘In Progress’ will hold items actively being worked on by team members. ‘Done’ will contain items that have been completed.
Items should each be assigned a priority from P0 being the highest priority, to P2 being the lowest priority. Each item should also be given a workload size estimate, from XS being the smallest to XL being the largest.
Tasks should be small in scope and based around features. For example, “Implement Module X”, or “Document Y Section A.B”. Tasks should be specific and measurable.
\\\\
\href{https://github.com/users/Nicholas-Fabugais-Inaba/projects/3}{GitHub project link}
\\\\
Project Schedule:\\
\indent \textbf{Revision 0}\\
\indent Problem Statement, POC, Development Plan \hfill September 24\\
\indent Requirements Document \hfill October 9\\
\indent Hazard Analysis \hfill October 23\\
\indent V\&V Plan \hfill November 1\\
\indent Proof of Concept Demonstration \hfill November 11--22\\
\indent Design Document \hfill January 15\\
\indent Revision 0 Demonstration \hfill February 3--14\\
\indent V\&V Report \hfill March 7\\
\indent \textbf{Revision 1}\\
\indent Final Demonstration \hfill March 24--30\\
\indent EXPO Demonstration \hfill April TBD\\
\indent Final Documentation \hfill April 2\\

\section{Proof of Concept Demonstration Plan}

To be a successful project, the project should be able to intake a set of 
availability data and generate an optimized schedule based on that data. 
Additionally, it should be able to display this schedule through a basic
user interface. The main risks for the success of this project are whether 
we are able to optimize schedule generation to a point which satisfies our 
stakeholders, and our team's lack of domain knowledge on scheduling problems 
and web development. \\

In order to demonstrate that these obstacles can be 
overcome, the goal of this proof of concept demonstration is to develop
an algorithm to generate highly optimized schedules based on team availability 
data. Furthermore these optimized schedules should be displayed through a 
basic UI hosted on a webserver.

Demonstration Plan:
\begin{itemize}
  \item Quick overview of the POC demonstration
    \subitem Scheduling algorithm and UI visualizer
  \item Explanation of scheduling algorithm
    \subitem Problem formulation
    \subitem Algorithm to solve
    \subitem Algorithm details and other notes
  \item Demonstration of POC
    \subitem Show UI
    \subitem Generate schedule
    \subitem Show that each constraint is satisfied
\end{itemize}

\section{Expected Technology}

\begin{itemize}
\item Programming languages
\subitem The project can roughly be divided into 3 main components:
\begin{enumerate}
  \item A database where the bulk of the site's data will be stored.
  \item A webserver which will host the website's visual data and code.
  \item Middleware which will allow for communication between the 
  webserver and database.
\end{enumerate}
\subitem The middleware will be written in Python due to its ease of use and 
familiarity among team members.
\subitem We will either use PostgreSQL scripts or an ORM (like SQLalchemy) 
to implement the database, the language used will depend on the implementation 
chosen. PostgreSQL scripts would require using only SQL, while an ORM would be written in Python.
\subitem The webserver will be written in Javascript using the React framework.
This is because of its ease of use, team member experience, and large set 
of available libraries which will be useful for implementing the user interface of the website. Additionally, it was expressed to us by stakeholders that the 
website should be easily maintainable. Implementing the website using an 
extremely popular and widespread framework such as React means that there 
are countless resources online to help future maintainers keep the project alive.

\item Libraries
\subitem Depending on the implementation, the middleware will use some combination
of the FastAPI and SQLalchemy libraries. FastAPI will be used to implement HTTPS 
communication routers between the database and webserver. SQLalchemy will be used to 
implement the database if it is decided that it will be implemented using an ORM.
\subitem The webserver will use a large set of both functional and visual React
libraries. For example, react-navigation will be used for its page traversal 
functionality while libraries such as react-datepicker and react-calendar will 
be very helpful when implementing the visual elements of a scheduling system. 
Additionally, the axios library will be used to form and send the HTTPS requests 
to the middleware from the webserver.
\item Pre-trained models
\subitem This project does not include an AI component and will not use 
a machine learning model
\item Linter tools
\subitem ESLint will be used for linting Javascript code while flake8 will be 
used to lint Python code. We chose these linters due to our previous experience 
using them and because they provide our preferred formatting style.
\item Unit testing frameworks
\subitem The pytest framework will be used to create unit tests for the middleware
code. We chose pytest over other python testing frameworks due to its simplicity, 
small amount of boilerplate code, and plugins which can add useful functionalities
like coverage reporting. We plan to incorporate these pytest unit tests as a part
of our CI plans for the project via Github actions.
\subitem Testing the database will likely be done using a dummy / development 
PostgreSQL database prior to making any changes to the production database to 
ensure that minimal migrations are required during development.
\item Code coverage measuring tools
\subitem The Coverage.py Python library will be used to measure the code coverage
of our middleware program. For the webserver's React code, Jest is included by 
default when using the 'create-react-app' command and will be used to measure 
the code coverage of the webserver.
\item Performance measuring tools
\subitem The webserver will be hosted on Azure's web app services, meaning that
Azure's suite of performance measurement tools and metrics will be used as the 
webserver's main performance measurement system. Additionally, we plan to do
practical performance tests with stakeholders by having them use the website 
casually to ensure that performance during regular use is up to their
standards/expectations.
\subitem Similarly, the database will be hosted on an Azure container, meaning 
that Azure's suite of performance measurement tools and metrics will once again 
be used as the databases main performance measurement system. Additionally, 
performance of queries will be timed throughout development to determine what 
indices should exist on the database for practical performance.
\subitem Finally, the middleware will use the profile library included with 
Python to measure the performance of its HTTPS routes.
\item Other tools
\subitem Azure containers and/or virtual machines will be used to host both the 
database and middleware components of this project. Additionally, the webserver 
will be hosted using Azure's web app services. This allows all main components 
of the project to be hosted in one place.
\subitem Node.js will be the server environment which we run our webserver on 
and npm will be what we use to manage our JavaScript 
packages.
\subitem Git/GitHub will be used for version control on the project. GitHub 
projects will be used as a general project management tool to help keep 
track of issues, work done, and available tasks. Finally, GitHub actions will 
be used for CI of tests as the repository is modified.
\end{itemize}

\section{Coding Standard}

Borrows from: \newline
\href{https://www.lambdatest.com/learning-hub/coding-standards}{A Complete Guide to Coding Standards and Best Practices}

\begin{itemize}
  \item When using JavaScript:
  \begin{itemize}
    \item Use camel case (exampleVariable) for variables and functions.
    \item Use pascal case (ExampleClass) for classes.
    \item Include semi-colons at the end of each statement.
  \end{itemize}
  \item When using Python:
  \begin{itemize}
    \item Use snake case (example\_variable) for variables and functions.
    \item Use pascal case (ExampleClass) for classes.
  \end{itemize}
  \item When using any language:
  \begin{itemize}
    \item Use four spaces when indenting.
    \item Use whitespace to separate functions and code blocks for readability.
    \item Include comments where applicable, in areas where code may be unclear.
    \item Use variable and function names that describe the use of the variable
    or function.
    \item Limit the use of global variables wherever possible.
  \end{itemize}
\end{itemize}


\newpage{}

\section*{Appendix --- Reflection}

\subsection*{Reflection -- Alex Verity}

It is important to create a development plan prior to starting a project to
allow the team to get on the same page and form a line of communication. By
setting standards and rules early it avoids unnessecary conflicts. It is also
important as a way to fasttrack important decisions like a workflow plan and
expected technologies. \newline

Some advantages of continuous integration are that teams can find errors
quickly and often, which can stop an error snowballing or hiding under the
radar until it reaches the user. It makes the development of software more
predictable and errors easier to find, as you are typically searching for
them in smaller chunks of software rather than the entire code base.
Some disadvantages of continuous integration are that it requires significant
effort to set up and modify if needed. This overhead in some cases is not
worth the gains, and it is up to the developer to decide this. As with all
automated testing, some more complicated errors can fall through the cracks,
so manual testing is still needed. \newline

In most things the team was in agreement, but whether or not to use GitHub's
label feature for issues was a small disagreement. To resolve this, we asked
outside sources and the team agreed on a consistent style for issues.

\subsection*{Reflection -- Nicholas Fabugais-Inaba}

The importance of the development plan is to help the team understand the
standards that we have set ourselves, so that throughout the development
process, everyone is aligned with the objectives that need to be
accomplished at a given time. Roles, communication methods, workflows,
and more, remind team members of the structure that takes place to
keep all of us and the project organized. \newline

There are many advantages that come with CI/CD being implemented in
the development process. The biggest role is its ability to aid developers
in making sure that everything compiles correctly and any errors are
highlighted to the user. This allows the developer to fix their mistakes
that may be fatal to what the user may interact with when the changes are
eventually deployed. One disadvantage may be when the developer relies
too much on the usage of CI/CD as although it may catch some errors
it might not catch all of them, even the ones that the developer may have
missed. As a developer, it is important to not only use automated testing,
especially with larger code bases that are constantly experiencing changes,
but also conduct their own testing to make sure the features that they are
implementing, work as intended. \newline

A disagreement that the group had in this deliverable was about the structure
of pull requests and the naming conventions associated with them. After some
deliberation, the group collectively came to the conclusion that the best
method to organize our pull requests would be to not only state what the
commits or changes were about, but also attach a label in Github, indicating
the category the pull request focuses on (i.e. documentation, bug).

\subsection*{Reflection -- Jung Woo Lee}

A development plan is crucial prior to beginning the project to lay out 
necessary information for the team to follow. The information it provides 
guides each team member’s actions regarding the project, ensuring the 
quality of their work, work ethic, and the project as a whole. The 
development plan being created beforehand further reduces risks of 
disagreements on fundamental details of project management and operations
occurring in the middle of the project, where it would be more cumbersome 
to adapt. It can also simplify the team’s work, as ambiguities regarding 
certain aspects of the project should be clarified in this document. \newline

An advantage of CI/CD is that it can quickly and automatically perform
tests on one’s code in a standardized method, such that a certain defined 
quality is met. This can catch common problems and also might reduce the
need of each developer testing every aspect of their code. It can also 
integrate with GitHub to automatically create issues, saving on a lot 
of boiler-plate actions; tests can be imported easily as well. A 
disadvantage is that setup may be complicated, and for an efficacious 
CI/CD system, a lot of thought and time needs to be put into it. There 
may also be a psychological aspect where developers may be led into a 
false sense of security when passing the automated tests. Lastly, 
automated tests do not pertain to usability aspects of the project, 
which requires manual testing and stakeholder interaction.\newline

Slight disagreement regarding use of labelling and tagging occurred 
but was resolved by consulting with a TA. Slight disagreement regarding 
meeting schedule and times was resolved through discussion and consensus. 
There was significant disagreement about the design of the scrum board 
about the sections. This was resolved through discussion and agreeing 
to changes later if needed.\newline

\subsection*{Reflection -- Casra Ghazanfari}

It is important to create a development plan before starting a project because
of the number of benefits it provide to a project’s development cycle. One
of its main benefits is increased clarity on the objectives of the project
for those involved. Development plans achieve this by identifying and defining
the project’s goals/objectives early in the development cycle, meaning that
those involved in the project will more easily and quickly understand the \
project's goals greatly reducing potential misunderstandings. Additionally,
it acts as a fantastic tool for risk management as it allows you to identify
any risks or challenges associated with the project at an early stage.
Furthermore, they allow you to develop strategies to mitigate these
risks/challenges from causing any failure in the project in the future. \newline

Two of the biggest advantages of CI/CD are early detection of bugs and
decreased manual errors. Through the automation of testing every code change
using CI/CD, bugs can be caught much earlier than if tests were manually done
by humans. Additionally, when automating pipelines using CI/CD a large amount
of manual processes are eliminated, greatly reducing the number of points of
potential manual error and greatly increasing reliability of the system.
However, CI/CD comes with its share of downsides, the two biggest being the
cost of setup and maintenance overhead. Setting up CI/CD is not cheap, it takes
a large amount of time when unfamiliar with the process or requires experience to
implement it quickly. Additionally, not only is CI/CD costly to set up but
also costly to maintain. CI/CD systems require continuous maintenance through
updating tools, dependencies and other means. Overall CI/CD comes with many
disadvantages and advantages that can weigh differently depending on the project
and team. \newline

The group disagreed on whether our proof of concept demonstration plan should
include a UI element. Some members argued that UI elements would not constitute
a “main risk for the success of the project” and that they wouldn’t qualify for
“something that would keep them up at night” while others thought that the UI
element of this project was such a key part that it should be included. We
resolved this by having a group discussion to explain our stance on the matter
and compared reasonings on each side of the argument. In the end, it was decided
that because a well designed UI was such an important factor to the stakeholders
of the project, if the UI were not extremely user friendly it could cause
problems for the project down the line. The proof of concept demonstration plan
now includes a UI element to reflect this.

\newpage{}

\section*{Appendix --- Team Charter}

Borrows from:
\href{https://engineering.up.edu/industry_partnerships/files/team-charter.pdf}
{University of Portland Team Charter}

\subsection*{External Goals}

Our team's external goals are to get a good mark in the course. Our team
considers a 85\% or above a good mark. We also want to have a well made and
well documented project to show on resumes and discuss in interviews.

\subsection*{Attendance}

\subsubsection*{Expectations}

Our team's exepectations regarding meeting attendance are flexible, as long as everyone is doing
a good share of the work. We ask all team members are on time and do not leave early during
meetings, although we are flexible as long as missed meeting time is minimal. We ask all team 
members attend meetings whenever possible, and if circumstances prevent a member from attending
a meeting they make an effort to be informed on what was discussed.

\subsubsection*{Acceptable Excuse}

An acceptable excuse for missing a meeting would be illness, family emergency. If the excuse does not fall within a listed category,
the entire team can unanimously accept the excuse. Excuses that will not be considered acceptable
are missing a meeting due to an upcoming assignment due date or test, or due to recreational
activity.

\subsubsection*{In Case of Emergency}

In case of emergency, we ask team members communicate how much work or how many meetings they will be missing, and
make a plan to make up missed work. The team will be understanding but will also expect that once the
emergency is resolved, the team member comes back ready and willing to get back on track.

\subsection*{Accountability and Teamwork}

\subsubsection*{Quality} 

Our expectations are that every member come to meetings ready to show what they have acomplished
in the time since last meeting. The deliverables should have no compilation errors and should
have no obvious faults. Deliverables should follow the coding standard and should not be
difficult to understand.

\subsubsection*{Attitude}

Team members should strive to be respectful of each other and their ideas, though critical review should not be discouraged due to this. 
Members should strive to appreciate each other’s contributions as well as being patient when mistakes are made.
Each team member will be respectful to the supervisor and any outside party directly or indirectly involved with the project. This means in demeanor as well as respecting their time.

\subsubsection*{Stay on Track}

The primary method our team will use to keep on track will be scrum meetings.
Our scrum master will ask each member what they have worked on since the last
meeting, which will inform the group of each other's progress. This will let
the group know if we are falling behind on the project, and will let any team
member know if they are less productive than the others.\newline

We will use attendance metrics to make sure team members are attending enough
meetings. If a member misses three meetings in the past two weeks, the team will do a
check in and make sure the member is on track. We will also loosely monitor
commit metrics on GitHub to make sure members are on track. 

\subsubsection*{Team Building}
The team will participate in a group crosswords and in sports days.

\subsubsection*{Decision Making} 

All disagreements should be discussed with the whole team, or at the very least, relevant team members.
Disagreements regarding a major aspect of the project should require a unanimous agreement.
More minor disagreements may be agreed upon by majority.
Upon ties, a relevant third-party should assist in the decision making.

\end{document}