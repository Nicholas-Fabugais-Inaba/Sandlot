\documentclass[12pt, titlepage]{article}

\usepackage{amsmath, mathtools}

\usepackage[round]{natbib}
\usepackage{amsfonts}
\usepackage{amssymb}
\usepackage{graphicx}
\usepackage{colortbl}
\usepackage{xr}
\usepackage{hyperref}
\usepackage{longtable}
\usepackage{xfrac}
\usepackage{tabularx}
\usepackage{float}
\usepackage{siunitx}
\usepackage{booktabs}
\usepackage{multirow}
\usepackage[section]{placeins}
\usepackage{caption}
\usepackage{fullpage}

\hypersetup{
bookmarks=true,     % show bookmarks bar?
colorlinks=true,       % false: boxed links; true: colored links
linkcolor=red,          % color of internal links (change box color with linkbordercolor)
citecolor=blue,      % color of links to bibliography
filecolor=magenta,  % color of file links
urlcolor=cyan          % color of external links
}

\usepackage{array}

\externaldocument{../../SRS/SRS}

\input{../../Comments}
\input{../../Common}

\begin{document}

\title{Module Interface Specification for \progname{}}

\author{\authname}

\date{\today}

\maketitle

\pagenumbering{roman}

\section{Revision History}

\begin{tabularx}{\textwidth}{p{3cm}p{2cm}X}
\toprule {\bf Date} & {\bf Version} & {\bf Notes}\\
\midrule
January 13, 2025 & 1.0 & TA Feedback\\
January 15, 2025 & 1.1 & Rev0\\
\bottomrule
\end{tabularx}

~\newpage

\section{Symbols, Abbreviations and Acronyms}

See SRS Documentation at \url{https://github.com/Nicholas-Fabugais-Inaba/Sandlot}

\newpage

\tableofcontents

\newpage

\pagenumbering{arabic}

\section{Introduction}

The following document details the Module Interface Specifications for
the implemented modules in a platform designed to organize a seasonal
softball league. It is intended to ease navigation through the platform
for design and maintenance purposes.

Complementary documents include the System Requirement Specifications
and Module Guide.  The full documentation and implementation can be
found at \url{https://github.com/Nicholas-Fabugais-Inaba/Sandlot}.  

\section{Notation}

The structure of the MIS for modules comes from \citet{HoffmanAndStrooper1995},
with the addition that template modules have been adapted from
\cite{GhezziEtAl2003}.  The mathematical notation comes from Chapter 3 of
\citet{HoffmanAndStrooper1995}.  For instance, the symbol := is used for a
multiple assignment statement and conditional rules follow the form $(c_1
\Rightarrow r_1 | c_2 \Rightarrow r_2 | ... | c_n \Rightarrow r_n )$.

The following table summarizes the primitive data types used by \progname. 

\begin{center}
\renewcommand{\arraystretch}{1.2}
\noindent 
\begin{tabular}{l l p{7.5cm}} 
\toprule 
\textbf{Data Type} & \textbf{Notation} & \textbf{Description}\\ 
\midrule
character & char & a single symbol or digit\\
integer & $\mathbb{Z}$ & a number without a fractional component in (-$\infty$, $\infty$) \\
natural number & $\mathbb{N}$ & a number without a fractional component in [1, $\infty$) \\
real & $\mathbb{R}$ & any number in (-$\infty$, $\infty$)\\
schedule & $S$ & a list of games to be played in a season, see Schedule
Structure Module \ref{mSS}\\
standings & $ST$ & a team's record in a season including wins, losses,
ties, forfeits, and point differential, see Standings
Structure Module \ref{mST}\\
game & $G$ & a time, date, location, score, team1, and team2 that defines a game to
be played\\
player & $P$ & a player on a team, uses the Account Structure Module
\ref{mAS}\\
team & $T$ & a team in the league including a team's id, team's name, team's division,
list of players on the team, and the team's standing in the league,
see Team Structure Module \ref{mTS}\\
% date & $d$ & a date in the format YYYY-MM-DD\\
start date & $d_s$ & a date that represents the start of the season\\
end date & $d_e$ & a date that represents the end of the season\\
% time & t & $a$ time in the format HH:MM (24 hour clock)\\
location & $l$ & an integer representing a field\\
division & $D$ & an integer indexing a list of teams that play each other in a
season\\
alert & $A$ & a message to be sent to a list of players, see Alerts Module
\ref{mAL}\\

\bottomrule
\end{tabular} 
\end{center}

\noindent
The specification of \progname \ uses some derived data types: sequences, strings, and
tuples. Sequences are lists filled with elements of the same data type. Strings
are sequences of characters. Tuples contain a list of values, potentially of
different types. In addition, \progname \ uses functions, which
are defined by the data types of their inputs and outputs. Local functions are
described by giving their type signature followed by their specification.

\section{Module Decomposition}

The following table is taken directly from the Module Guide document for this project.

\begin{table}[h!]
\centering
\begin{tabular}{p{0.3\textwidth} p{0.6\textwidth}}
\toprule
\textbf{Level 1} & \textbf{Level 2}\\
\midrule

{Hardware-Hiding Module} & ~ \\
\midrule

\multirow{11}{0.3\textwidth}{Behaviour-Hiding Module} & Account Module\\
& Player Module\\
& Team Module\\
& Commissioner Module\\
& Account Structure Module\\
& Team Structure Module\\
& Schedule Structure Module\\
& Standings Structure Module\\
& Reschedule Module\\
& Alerts Module\\
& Database Module\\
\midrule

\multirow{2}{0.3\textwidth}{Software Decision Module} & Season Scheduler
Module\\
& Web Application Framework Module\\
\bottomrule

\end{tabular}
\caption{Module Hierarchy}
\label{TblMH}
\end{table}

\newpage
~\newpage

\section{MIS of Account Module} \label{mAC}

\wss{You can reference SRS labels, such as R\ref{R_Inputs}.}

\wss{It is also possible to use \LaTeX for hypperlinks to external documents.}

\subsection{Module}

Account

\subsection{Uses}

account\_structure \ref{mAS} \\
Database \ref{mDB} \\
web\_app\_framework \ref{mWA}

\subsection{Syntax}

\subsubsection{Exported Constants}

\subsubsection{Exported Access Programs}

\begin{center}
\begin{tabular}{p{3cm} p{9cm} p{2cm} p{2cm}}
\hline
\textbf{Name} & \textbf{In} & \textbf{Out} & \textbf{Exceptions} \\
\hline
create\_account & name:$string$, email:$string$, pass:$string$, num:$\mathbb{N}$ & - & - \\
delete\_account & - & - & - \\
change\_name & new\_name:$string$ & - & - \\
change\_pass & new\_pass:$string$ & - & - \\
change\_email & new\_email:$string$ & - & - \\
change\_num & new\_num:$\mathbb{N}$ & - & - \\
account\_login & email:$string$, pass:$string$ & - & - \\
\hline
\end{tabular}
\end{center}

\subsection{Semantics}

\subsubsection{State Variables}

account: $P$

\subsubsection{Environment Variables}

None

\subsubsection{Assumptions}

For every team in the league, each team plays at least twice the amount of games
as the number of teams in a division.

% Every division has the same amount of teams? Not necessarily if number of
% teams is odd.

% Every team plays the same amount of games? Not necessarily because of above.

\subsubsection{Access Routine Semantics}

\noindent create\_account():
\begin{itemize}
\item transition: Update the database with the new account information.
\end{itemize}

\noindent delete\_account():
\begin{itemize}
\item transition: Update the database to remove the account.
\end{itemize}

\noindent change\_name():
\begin{itemize}
\item transition: Update the database to change the account's name.
\end{itemize}

\noindent change\_pass():
\begin{itemize}
\item transition: Update the database to change the account's password.
\end{itemize}

\noindent change\_email():
\begin{itemize}
\item transition: Update the database to change the account's email.
\end{itemize}

\noindent change\_num():
\begin{itemize}
\item transition: Update the database to change the account's phone number.
\end{itemize}

\noindent account\_login():
\begin{itemize}
\item transition: Access is given to the account if the email and password match.
\end{itemize}

\newpage

\section{MIS of Player Module} \label{mPL}

\wss{You can reference SRS labels, such as R\ref{R_Inputs}.}

\wss{It is also possible to use \LaTeX for hypperlinks to external documents.}

\subsection{Module}

Player

\subsection{Uses}

Account \ref{mAC}

\subsection{Syntax}

\subsubsection{Exported Constants}

\subsubsection{Exported Access Programs}

\begin{center}
\begin{tabular}{p{2cm} p{4cm} p{4cm} p{2cm}}
\hline
\textbf{Name} & \textbf{In} & \textbf{Out} & \textbf{Exceptions} \\
\hline
join\_team & team:$T$ & - & - \\
leave\_team & - & - & - \\
\hline
\end{tabular}
\end{center}

\subsection{Semantics}

\subsubsection{State Variables}

None

\subsubsection{Environment Variables}

None

\subsubsection{Assumptions}

\wss{Try to minimize assumptions and anticipate programmer errors via
  exceptions, but for practical purposes assumptions are sometimes appropriate.}

\subsubsection{Access Routine Semantics}

\noindent join\_team():
\begin{itemize}
\item transition: Modify the inputted team's data to include the player and
update the player's data to include the team.
\end{itemize}

\noindent leave\_team():
\begin{itemize}
\item transition: Modify the team the player is currently on structure to not
include the player and update the player's data to not include the team.
\end{itemize}

\subsubsection{Local Functions}

\wss{As appropriate} \wss{These functions are for the purpose of specification.
  They are not necessarily something that is going to be implemented
  explicitly.  Even if they are implemented, they are not exported; they only
  have local scope.}

\newpage

\section{MIS of Team Module} \label{mTE}

\wss{You can reference SRS labels, such as R\ref{R_Inputs}.}

\wss{It is also possible to use \LaTeX for hypperlinks to external documents.}

\subsection{Module}

Team

\subsection{Uses}

Account \ref{mAC} \\
team\_structure \ref{mTS} \\
Reschedule \ref{mRE}

\subsection{Syntax}

\subsubsection{Exported Constants}

\subsubsection{Exported Access Programs}

\begin{center}
\begin{tabular}{p{3cm} p{7cm} p{2cm} p{2cm}}
\hline
\textbf{Name} & \textbf{In} & \textbf{Out} & \textbf{Exceptions} \\
\hline
submit\_score & team\_1:$\mathbb{N}$,team\_2:$\mathbb{N}$ & - & - \\
\hline
\end{tabular}
\end{center}

\subsection{Semantics}

\subsubsection{State Variables}

team: $T$

\subsubsection{Environment Variables}

None

\subsubsection{Assumptions}

\wss{Try to minimize assumptions and anticipate programmer errors via
  exceptions, but for practical purposes assumptions are sometimes appropriate.}

\subsubsection{Access Routine Semantics}

\noindent submit\_score():
\begin{itemize}
\item transition: The team's $ST$ is updated with the new team record.
\end{itemize}

\newpage

\section{MIS of Commissioner Module} \label{mCM}

\wss{You can reference SRS labels, such as R\ref{R_Inputs}.}

\wss{It is also possible to use \LaTeX for hypperlinks to external documents.}

\subsection{Module}

Commissioner

\subsection{Uses}

Player \ref{mPL}\\
Alerts \ref{mAL}\\
Season Scheduler \ref{mS}

\subsection{Syntax}

\subsubsection{Exported Constants}

\subsubsection{Exported Access Programs}

\begin{center}
\begin{tabular}{p{5cm} p{4cm} p{4cm} p{2cm}}
\hline
\textbf{Name} & \textbf{In} & \textbf{Out} & \textbf{Exceptions} \\
\hline
overwrite\_team\_comp & team:$T$, player\_list:P[] & - & - \\
overwrite\_schedule & sched\_new:$S$ & - & - \\
overwrite\_standings & new\_standing:$ST$ & - & - \\
overwrite\_game\_score & game:G, new\_score:$\mathbb{N}$ & - & - \\
update\_team\_contact & team:$T$, new\_info:$string$ & - & - \\
set\_team\_division & team:$T$ division:$D$ & - & - \\
add\_division & division:$D$ & - & - \\

\hline
\end{tabular}
\end{center}

\subsection{Semantics}

\subsubsection{State Variables}

None

\subsubsection{Environment Variables}

None

\subsubsection{Assumptions}

\wss{Try to minimize assumptions and anticipate programmer errors via
  exceptions, but for practical purposes assumptions are sometimes appropriate.}

\subsubsection{Access Routine Semantics}

\noindent overwrite\_team\_comp():
\begin{itemize}
\item transition: The player list of the inputted team is updated with the 
      inputted player list.
\end{itemize}

\noindent overwrite\_schedule():
\begin{itemize}
\item transition: The current schedule is updated with the inputted new
      schedule.
\end{itemize}

\noindent overwrite\_standings():
\begin{itemize}
\item transition: The current standings are updated with the inputted new
      standings.
\end{itemize}

\noindent overwrite\_game\_score():
\begin{itemize}
\item transition: The inputted game's score is updated with the inputted new
      score.
\end{itemize}

\noindent update\_team\_contact():
\begin{itemize}
\item transition: The inputted team's contact information is updated with the
      inputted new information.
\end{itemize}

\noindent set\_team\_division():
\begin{itemize}
\item transition: The inputted team's division is updated with the inputted
      division.
\end{itemize}

\noindent add\_division():
\begin{itemize}
\item transition: The inputted division is added to the league.
\end{itemize}

\subsubsection{Local Functions}

\wss{As appropriate} \wss{These functions are for the purpose of specification.
  They are not necessarily something that is going to be implemented
  explicitly.  Even if they are implemented, they are not exported; they only
  have local scope.}

\newpage

\section{MIS of Account Structure Module} \label{mAS}

\subsection{Module}

account\_structure

\subsection{Uses}

N/A

\subsection{Syntax}

N/A

\subsection{Semantics}

\subsubsection{State Variables}

name: string\\
pass: string\\
email: string\\
num: string\\
team\_id: $\mathbb{N}$

\subsubsection{Environment Variables}

None

\newpage

\section{MIS of Team Structure Module} \label{mTS}

\subsection{Module}

team\_structure

\subsection{Uses}

account\_structure \ref{mAS}

\subsection{Syntax}

N/A

\subsection{Semantics}

\subsubsection{State Variables}

team\_id: $\mathbb{N}$\\
team\_name: string\\
div: $D$\\
players: $P[]$\\
standing: $ST$

\subsubsection{Environment Variables}

None

\newpage

\section{MIS of Schedule Structure Module} \label{mSS}

\subsection{Module}

schedule\_structure

\subsection{Uses}

N/A

\subsection{Syntax}

N/A

\subsection{Semantics}

\subsubsection{State Variables}

games: $G[]$

\subsubsection{Environment Variables}

None

\newpage

\section{MIS of Standings Structure Module} \label{mST}

\subsection{Module}

standings\_structure

\subsection{Uses}

N/A

\subsection{Syntax}

N/A

\subsection{Semantics}

\subsubsection{State Variables}

wins: $\mathbb{N}$\\
losses: $\mathbb{N}$\\ 
ties: $\mathbb{N}$\\
forfeits: $\mathbb{N}$\\
point\_diff: $\mathbb{Z}$

\subsubsection{Environment Variables}

None

\subsubsection{Local Functions}

calc\_score: $\mathbb{N}, \mathbb{N}, \mathbb{N}, \mathbb{N} \rightarrow \mathbb{N}$\\
calc\_score(wins, losses, ties, forfeits) = $2*wins + ties - forfeits$

\newpage

\section{MIS of Reschedule Module} \label{mRE}

\wss{You can reference SRS labels, such as R\ref{R_Inputs}.}

\wss{It is also possible to use \LaTeX for hyperlinks to external documents.}

\subsection{Module}

Reschedule

\subsection{Uses}

Team \ref{mTE} \\
web\_app\_framework \ref{mWA}

\subsection{Syntax}

\subsubsection{Exported Constants}

\subsubsection{Exported Access Programs}

\begin{center}
\begin{tabular}{p{5cm} p{4cm} p{4cm} p{2cm}}
\hline
\textbf{Name} & \textbf{In} & \textbf{Out} & \textbf{Exceptions} \\
\hline
request\_reschedule & old\_game:G, new\_game:G & - & - \\
accept\_reschedule & - & - & - \\
\hline
\end{tabular}
\end{center}

\subsection{Semantics}

\subsubsection{State Variables}

None

\subsubsection{Environment Variables}

None

\subsubsection{Assumptions}

\wss{Try to minimize assumptions and anticipate programmer errors via
  exceptions, but for practical purposes assumptions are sometimes appropriate.}

\subsubsection{Access Routine Semantics}

\noindent request\_reschedule():
\begin{itemize}
\item transition: The inputted old game is replaced with the inputted new game that has
      the new date, time, and location.
\end{itemize}

\noindent accept\_reschedule():
\begin{itemize}
\item transition: If the the rescheduling is accepted, the old game is removed and the
      new game is added to the schedule.
\end{itemize}

\wss{A module without environment variables or state variables is unlikely to
  have a state transition.  In this case a state transition can only occur if
  the module is changing the state of another module.}

\wss{Modules rarely have both a transition and an output.  In most cases you
  will have one or the other.}

\subsubsection{Local Functions}

\wss{As appropriate} \wss{These functions are for the purpose of specification.
  They are not necessarily something that is going to be implemented
  explicitly.  Even if they are implemented, they are not exported; they only
  have local scope.}

\newpage

\section{MIS of Alerts Module} \label{mAL}

\subsection{Module}

Alerts

\subsection{Uses}

web\_app\_framework \ref{mWA}

\subsection{Syntax}

\subsubsection{Exported Constants}

\subsubsection{Exported Access Programs}

\begin{center}
\begin{tabular}{p{3cm} p{5cm} p{2cm} p{2cm}}
\hline
\textbf{Name} & \textbf{In} & \textbf{Out} & \textbf{Exceptions} \\
\hline
create\_alert & msg:string, players:$P[]$ & alert:$A$ & - \\
send\_alert & alert:$A$ & - & - \\
\hline
\end{tabular}
\end{center}

\subsection{Semantics}

\subsubsection{State Variables}

None

\subsubsection{Environment Variables}

None

\subsubsection{Assumptions}

\wss{Try to minimize assumptions and anticipate programmer errors via
  exceptions, but for practical purposes assumptions are sometimes appropriate.}

\subsubsection{Access Routine Semantics}

\noindent create\_alert():
\begin{itemize}
\item output: $A$, an alert created according to the user specifications.
\end{itemize}

\noindent send\_alert():
\begin{itemize}
\item transition: An alert is sent to the players in the inputted player list.
\end{itemize}

\subsubsection{Local Functions}

\noindent select\_target():
\begin{itemize}
\item output: $P[]$, a list of players to send the alert to
\end{itemize}

\newpage

\section{MIS of Database Module} \label{mDB}

\wss{You can reference SRS labels, such as R\ref{R_Inputs}.}

\wss{It is also possible to use \LaTeX for hypperlinks to external documents.}

\subsection{Module}

Database

\subsection{Uses}

team\_structure \ref{mTS} \\
schedule\_structure \ref{mSS} \\
standings\_structure \ref{mST}

\subsection{Syntax}

\subsubsection{Exported Constants}

\subsubsection{Exported Access Programs}

\begin{center}
\begin{tabular}{p{2cm} p{4cm} p{4cm} p{2cm}}
\hline
\textbf{Name} & \textbf{In} & \textbf{Out} & \textbf{Exceptions} \\
\hline
\wss{accessProg} & - & - & - \\
\hline
\end{tabular}
\end{center}

\subsection{Semantics}

\subsubsection{State Variables}

None

\subsubsection{Environment Variables}

None

\subsubsection{Assumptions}

\wss{Try to minimize assumptions and anticipate programmer errors via
  exceptions, but for practical purposes assumptions are sometimes appropriate.}

\subsubsection{Access Routine Semantics}

\noindent \wss{accessProg}():
\begin{itemize}
\item transition: \wss{if appropriate} 
\item output: \wss{if appropriate} 
\item exception: \wss{if appropriate} 
\end{itemize}

\wss{A module without environment variables or state variables is unlikely to
  have a state transition.  In this case a state transition can only occur if
  the module is changing the state of another module.}

\wss{Modules rarely have both a transition and an output.  In most cases you
  will have one or the other.}

\subsubsection{Local Functions}

\wss{As appropriate} \wss{These functions are for the purpose of specification.
  They are not necessarily something that is going to be implemented
  explicitly.  Even if they are implemented, they are not exported; they only
  have local scope.}

\newpage

\section{MIS of Season Scheduler Module} \label{mS}

\wss{You can reference SRS labels, such as R\ref{R_Inputs}.}

\wss{It is also possible to use \LaTeX for hypperlinks to external documents.}

\subsection{Module}

season\_scheduler

\subsection{Uses}

schedule\_structure \ref{mSS}

\subsection{Syntax}

\subsubsection{Exported Constants}

\subsubsection{Exported Access Programs}

\begin{center}
\begin{tabular}{p{2cm} p{4cm} p{4cm} p{2cm}}
\hline
\textbf{Name} & \textbf{In} & \textbf{Out} & \textbf{Exceptions} \\
\hline
gen\_sched & $D[], d_s, d_e, l[]$ & $S$ & - \\
\hline
\end{tabular}
\end{center}

\subsection{Semantics}

\subsubsection{State Variables}

None

\subsubsection{Environment Variables}

None

\subsubsection{Assumptions}

\wss{Try to minimize assumptions and anticipate programmer errors via
  exceptions, but for practical purposes assumptions are sometimes appropriate.}

\subsubsection{Access Routine Semantics}

\noindent gen\_sched():
\begin{itemize}
\item output: $out := S_o \in S$ such that:
\begin{itemize}
  \item No two games in $S$ should share a date, time, and location:\\
  $c_1(s:S) = \forall(g_1, g_2 \in s | g_1 \neq g_2 : g_1.date \neq
  g_2.date \land g_1.time \neq g_2.time \land g_1.location \neq g_2.location)$
  \item No game's parameters should include the same team twice.\\
  $c_2(s:S) = \forall (g \in s |: g.team1 \neq g.team2)$
  \item No team can play more than one game in a day.\\
  $c_3(s:S) = \forall (g_1, g_2 \in s |g_1 \neq g_2 \land g_1.date =
  g_2.date : g_1.team1 \neq g_2.team1 \land g_1.team1 \neq g_2.team2 \land
  g_1.team2 \neq g_2.team1 \land g_1.team2 \neq g_2.team2)$
  \item Teams in division 1 (A) should all play against every team in division
  2 (B) once.\\
  count($s:S$, $t_1:T$, $t_2:T$) = $+(g \in s | (g.team1 = t_1 \land
  g.team2 = t_2) \lor (g.team1 = t_2 \land g.team2 = t_1) : 1)$\\
  $c_4(s:S) = \forall (t_1 \in D_1, D_1 \in D |: \forall (t_2 \in D_2 |:$ count$(s, t_1,
  t_2) = 1))$
  \item Teams in divisions 3 and onward should only play against teams in
  their own division.\\
  $c_5(s:S) = \forall (D_i \in D, i \in \mathbb{N} | i > 2 : \forall (g \in s |:
  g.team1.division = g.team2.division))$
  \item All games must be within the start and end dates of the season.\\
  $c_6(s:S) = \forall (g \in s |: d_s \leq g.date \leq d_e)$
  \item All constraints should hold for the output schedule $S_o$:\\
  $C = {c_1, c_2, c_3, c_4, c_5}$\\
  $\forall(c \in C |: c(S_o) = true)$
  \item The schedule $S_o$ should also be optimized according the soft
  constants below:
  \begin{itemize}
    \item Minimize the maximum number of games played in a week over the
    season.
    $n_ji$ = number of games played by a team in a week\\
    $m$ = frequency of $n$ occuring over each week of the season\\
    $f(t) = $
    \item Minimize the number of games played on a team's scheduled off day
    according to the team's availability.
  \end{itemize}
\end{itemize}
\end{itemize}

\wss{A module without environment variables or state variables is unlikely to
  have a state transition.  In this case a state transition can only occur if
  the module is changing the state of another module.}

\wss{Modules rarely have both a transition and an output.  In most cases you
  will have one or the other.}

\subsubsection{Local Functions}

\wss{As appropriate} \wss{These functions are for the purpose of specification.
  They are not necessarily something that is going to be implemented
  explicitly.  Even if they are implemented, they are not exported; they only
  have local scope.}

\newpage

\section{MIS of Web Application Framework Module} \label{mWA}

\wss{You can reference SRS labels, such as R\ref{R_Inputs}.}

\wss{It is also possible to use \LaTeX for hypperlinks to external documents.}

\subsection{Module}

web\_app\_framework

\subsection{Uses}

N/A

\subsection{Syntax}

\subsubsection{Exported Constants}

\subsubsection{Exported Access Programs}

\begin{center}
\begin{tabular}{p{2cm} p{4cm} p{4cm} p{2cm}}
\hline
\textbf{Name} & \textbf{In} & \textbf{Out} & \textbf{Exceptions} \\
\hline
\wss{accessProg} & - & - & - \\
\hline
\end{tabular}
\end{center}

\subsection{Semantics}

\subsubsection{State Variables}

None

\subsubsection{Environment Variables}

None

\subsubsection{Assumptions}

\wss{Try to minimize assumptions and anticipate programmer errors via
  exceptions, but for practical purposes assumptions are sometimes appropriate.}

\subsubsection{Access Routine Semantics}

\noindent \wss{accessProg}():
\begin{itemize}
\item transition: \wss{if appropriate} 
\item output: \wss{if appropriate} 
\item exception: \wss{if appropriate} 
\end{itemize}

\wss{A module without environment variables or state variables is unlikely to
  have a state transition.  In this case a state transition can only occur if
  the module is changing the state of another module.}

\wss{Modules rarely have both a transition and an output.  In most cases you
  will have one or the other.}

\subsubsection{Local Functions}

\wss{As appropriate} \wss{These functions are for the purpose of specification.
  They are not necessarily something that is going to be implemented
  explicitly.  Even if they are implemented, they are not exported; they only
  have local scope.}

\newpage

\bibliographystyle {plainnat}
\bibliography {../../../refs/References}

\newpage

\section{Appendix} \label{Appendix}

\wss{Extra information if required}

\newpage{}

\section*{Appendix --- Reflection}

\wss{Not required for CAS 741 projects}

The information in this section will be used to evaluate the team members on the
graduate attribute of Problem Analysis and Design.

\input{../../Reflection.tex}

\begin{enumerate}
  \item What went well while writing this deliverable? 
  \item What pain points did you experience during this deliverable, and how
    did you resolve them?
  \item Which of your design decisions stemmed from speaking to your client(s)
  or a proxy (e.g. your peers, stakeholders, potential users)? For those that
  were not, why, and where did they come from?
  \item While creating the design doc, what parts of your other documents (e.g.
  requirements, hazard analysis, etc), it any, needed to be changed, and why?
  \item What are the limitations of your solution?  Put another way, given
  unlimited resources, what could you do to make the project better? (LO\_ProbSolutions)
  \item Give a brief overview of other design solutions you considered.  What
  are the benefits and tradeoffs of those other designs compared with the chosen
  design?  From all the potential options, why did you select the documented design?
  (LO\_Explores)
\end{enumerate}


\end{document}